\begin{savequote}[105mm]
  \begin{singlespace}
    Educators have been talking about \metaphor{foundations} for so long
that it no longer seems metaphorical---but that is when metaphors
become the most dangerous.  All that \dquote{foundation} literally
means in the context of instruction is something taught initially in
order to facilitate future learning.  This may or may not have
anything to do with foundational ideas of the discipline, but the
metaphor disposes people to prejudge this issue.
\qauthor{\mycite[page 335]{bereiter2005}}
\end{singlespace}
\end{savequote}

\chapter{Metaphor in educational policy}
\label{chapter2} 

\section{Overview}

In this chapter I investigate metaphor in educational planning and
policy, with special reference to undergraduate statistics education.
I discuss metaphor in general, and then show why metaphorical language
is both important and informative in educational policy.  I present a
literature review of research that has been carried out in this area, and
set out a proposal for further work.

\mycite{miller1990} consider metaphor in the context of educational
policy.  They state that metaphors may serve as important clues in
better understanding of the implicit ideological preferences of the
policymakers themselves.

There are three metaphors that appear to be particularly pervasive in
the structuring of undergraduate education: the \metaphor{level}
metaphor, the \metaphor{foundation} metaphor, and the
\metaphor{framework} metaphor.  These metaphors are commonly used in
the context of undergraduate statistics education, and this chapter
gives an overview of these and related metaphors.

\section{The \metaphor{level} metaphor}

The Oxford English Dictionary (henceforth OED) gives a number of
literal senses of the word \dquote{level}, the most germane of which
are \dquote{a horizontal plane} and \dquote{position on a real or
  imaginary scale}.  Much undergraduate mathematics is structured into
\dquote{levels} (see, for example, the New Zealand Qualifications
Framework) that broadly correspond to time spent in further education.
Typically, a first year undergraduate course would be described as
\dquote{level 5}, a second year course as \dquote{level 6}, and so on.
However, it should be noted that the OED does not admit that levels
are discrete, although other dictionaries include \dquote{a floor
  within a multi-storey building} which makes the discretization of
the vehicle explicit.

Note that this metaphor is easily adapted to a constructivist
framework: the students construct the multi-storey building as they
ascend the levels.  However, the metaphor fails in certain key
respects.  Firstly, the content of each level is generally held to be
of equal, uniform, difficulty.  This is questionable at best and, I
would claim, demoralizing at worst.  Secondly, higher levels are
supposed to be of successively greater difficulty; this is unlikely to
be true if \dquote{foundational} mathematics is studied.  And thirdly,
any cohort is imbued with some form of magical elevation from one
level to the next at the beginning of a school year.
\mycite{paechter2004} points out that the \metaphor{level} metaphor
entails that every layer rests on the one before, observing that such
metaphors are temporal rather than spatial.

\mycite{miller1990} consider such metaphors in the context of national
educational policies and point out that such metaphors reflect a
desire for permanence, stability, and predictability.

In the context of statistics education, there are many concepts for
which the discrete nature of levels is at odds with the wide range of
sophisticated needed for their understanding.  One example might be
the concept of statistical independence.  This is considered to be a
\dquote{level 5} concept and indeed the basic definition is readily
understandable: events~$A$ and~$B$ are independent
if~$P\left(A\left|B\right.\right)=P\left(A\right)$.  However, the
concept of statistical independence is notoriously tricky, even for
professional statisticians---with~\mycite{dawid1979} detailing a
number of common fallacies surrounding the distinction between
independence and conditional independence.

\subsection{The International Standard Classification of Education}

Explicit statements on the nature of educational levels appear to be
rare\footnote{The Bologna process~\parencite{keeling2006} discusses
  \dquote{cycles} corresponding broadly to BA, MA, and PhD degrees.}.
However, one of the very few places where the level metaphor is
discussed explicitly is the International Standard Classification of
Education,
ISCED~\parencite{unesco_institute_for_statistics_international_2012}.
This is the \dquote{standard framework used to facilitate
  international comparisons of education systems}.  Item~47 is worth
quoting in full:

\begin{singlespace}
\begin{quote}
\dquote{The notion of \dquote{\emph{levels}} of education is
  represented by an ordered set, grouping education \emph{programmes}
  in relation to \emph{gradations of learning experiences}, as well as
  the knowledge, skills and competencies which each programme is
  \emph{designed} to \emph{impart}.  The ISCED level reflects the
  \emph{degree} of complexity and specialization of the \emph{content}
  of an education programme, from \emph{foundational} to
  complex}---\parencite[item
  47]{unesco_institute_for_statistics_international_2012}
\end{quote}
\end{singlespace}

\noindent
(here, salient metaphorical terms are indicated in italics).  In these
short quotes, ISCED uses the level, foundation, and framework policy
in concert.  All three are examples of the conceptual metaphor
\conmet{ideas are buildings}; but note that the first and second are
also examples of orientational metaphors, in this case \conmet{up is
  good}; and the third is---arguably---an example of \conmet{the mind
  is a machine}; or just possibly \conmet{theories are objects}.
ISCED seems to be aware that the \metaphor{level} metaphor is indeed
only a metaphor.  Item~48 reads:

\begin{singlespace}
\begin{quote}
  \dquote{Levels of education are therefore a construct based on the
    assumption that education programmes can be grouped into an
    ordered series of categories.  These categories represent broad
    steps of educational progression in terms of the complexity of
    educational content.  The more advanced the programme, the higher
    the level of education}---\parencite[item
    47]{unesco_institute_for_statistics_international_2012}
\end{quote}
\end{singlespace}

\noindent
Documents such as the New Zealand Qualifications Framework employ the
level metaphor extensively, presenting tables of properties of study
from level~1 (certificate) through level~10 (PhD).

The other discrete orientational metaphor used in this context is that
of a \metaphor{taxonomy}, the most prominent examples of which are
Bloom's \parencite{anderson2001} and SOLO~\parencite{biggs1982}.
However, there does not appear to be any connection between the
taxonomy metaphor and the level metaphor as used in education.

\section{The \metaphor{foundation} metaphor}

For \emph{foundation}, the OED gives \dquote{the solid ground or base
  on which an edifice or other structure is erected} and the word is
often used to refer to mathematics content that is more basic or
fundamental than other material.  In this context the phrase is an
instantiation of the \conmet{theories are buildings} conceptual
metaphor.

The term \dquote{foundation} in the context of statistics education
has two meanings: firstly, it refers to statistical knowledge and
techniques that are frequently assumed knowledge in more advanced
courses; and secondly, it refers to the foundations of the discipline,
usually meaning the relationship between statistical reasoning and the
more fundamental science of probability.

\mycite{bereiter2005} gives a disarming, yet devastating, observation on
the first sense of \metaphor{foundation} (itself rich in metaphor):

\begin{singlespace}
\begin{quote}
  \dquote{But the insidious effect of the foundation metaphor does not
    stop there.  No builder would construct a foundation without
    having a pretty clear idea of the building to be erected upon it;
    only a subcontractor would do that.  Beginning students, having no
    way to foresee the eventual structure of knowledge, are
    therefore cast into the role of subcontractors}---\mycite[page
    336]{bereiter2005}
\end{quote}
\end{singlespace}

\noindent
The two operative conceptual metaphors, specifically \conmet{up is
  good} and \conmet{theories are buildings}, embody a spatial
contradiction in the sense that \metaphor{foundations} are the lowest
level of a building, yet are often held to be of the
\metaphor{highest} importance.  It is perhaps worth pointing out that,
to the professional mathematician or physicist, those working in the
foundations of the discipline enjoy the highest status in the
profession: \cite[pages 469--571]{mcculloch2013} talk of the
\dquote{high status} and \dquote{great prestige} of pure mathematics
when compared with applied; note that there is no equivalent of the
\emph{Apology}~\parencite{hardy1940} for applied mathematics (or
indeed statistics).

\subsection{Foundational statistics}

The term \dquote{foundational statistics} usually refers to study of
the relationship between probability and statistics.
Probability---itself on a shaky and arguably meaningless logical
footing\footnote{The hugely influential treatise
  of~\mycite{definetti1975} famously begins with the provocative
  statement that PROBABILITY DOES NOT EXIST (the intended sense was
  that probability has no objective meaning).  Many subsequent
  authors, notably~\mycite{nau2001}, quote this rather subversive
  assertion with approval, retaining the startling capitalization of
  the original.}---has its roots in pure mathematics, and I consider
metaphor usage in mathematics in Chapters~\ref{chapter4}
and~\ref{chapter5}.

The \metaphor{foundation} metaphor might suggest that foundational
statistics is somehow more fixed, more solid, or more
well-established, than other branches of statistics.  One could
reasonably demand that \dquote{foundations} of any discipline be firm.
How can a study of statistics be built on anything but the most sturdy
of fundaments?

Even a cursory study of foundational issues in statistics reveals two
surprising features: firstly, the large number of mutually exclusive
and inconsistent statistical principles in common
use~\parencite{edwards1984}; and secondly, the deficiencies and
unavoidable contradictions of inferential statistics as practised in
the applied sciences~\parencite{wasserstein2016}.  In the context of
statistics and statistics education, \mycite{robins2000} consider this
and observe that there is no agreement on which principles are
\dquote{right}, nor on which should take precedence over others.  Thus
statistical education must be structured in such a way that these
difficulties are obscured by pedagogical strategies.

\section{The \metaphor{framework} metaphor}

For \dquote{framework}, the OED gives \dquote{a structure made of
  parts joined to form a frame; esp. one designed to enclose or
  support; a frame or skeleton}, with senses supporting this use in an
industrial or horticultural context.

In an educational policy, \dquote{framework} usually refers to an
organized set of standards, aims, or learning objectives that loosely
specify the type of material to be learned.  This is frequently used
to support the \metaphor{level} and \metaphor{foundation} metaphors.
However, frameworks are generally held to be rigid and inflexible
structures and the effect of the framework metaphor is not necessarily
beneficial.

\mycite{paechter2004} observes that the framework metaphor, along with
its close relation the \metaphor{scaffolding} metaphor, is an exemplar
of a wider class of structural spatial metaphors.
\citeauthor{paechter2004} goes on to speculate that these metaphors
are unusual in that they explicitly privilege space over time.

\section{Metaphors in educational policy}
The value of metaphor has been clear to practising educators for a
very long time; \mycite{cameron2003} observes that metaphors are
peculiarly susceptible to being misinterpreted in a classroom context
but emphasizes the fact that education simply cannot function without
them.

In a wide-ranging review, \mycite{botha2009} considers the epistemic and
ideological freight carried by metaphor in an educational context; yet
she omits entirely any mention of metaphor in educational policy.  One
of the very few scholarly writings to consider metaphor's role in
educational policy per se is that of \mycite{bessant2002}.  Bessant
considers the political rhetoric surrounding an influential period of
educational reform in Australia, focusing on the use of metaphorical
language.  Like \mycite{charteris-black2004} and \mycite{deignan2005},
Bessant considers metaphor as a persuasive device, but emphasizes
metaphor's ability to inform our thinking without us being aware of
its influence.

There are two further metaphors that appear in connection with
education: the {\em factory} metaphor, and the {\em acquisition
  metaphor} of \mycite{sfard1998}.  I discuss each in turn below.

\subsubsection*{The factory metaphor}

By far the best-known educational metaphor is the factory metaphor.
\mycite{claxton2013}, following \mycite{toffler1990}, draws several
paragraphs of analogies between modern schools and production lines:
cohorts become batches; (educational) standards and indeed examination
grading become quality control; and so on.  Mass production of
literate, honest, punctual and dutiful workers was conceived of in
exactly the same way as mass production of anything else.  Although
\citeauthor{claxton2013} did not actually use the term
\dquote{conceptual metaphor} here, he emphasized elsewhere the power
of metaphor to guide and sculpt thinking.

\subsubsection*{Participation vs acquisition}

No study of metaphor in educational theory would be complete without
mentioning the work of \mycite{sfard1998}, who points out that
\dquote{human learning [has always been] conceived of as an
  acquisition of something}.  She goes on to develop this acquisition
metaphor, observing its ubiquity in educational discourse, and its
implicit comparison of education with accumulation of material wealth.
\citeauthor{sfard1998} then posits a new \metaphor{participation}
metaphor, on the grounds that education is something one {\em does},
rather than {\em gets}.

This dichotomy has proved fruitful and persistent: \mycite{wegner2015},
for example, point out that the very existence of tuition fees and
credit points highlights the acquisition metaphor and actively
discourages the participation metaphor.  They go on to observe that
the acquisition metaphor is predicated on the assumption that
knowledge can be seen as an entity; the appropriate conceptual
metaphors are \conmet{ideas are objects} and \conmet{the mind is a
  container}.

Note too that the acquisition metaphor is neutral with respect to
constructivism: learners may either receive knowledge entities or
actively construct them.

Expressions like \dquote{knowledge \metaphor{transfer}},
\dquote{intellectual \metaphor{property}} or
\dquote{\metaphor{grasping} ideas} show how deeply engrained this
metaphor is in western language.  The acquisition metaphor includes
both transmissive views (the assumption that knowledge can be passed
by transmission from one person to the other) and constructivist views
(knowledge is constructed individually by each person), because both
conceptualize knowledge as an entity.

\mycite{sfard1998} considers these issues and, using the central
thesis of \mycite{lakoff1980}, applies them in the context of
educational policy.  She leaves us in no doubt that metaphors are
important and influential: \dquote{Different metaphors lead to
  different ways of thinking and to different
  activities}~\parencite[page 5]{sfard1998}.

\section{Conclusions}

Metaphor thus appears to be threaded through educational policy, and
exerts a powerful yet hidden effect on our thought.  Our unthinking
use of the \metaphor{level} metaphor, for example, normalizes the
notion that mathematics and indeed statistics has discrete units of
ascending difficulty.

%% following three epigraphs refer to the same text, with increasing
%% tersification.


%\epigraph{The educational discourse in England has thus become so
%  dominated by metaphors of height-privileged hierarchical space that
%  almost everything now operates in relation to it, including
%  teachers' and students' views of themselves and each other, the
%  operation of schools (targeted mentoring to enable students to reach
%  the benchmark) and the relationship between parents and teachers
%  (the last time I went to a parents' consultation evening, the first
%  thing my eight-year-old's teacher said about him was that he was in
%  the top groups for mathematics and English, as if this was the most
%  important thing to report about his education).  This is a clear
%  example of the way spatial metaphors of schooling can capture us,
%  almost unthinking, in particular, pernicious,
%  discourses}{\mycite{paechter2004}}

%\epigraph{The educational discourse in England has thus become so
%  dominated by metaphors of height-privileged hierarchical space that
%  almost everything now operates in relation to it, including
%  teachers' and students' views of themselves and each other, the
%  operation of schools (targeted mentoring to enable students to reach
%  the benchmark) and the relationship between parents and
%  teachers\ldots spatial metaphors of schooling can capture us, almost
%  unthinking, in particular, pernicious,
%  discourses}{\mycite{paechter2004}}

\begin{singlespace}
\epigraph{The educational discourse in England has thus become so
  dominated by metaphors of height-privileged hierarchical space that
  almost everything now operates in relation to it, including
  teachers' and students' views of themselves and each other\ldots
  spatial metaphors of schooling can capture us, almost unthinking, in
  particular, pernicious, discourses}{\mycite[page 458]{paechter2004}}
\end{singlespace}
