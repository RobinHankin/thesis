\begin{singlespace}
\begin{savequote}[105mm]
For a very long time, word problems have played
  their role as an unproblematic and transparent bridge between the
  world of mathematics and the real world.
\qauthor\mycite[page 644]{verschaffel2014}
\end{savequote}
\end{singlespace}

\chapter{Metaphor in assessment}
\label{chapter6}

\section{Overview}

In this chapter, I consider metaphor in the assessment phase of an
undergraduate statistics course.  Metaphor occurs in both the
assessment cue and the student's response and I consider both
separately below.

Metaphor is common in statistics examination questions, specifically
occurring when (proper) placeholder nouns are used in word problems.

I also briefly discuss students' use of metaphor when being assessed,
considering both controlled assessment (examination) and uncontrolled
(portfolio) assessment items.

\section{Essay-type questions}

Essay-type questions {\em per se} are rare in undergraduate
statistics, with one exception: a relatively open-ended cue to analyze
a dataset using the methods taught in the course, and communicate any
findings.  \mycite{horowitz1986} considers the use of language in such
assessment cues, with a typical exam prompt being along the lines of

\begin{quote}
  \emph{Using the Yanomamo as an example briefly explain Marvin
    Harris's theory of primitive warfare}
\end{quote}
  
\noindent\citeauthor{horowitz1986} went on to identify a number of
\dquote{micro-functions} of such cues which characterize acceptable
responses.  He ordered these micro-functions along an axis from
content-oriented (\dquote{identify the topic}) to form-oriented
(\dquote{specify the length of the essay}).  From the perspective of
undergraduate statistics education, the most germane micro-function
was his number 5: \emph{specify the writer's persona}.  For a typical
written assignment, \mycite{horowitz1989} points out that students
must pretend that the marker is not familiar with the issues
discussed.

However, in the context of undergraduate statistics, a typical
assignment might be to analyze a specific dataset using algebraic and
visual methods.  In this situation, a student need only make the
realistic assumption that the marker has not actually carried out such
analysis.

Is \emph{metaphor} part of this aspect of language use?  I would argue
that typical undergraduate statistics assessments do use metaphoric
language, in several senses.  Firstly, the students generally treat
the dataset provided with the assessment as just one representative of
an ensemble of possible datasets, all of which would elicit identical
statistical analyses: they would perform the same steps if the data
were perturbed slightly.  The data is thus meronymically defined.
Secondly, the student understands that the analysis is not actually
important and the premises of the cue are merely a plausible fiction
which may or may not exist.

\section{Word problems}

A \emph{word problem} is a verbal description of a problem situation
wherein one or more questions are raised, and the answer to which can
be obtained by the application of mathematical operations to numerical
data available in the problem statement~\citep{verschaffel2014}.  Word
problems are \dquote{considered to be an important part of mathematics
  education}~\citep{reusser1997}.

\mycite{gerofsky1996} states that the overwhelming majority of word
problems have three components: firstly, a backstory which establishes
the characters and possibly the location of the putative story; an
information component, in which the information needed to solve the
problem is given; and a question.

However, word problems are a problematic assessment item in terms of
educational value~\citep{gerofsky1996},
transferability~\citep{reusser1997}, and low achievement
rate~\citep{cummins1988}.  Given these concerns, it is somewhat
surprising that \mycite{johnson1976}, in a 166-page book devoted
purely to the solution of word problems, offers not the slightest
motivation for their study.

In the context of statistics education, \mycite{quilici1996} show that
word problems encourage students to attend to surface elements of the
question (such as inclusion of words such as \dquote{compare} which
induce the use of a two-sample $t$-test) at the expense of underlying
structural features.

In the following sections I consider the extent to which metaphor
occurs in this assessment trope.

\subsection{Truth value and word problems}

Metaphor analysis of word problems is not straightforward because a
word problem is semantically ambiguous.  One concept useful in the
analysis of word problems is \dquote{truth value} as originally
defined by Frege in 1891: the truth value is the attribute assigned to
a proposition in respect of its truth or falsehood.  Frege considered
the relation between propositions and truth from a philosophical
perspective; but the relevance of truth value to language encountered
in fiction remains \dquote{problematic}~\citep{lamarque1994}.

In the context of education, \mycite{gerofsky1996} considers the
semantics of word problems and observes that their truth value has no
clear status.  He observes that word problems may be rephrased without
altering their truth value and offers:

\begin{singlespace}
\begin{quote}
Every year (but it has never happened), Stella (there is no Stella)
rents a craft table at a local fun fair (which does not exist). She
has a deal for anyone who buys more than one sweater (we know this to
be false\ldots there are no people, or sweaters, or prices)
\end{quote}
\end{singlespace}

\noindent As \mycite{gerofsky1996} points out, word problems are a
peculiar trope in which one is instructed to pretend that the
background story is true, under (implicit) direction from the writer
of the problem; but simultaneously the competent reader considers the
background story to be irrelevant.  \mycite{reusser1997} contrast the
chimerical text of the backstory with the implicit web of mathematical
relations in the problem itself; they conclude that these two
\dquote{interwoven semiotic worlds} are poorly aligned and largely
irreconcilable.

The entire backstory may thus be considered to be meronymic in the
sense that the one provided is but one representative of many
possible, functionally identical, backstories.

%Representing two interwoven semiotic worlds, the story-like
%description of non-mathematical real-world situations and an implicit
%web of mathematical relations, mathematical word problems are
%considered to be an important part of mathematics education.  Reusser
%1997

%\begin{enumerate}
%\item that \dquote{this} is solvable,
%\item that \dquote{X} can be found,
%\item that the word problem itself contains all the information needed
%  to do this task,
%\item that no information extraneous to the problem may be sought
%  (apart from conventional mathematical operations which likely must
%  be supplied),
%\item that the task can be achieved using the mathematics that the
%  student has access to,
%\item that the problem has been provided to get the student to
%  practice an algorithm recently presented in their math course,
%\item that there is a single correct mathematical interpretatiof the
%  problem,
%\item that there is one right answer,
%\item that the teacher can judge an answer to be correct or incorrect,
%  and especially,
%\item that the problem can be reduced to mathematical form---in fact,
%  that the problem is at heart an arithmetic or algebraic formulation
%  which has been \dquote{dressed up} in words, and that the student's job
%  is to \dquote{undress} it again---to transform the words back into the
%  arithmetic or algebra that the writer was thinking of, then to solve
%  the problem.
%\end{enumerate}



%\mycite{reusser1997}: \dquote{\ldots most students perceive word problem
%  solving as a puzzle-like activity with no grounding in factual
%  real-world structures and with no relation to a goal-directed, more
%  authentic activity of mathematization or realistic mathematical
%  modelling\ldots at the bottom of the critique of the impoverished
%  nature of word problems is the many-faceted issue of probem
%  formulation and problem posing}.

\mycite{boaler2000} describes one enthusiastic student who, meeting word
problems for the first time, was dismayed to find that bringing her
competent, adult-level situational knowledge to bear on the problems
was counterproductive to academic success.  Boaler went on to question
the (practical) competence of the question-setter, although she does
not query the ontological status---or educational value---of the word
problem.

% Direct quote from Boaler 2000, p392:
%Rose describes feeling particularly alienated when \dquote{real world}
%problems were introduced, with which she enthusiastically engaged
%drawing upon her knowledge of the situations described, only to find
%that such knowledge was not allowed, and that engagement with the
%problems involved a step away, rather than towards the real world
%\ldots school children recognize that school mathematics is not a part
%of the world outside school, partly because of the artificiality of
%school problems.
%
%
% Boaler goes on to quote Rose (which I have not tracked down):
%It was obvious to me that many of the questions simply indicated that
%the questioner did not know enough about the craft skills involved in
%real world solutions.  Lawn rollers being pulled up slopes,
%wallpapering rooms by calculating square feet and inches: these were
%tedious and as far as that highly practical child could see, stupid
%

\section{Students' use of metaphor in assessment}

\begin{singlespace}
\epigraph{Essay examination writing is, indeed, a \dquote{display} for
  the purposes of evaluation, a time to show that one has studied
  hard, not that one is especially clever or possessed of broad
  general knowledge}{\mycite{horowitz1986}}
\end{singlespace}

\noindent
The nature of the language used by students in the assessment phase of
a statistics course is problematic: \mycite{horowitz1986} argues that a
student's response to any assessment cue is a perlocutionary act,
specifically one that persuades the reader to proffer an acceptable
grade in a course.

This viewpoint makes the analysis of metaphor difficult, as the
purpose of the writing is not clear.  On the one hand, the student is
expected to create an exposition to convince the reader of the truth
of a proposition, but on the other, the reader is already convinced.

\mycite{read2001} consider the ways in which students develop a
\dquote{voice} and points out that students must master the complex
culture of academic language in order to succeed.  These authors point
out that the conflict between the desire to score a high grade is
counter to the desire for a student to have their own voice: high-GPA
students were reported to have sought out their tutors' viewpoints in
order to write from their perspective.  \mycite{read2001} go on to pose
the rhetorical question: is such writing the students' voice or that
of their tutors?  

In pure mathematical disciplines, essay-type questions are rare but do
exist~\citep{johnson1983}.  However, by far the most common type of
mathematical examination cue requires the student to prove a (given)
statement.  The linguistic status of a student's proof (under
examination conditions) is again problematic.  The definition of
\dquote{proof} is a logically watertight demonstration of the truth of
a statement: all the student has to do is to reproduce an existing
proof.

However, analogously to an undergraduate essay, an examination proof
is intended to have a perlocutionary effect, specifically of inducing
the reader or marker to give an acceptable grade to the student.  A
poor student will (attempt to) reproduce an existing proof, with no
understanding.  This, however, is a very difficult task as without the
cognitive scaffolding that understanding provides, a proof typically
has no discernible structure and is very difficult to memorize.  The
natural way for a more able student to proceed is to convince the
marker that the writer has actually grokked (sic) the proof and can
convey this to the reader.  In this sense, the mathematics examination
is a peculiar form of performance assessment in which one has to
convince the examiner that you have indeed had the flash of insight
which mathematicians call \dquote{proof}.

Under these circumstances, can the student be said to employ metaphor?
\mycite{kyung2004} argue that students constantly employ mathematical
metaphor in class and, as such, identifies the machine metaphor and
the fictive motion metaphor as dominant metaphors in the case of
partial differential equations.

However, in the context of undergraduate statistics, a typical
assignment might be to analyze a specific dataset using algebraic and
visual methods.  In this situation, a student need only make the
realistic assumption that the marker has not actually carried out such
analysis~\citep{horowitz1986}.

%  [The dichotomy between expository and argumentative essay styles] is
%somewhat muddied by the fact that all essay examination writing
%contains an argumentative element, namely the writer's attempt to
%persuade the reader/instructor to proffer an acceptable grade in a
%course.  This intention must be hidden, however\ldots
%because of the social
%convention that states that an expository exam essay is to be written
%as if the reader did not already understand the ideas being
%presented---as if a prompt were really a question---and that an
%argumentative essay is to be written as if the reader did not already
%agree with the thesis---as if a prompt were really part of a Socratic
%dialogue.  Thus, essay examination writing is doubly false, in that
%writers must hide their true intention (to pass the course) behind a
%wall of prose designed to do what has already been done

%\epigraph{an expository exam essay is to be written as if the reader
%  did not already understand the ideas being presented---as if a
%  prompt were really a question---and that an argumentative essay is
%  to be written as if the reader did not already agree with the
%  thesis---as if a prompt were really part of a Socratic dialogue.
%  Thus, essay examination writing is doubly false, in that writers
%  must hide their true intention (to pass the course) behind a wall of
%  prose designed to do what has already been
%  done}{\mycite{horowitz1986}}

\subsection{The British Academic Written English corpus}

Assessed student writing is difficult to study owing to a scarcity of
suitable corpora~\citep{nesi2004}; and those that exist are typically
focused on English as a second language.

One of the few systematic corpora of assessment is the British
Academic Written English corpus, the BAWE~\citep{bawe2016}.  The BAWE
is a collection of student writing from undergraduate to taught
Masters level, restricted to assignments consistent with an upper
second or first class honours degree~\citep{nesi2012}.  The BAWE is
unique in the wide range of disciplines represented.

The BAWE includes samples of statistics assessment.  It is clear from
context (the anonymization protocol redacted the cue) that the student
was a first year undergraduate required to assess the relative merits
of two measures of central tendency: the mode and the median.  I will
analyse these two pieces of work for metaphor using
the~\mycite{pragglejaz2007} protocol.

The first piece of student work was responding to a cue asking to
characterize the median as a measure of central tendency:

\begin{singlespace}
\begin{quote}
\say{Median is useful in this case because it tells us that half the
  sample has more money than this with them and half has less.  It is
  not influenced by outliers as the mean is.  For example if we had a
  very high value such as \pounds 200 this would increase the mean
  greatly so that it is no longer as representative of the sample as
  the median}---Anonymous student, quoted in \mycite{bawe2016}
\end{quote}
\end{singlespace}

\noindent It is possible to analyze this assessed writing using the
\mycite{pragglejaz2007} metaphor identification protocol;

\begin{description}
\item[\squareb{The} median \metaphor{is useful} in this case]\qquad An
  example of the conceptual metaphor \conmet{equations are tools}.  In
  this case, the metaphor is likely to originate in the course itself.
\item[because \metaphor{it tells} us that]\qquad A peculiar agency
  metaphor; the estimator is personified, and in addition given a
  \dquote{voice}.
\item[half the sample has more\ldots and half less.]\qquad No
  metaphor: this is a literal definition of the median.
\item[It is not \metaphor{influenced by outliers}]\qquad This is
  metaphorical: what the student is trying to say (successfully) is
  that the \emph{value} of the median does not change as a result of a
  marginal change to an outlier.  However, she is here using a
  metonym: the topic is the measure of central tendency known as the
  median, but the vehicle is the value of the median.
\item[as the mean is.]\qquad non-metaphorical.
\item[For example if \metaphor{we} had]\qquad Another example of
  Pimm's \dquote{we}
\item[a \metaphor{very high} value such as \pounds 200]\qquad
  Orientational metaphor: \conmet{increase is up}.
\item[this would increase the mean greatly]\qquad no metaphor
\item[so that it is no longer as representative of the sample as the
  median]\qquad no metaphor
\end{description}
  
\noindent
The second piece of assessed work is from the same student considering
the mode of a dataset.

\begin{singlespace}
\begin{quote}
  \say{The mode is of a varying degree of usefulness---it tells us the
    most common value but here this is very low and not representative
    of the sample values as a whole.  If we look at various major
    peaks in the data set then this is a fairly useful tool.  For
    example, here the data is bimodal---there is a split between
    people carrying a relatively large amount of money and a small
    amount.}
\end{quote}
\end{singlespace}

\begin{description}
\item[The mode is of a varying degree of
  \metaphor{usefulness}---]\qquad \conmet{equations are tools}
\item[\metaphor{it tells us} the most common value]\qquad Clear
  personification metaphor.
\item[but here this is very low]\qquad Orientational metaphor
  (\conmet{less is down})
\item[and not representative of the sample values as a whole.]\qquad
  Non-metaphoric
\item[If \metaphor{we} look at]\qquad Another example of Pimm's
  \dquote{we}
  \item[various major peaks \metaphor{in} the data set] Possible
    conceptual metaphor \conmet{data is a container and features of
      the data objects in it}.  This is a reasonably common metaphor.
\item[then this is a fairly useful tool.]\qquad\mycite{lakoff1980}
  state that explicit use of conceptual metaphors is rare but here is
  an example of \conmet{equations are tools} being used explicitly.
\item[For example, here the data is bimodal---]\qquad Note that
  \conmet{data is a container and features of the data objects in it}
  is not being used here: the bimodality is attributed directly to the
  dataset rather than asserting that the feature is contained inside
  it.
\item[there is a split between people carrying a relatively]{\bf large
  amount of money and a small amount.}\qquad Metaphoric use of
  \metaphor{split}.  Here, \dquote{split} is being used in the sense
  of chasm or divide: the datapoints (here people) are placed on a
  number line (itself a metaphor for the real numbers) and the student
  is observing that there is a region along this line that is sparsely
  populated.

  In this case, there is one metaphor that is conspicuous by its
  absence.  The data comprised a finite number of observations, each
  one being the amount of money carried by a \emph{specific} person in
  the sample.  The student is not metonymically referring to a person
  by the amount of money they were carrying: the student refers
  directly to the split between the people.
\end{description}

\noindent These pieces of assessed writing use metaphor in a routine
and unremarkable manner.  The metaphors used appear to be of the same
general type as used by educators in typical course resources.

\subsection{Measures of central tendency in undergraduate statistics}

Calculating measures of central tendency is a distressingly common
trope in undergraduate statistics education to the extent that one
sees derogatory descriptions of \dquote{mean-median-mode} education.
It is an easy matter to assess understanding of these measures by
requiring students to calculate the three measures for different
datasets.

From the perspective of a practising statistician, the mean and mode
are simply measures of central tendency of a sample\footnote{In the
  assessed writing above, the student was discussing the sample mean
  and sample mode.  The sample mean is the arithmetic mean of one's
  observations and is useful because it is an estimate of the
  population mean (if defined).  The mode is an altogether more
  problematic measure, having different definitions for continuous and
  discrete distributions.} and do not have \dquote{usefulness}.  The
concept of one measure being more \dquote{representative} than
another is meaningless: the measures summarize a sample in different
ways and illustrate different aspects of the sample.  The situation is
analogous to an engineer asking whether the radius is more
\dquote{representative} of the size of a given a circle than the
diameter.  So in this case the conceptual metaphor \conmet{equations
  are tools} is leading to flawed thinking.

But it should be noted that my comments above---correct as they are
from a theoretical statisticians' perspective---would probably not
attract a passing grade at undergraduate level: as I argue elsewhere,
deep thinking about any aspect of inferential statistics inevitably
interferes with successful assessment.

\section{Conclusions}

In assessed work, students do employ metaphor when enrolled in
statistics courses.  However, ethical considerations mean that there
is very little assessed work that may be analysed for metaphorical
language: fully informed consent is difficult to obtain.

The British Academic Written English corpus is one of the very few
corpuses of assessed student writing available, and this contains two
pieces of assessed work which were part of a statistics course.

The metaphors employed by the students were broadly comparable with
those used in statistics textbooks and spoken statistics lectures.
The students' metaphors were similar in terms of frequency and
intensity to those used in the different instructional materials
available.

\vfill

\begin{singlespace}
\setlength{\epigraphwidth}{.7\textwidth} % default is .4 
\epigraph{As educators, we usually do not notice [metaphors such as
    \dquote{fall behind}] because they have been buried for so long in
  our cognitive framework that they are literally (sic) beyond notice.
  But we should notice them.  We should notice how words like
  \emph{top} and \emph{bottom} contain a specific understanding of
  assessment, one based almost entirely on comparisons between
  students rather than on whether each student has achieved the
  objectives of the unit of study at hand}{\cite[page 6]{badley2012}}
\end{singlespace}
