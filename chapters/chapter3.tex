\begin{singlespace}
\begin{savequote}[105mm]
The framework [for higher education qualifications] should be regarded
as a framework, not a straitjacket\qauthor{Quality Assurance Agency
  for Higher Education, \citeyear{fheq2008}, page 3}

[W]hen there are explicit culturally sanctioned warnings not to do
something, you can be sure that people are doing it.  Otherwise there
would be no point to the warnings.\qauthor{\mycite[page
    164]{lakoff2000}}
\end{savequote}
\end{singlespace}

\chapter{Metaphor in educational administration}
\label{chapter3}

\begin{singlespace}
\setlength{\epigraphwidth}{.7\textwidth} % default is .4 
\epigraph{Metaphors, and their frequency of use, may serve as
  important clues not only in better understanding the stated intent
  of the policy but also the implicit ideological preferences of the
  policymakers themselves.  In our view, the use of metaphorical
  expressions in major policy statements reflects a largely
  unconscious process whereby implicit beliefs, attitudes, and
  ideological presuppositions concerning the desirability or utility
  of a course of action are made explicit}{\mycite[page 68]{miller1990}}
\end{singlespace}

\section{Overview}

In education a \emph{course descriptor} is a terse, self-contained
specification for a unit of study; a \emph{syllabus} lists the
specific course requirements a student must complete.  The term
\emph{curriculum} is usually reserved for the entirety of student
experience while attending the institution.

In this chapter I will consider metaphor in course descriptors, using
first year statistics course from AUT and the University of Cambridge
as examples.

\subsection{The Metaphor Identification Procedure}

In this thesis, I will analyse language (text and speech) using the
metaphor identification protocol (MIP) of~\mycite{pragglejaz2007}.
This is a formal procedure in which metaphor may be identified;
\dquote{\citeauthor{pragglejaz2007}} is the name of a group of
scholars.

Slightly paraphrased, the MIP is as follows:
\begin{enumerate}
\item Establish a general understanding of the text's meaning
\item Determine appropriate lexical units for analysis
\item For each lexical unit, establish whether the contextual meaning
  is the same as the \emph{basic meaning}, which is effectively the
  meaning of the lexical unit when used in isolation
  \item If the basic meaning differs from the contextual meaning, mark
    the unit as metaphorical.
\end{enumerate}

The term \dquote{metaphor} is not used in its literary or poetic
sense, and the procedure identifies many metaphors that would not be
recognised as such by ordinary listeners or readers without prompting.
Also, the MIP does not address the issue of \emph{intent:} the speaker
or writer's intentions are not part of the procedure.  This procedure,
while somewhat subjective, has been used with some success on a
variety of spoken and written sources.

\section{Introduction}

The syllabus and course descriptor are an ideal place to investigate
metaphor usage in education: they are terse, short documents, with an
intended audience comprising both educators and students.  Course
descriptors are official documents, representing a \dquote{legal
  contract of sorts between academy and student}~\parencite{luke2013}

\mycite{obrien2009} consider syllabus from a learning perspective, with
a strong emphasis on students' taking responsibility for their own
learning.  They observe that a syllabus can serve a wide variety of
functions that will support, engage, and challenge students; and it
can establish an early point of contact and connection between student
and instructor.

\mycite{rubin1985}, however, takes a rather pessimistic view of the
institutional course syllabus, mentioning \dquote{badly thought-out
  efforts}.  She classifies syllabuses into \emph{listers} which
emphasize reading lists and topics covered (rather than selection
principles); and \emph{scolders} which give little more than lengthy
sets of instructions detailing what will happen if a student commits
any of a list of minor misdemeanours such as handing work in after the
deadline.

In the field of undergraduate statistics education, many syllabuses
appear to be somewhat unhelpful.  Take The University of Sydney's
\paper{STAT5002}, for example.  This promises to introduce students to
\dquote{basic statistical concepts} and develop
\dquote{computer-oriented estimation procedures}.

Observe that in the absence of any contextual information, these
phrases are highly ambiguous.  \dquote{Basic statistical concepts},
for example, might be the use of histograms, or Fisher sufficiency;
\dquote{computer-oriented estimation procedures} might be using
\texttt{t.test()} in an R session, or writing recursive Archimedean
copul\ae.  It is only when accounting for the fact that
\paper{STAT5002} is a first-year service course that one would be able
to say what sense these phrases are being used; but if this is the
case, what is the point of a syllabus at all?

\section{Auckland University of Technology}

Auckland University of Technology (AUT) is the newest of New Zealand's
universities, having attained university status in 2000.  AUT has the
lowest QS ranking of the six and its promotional literature emphasizes
the practical nature of the study and the employability of its
graduates.

AUT's \paper{STAT500} paper is a first-year mandatory course in
introductory statistics, intended to ensure that second-year science
subjects have appropriate statistical underpinning.  The course
descriptor has several sections, the most salient of which is the
informal course description: 

\begin{singlespace}
\begin{quote}
  An introduction to applied statistics.  Provides techniques for
  describing and summarizing a data set.  Delivers an understanding of
  how to use a sample to infer the properties of the population it was
  taken from.  Students in this course also learn how to use
  statistical software to undertake descriptive analysis as well as
  perform statistical inference.
\end{quote}
\end{singlespace}

The metaphors in this course description may be identified by
following the metaphor identification protocol (MIP)
of~\mycite{pragglejaz2007}.  There are several groups of words that
are used possibly metaphorically:

\begin{description}
\item[introduction] The OED actually gives this sense of the word
  (\dquote{bringing into use or practice}) as the first sense.
  Conclusion: not metaphorical (or, at best, lexicalized metaphor)
\item[applied] the OED gives \dquote{put to practical use}; not
  metaphorical
\item[provides; delivers] Clearly metaphorical.  The notion that a
  course \metaphor{provides} anything is a clear example of the
  conduit metaphor, \conmet{communication is
    transfer}~\citep{reddy1993}; this is supported by
  \metaphor{delivers} as used in the next sentence.  Together these
  suggest that the student is a passive recipient of information,
  specifically information which is delivered by the institution.  The
  form of words specifically excludes the constructivist approach.

  Authors such as~\mycite{yoon2011} consider the \dquote{transmission
    model} in the context of undergraduate mathematics lectures,
  associating it with student passivity.  \mycite{krippendorff1993}
  concurs, offering \dquote{passive and uninformed receivers or
    consumers} as a side-effect of the conduit metaphor.
\item[use; infer; population] metaphorical but part of the
  disciplinary jargon; see Chapter~\ref{chapter7}
\item[in (this course)] The use of the preposition \dquote{in}
  suggests that the cognitive metaphor \conmet{courses are containers}
  is being used here.
\end{description}

It is clear that this particular course descriptor is rich in
metaphor, which reveals underlying assumptions and attitudes to
education.

\section{The University of Cambridge}

The University of Cambridge is consistently ranked as one of the
world's best universities and some measure of its status is indicated
by the fact that its alumni include 95 Nobel laureates.  Undergraduate
mathematics at Cambridge includes a substantial amount of compulsory
statistics education.  The course descriptor for the first year
statistics course reads, in part:

\begin{singlespace}
\begin{quote}
The course introduces the basic ideas of probability and should be
accessible to students who have no previous experience of probability
or statistics.  While developing the underlying theory, the course
should strengthen students' general mathematical background and
manipulative skills by its use of the axiomatic approach.  There are
links with other courses [a list is given].  Students should be left
with a sense of the power of mathematics in relation to a variety of
application areas.  After a discussion of basic concepts [list] the
course studies [list] \ldots Through its treatment of discrete and
continuous random variables, the course lays the foundation for the
later study of statistical inference.
\end{quote}
\end{singlespace}

\noindent
This text is rich in metaphor.   In the list below, I discuss the
metaphorical language used, with the exception of the
\metaphor{agency} metaphor, which is discussed at the end.

\begin{description}
\item[The course \metaphor{introduces}]\qquad (agency metaphor);
  also an example of \conmet{ideas are objects}.  Note the active
  voice here, contrasted with the passive voice (\dquote{the course
    introduces}) used by AUT.
\item[the \metaphor{basic} ideas of probability]\qquad the OED
  gives \dquote{fundamental, essential}.  Note the absence of the word
  \dquote{foundation} here: the authors presumably wished to avoid
  confusion with foundational statistics, a separate academic
  discipline (not one typically taught at undergraduate level).
\item[and should be \metaphor{accessible}]\qquad clearly an
  orientational metaphor.  The basic meaning of \dquote{accessible}
  given by the OED is \dquote{readily approached or reached}; however,
  the contextual meaning is that the course can be understood by a
  student with no specialized knowledge.
\item[to students who \metaphor{have} no previous
  experience]\qquad\conmet{abilities are entities}; \conmet{knowledge
  is a possession}.  The word \emph{have} is being used
  metaphorically: familiarity with mathematics is expressed using the
  language of ownership.
\item[of probability or statistics]\qquad Both words used
  metonymically for \dquote{formal courses of study covering
    probability or statistics}.
\item[while \metaphor{developing}]\qquad The OED gives \dquote{to
  bring out what is implicitly contained}.  But note the inherent
  ambiguity in the contextual meaning: is it the theory that is being
  developed, or the student's understanding of it?  If the former,
  this is metaphorical usage: there is no ready sense in which
  underlying theory is implicitly contained in anything.  If the
  latter, the word is being used literally (although then one would
  arguably be using the conceptual metaphor \conmet{the mind is a
    container}).
\item[the \metaphor{underlying} theory]\qquad Underlying:
  \dquote{lying under or beneath}.  Possibly an example of
  \conmet{sophisticated is up} or just possibly \conmet{theories are
    buildings}
\item[the course should]\qquad (agency metaphor)
\item[\metaphor{strengthen}]\qquad\conmet{theories are constructed
  objects}; in particular, \mycite{kovecses2010} offers
  \conmet{abstract stability is physical strength}.
\item[\metaphor{students'}]\qquad The apostrophe would indicate
  \conmet{knowledge is a possession} again.  Observe its placing: the
  possessor is plural, suggesting that the students as a group (in
  contrast to individual students) somehow possess the knowledge in
  question.
\item[mathematical \metaphor{background}]\qquad An orientational
  metaphor; the course is being metaphorically identified with an
  object, to which previous mathematical learning is cast as a
  \dquote{background}. \mycite{lakoff1980} observe that the use of
  such terms imputes two features on the course: firstly, the course
  is rendered as an object; and secondly, it imposes an
  anthropomorphic orientation on it.  In this case the course
  \emph{per se} is not only personified but given an orientation
  facing the speaker (student?).
\item[and \metaphor{manipulative} skills]\qquad\conmet{ideas are
  objects}.  In mathematical education, one so often encounters the
  word \dquote{manipulate} (or various declensions thereof) that it is
  easy to forget that the word is being used metaphorically.
  \mycite{kovecses2010} lists \conmet{control is holding something in
    the hand}.  Note that the word \dquote{skill} is being used
  \emph{literally} here.  Skill is not necessarily adroit physical
  manipulation of objects---one can be a skilled orator or investment
  banker (or indeed educator).
\item[\metaphor{by its use of}]\qquad(agency metaphor)
\item[the axiomatic \metaphor{approach}.]\qquad Clearly an orientational
  metaphor, but does not appear to be systematic.
\item[There are \metaphor{links} with other
  courses\ldots]\qquad\conmet{courses are objects}, or possibly
  \conmet{theories are objects}.  In this case \metaphor{link} is
  understood in its sense of a component of a chain; a rare example of
  a non-hierarchical vehicle used to describe relationships between
  mathematical ideas.
\item[Students should be \metaphor{left with}]\qquad\conmet{the mind
  is a container}.  This phrase suggests very explicitly that the
  student is a passive receptacle of information.  Also \conmet{a
    course is a journey}: in this case the student is visualized as
  being stationary while the course moves past him or her; hardly
  conducive to a constructionist attitude.
\item[a \metaphor{sense} of]\qquad A lexicalized metaphor.  Here,
  \metaphor{sense} is being used in sense~17 of the OED:
  \dquote{Mental apprehension, appreciation}, or possibly sense~13:
  \dquote{more or less vague perception or impression}.
\item[the \metaphor{power} of mathematics]\qquad In modern usage,
  largely lexicalized.  Possibly \conmet{ideas are objects} or
  \conmet{theories are machines}.  The OED explicitly lists mental
  strength and effectiveness in the primary sense for \dquote{power}
  but the metaphorical interpretation is one of personification.
\item[in relation to a variety of \metaphor{application}]\qquad
  Lexicalized metaphor.  The sense intended is that the ideas
  presented in course are useful in other (perhaps non-mathematical)
  disciplines such as medicine or physics.
\item[areas.]\qquad clearly metaphorical: \conmet{theories are
  containers}.
\item[After a \metaphor{discussion}]\qquad Not quite metaphorical but
  not literal either.  This is arguably a personification metaphor
  (the course does not \metaphor{discuss} anything);
  \mycite{knowles2006} would classify it as a nominalization metaphor,
  observing that nominalization generally tends to de-emphasize human
  agency, while \mycite{miller1990} observe that personification
  metaphors contribute towards a sense of real individuals with real
  concerns.
\item[of \metaphor{basic} concepts]\qquad as above
\item[\metaphor{the course studies}\ldots]\qquad (agency metaphor)
\item[Through its treatment]\qquad Personification again, this time in
  the passive voice.
\item[of discrete and continuous \metaphor{random variables}]\qquad
  The phrase \dquote{random variables} is another lexicalized
  metaphor, occurring frequently in learning resources and research
  literature alike.  It is discussed at length in
  chapters~\ref{chapter3} and~\ref{chapter4}.
\item[The course lays the foundation]\qquad (agency metaphor); also
  \conmet{theories are buildings}, as discussed in~\ref{chapter5}.
\item[for the later study of statistical inference]\qquad Note the
  implicit assumption the the course in question is merely a
  pre-requisite for more advanced study.
\end{description}

\noindent One notable conceptual metaphor in the above course
description is the \emph{agency} metaphor: the course itself is
ascribed agency.  \mycite{tourish2012} consider agency metaphors as
language that portrays events as volitional actions that reflect
internal mental states, and this seems appropriate in the present
context.  \mycite{lakoff1980} might categorize this such agency
metaphors as \conmet{controlled for controller}, whereas
\mycite{knowles2006} suggest that ascribing agency directly to a
controlled item is a \metaphor{personification} metaphor.


%Learning outcomes By the end of
%this course, you should:
%\begin{itemize}
%  \item understand the basic concepts of
%probability theory, including independence, conditional probability,
%Bayes' formula, expectation, variance and generating functions;
%\item be familiar with the properties of commonly-used distribution
%  functions for discrete and continuous random variables;
%\item understand and be able to apply the central limit theorem.
%\item be able to apply the above theory to ``real world'' problems,
%  including random walks and branching processes.
%\end{itemize}


%Statistics is the study of what can be learnt from data.  We regard
%our data as realisations of random variables, and consider models for
%the (joint) distribution of these random variables.  In this course,
%we focus entirely on parametric models, where the class of
%distributions considered can be indexed by a finite-dimensional
%parameter.  As a simple example, the family of normal distributions
%can be indexed by a two-dimensional parameter, representing the mean
%and variance.  Nonparametric models are treated in more advanced
%courses.  Our aim is to make inference about the unknown parameter by,
%for example, providing a point estimate, a confidence interval or
%conducting a hypothesis test.  Building on Part IA Probability, this
%course will present basic techniques of inference, together with their
%theoretical justification.  The final chapter will cover the
%ubiquitous linear model, with its elegant theory of orthogonal
%projection and application of results from linear algebra.  The most
%appropriate book for the course is Statistical inference by Casella
%and Berger (Duxbury, 2001).
%
%
%Learning outcomes
%By the end of this course, you should:
%\begin{itemize}
%\item understand the basic concepts involved in point estimation, the
%  construction of confidence intervals and Bayesian inference;
%\item understand and be able to apply the ideas of hypothesis testing,
%  including the Neyman-Pearson lemma, and generalised likelihood ratio
%  tests, including applications to goodness of fit tests and
%  contingency tables.
%  \item understand and be able to apply the theory of the linear
%    model, including examples of linear regression and one-way
%    analysis of variance.
%\end{itemize}
%

%\section{UNC, Chapel Hill}
%
%University of North Carolina, STOR 555: \dquote{Functions of random
%  samples and their probability distributions, introductory theory of
%  point and interval estimation and hypothesis testing, elementary
%  decision theory}.  Just a list!


\section{Graduate outcomes}

State support for tertiary education is often justified by the belief
that graduates have superior cognitive abilities to non-graduates, and
such superiority will result in a more effective workforce.

Such beliefs are often embodied in terms of institutionalized
\emph{graduate outcomes}, brief idealized statements of the qualities
that successful graduates are expected to possess.

Graduate outcomes, in general, make heavy use of \conmet{abilities are
  possessions}, also reinforcing the \conmet{education is acquisition}
metaphor of \mycite{sfard1998}.  Australia's Higher Education Council,
for example, explicitly state that graduate outcomes are \dquote{the
  personal attributes and values which should be \metaphor{acquired}
  by all graduates\ldots} (my emphasis).  Such wording echoes course
descriptors (chapter~\ref{chapter2}) in their promotion of the
acquisition metaphor at the expense of the participation metaphor.

\section{Conclusions}

Metaphor occurs frequently in educational administration, with the
metaphorical language being a pervasive component of the course
descriptors chosen for detailed analysis.  Pre-eminent among the
metaphors used in course descriptors is the agency metaphor: a course
is held to possess agency and as such is personified.  Such metaphors
help to create an impression of \dquote{a sense of real individuals
  with real concerns}.

Metaphor occurs in other parts of educational planning, notably
institutional \emph{graduate outcomes}.  The metaphorical language
appears to promote the acquisition metaphor at the expense of the
participation metaphor.


\begin{singlespace}
\epigraph{Metaphors are important both as indicators of the ways we
  think and as rallying-cries for particular
  worldviews}{\mycite[page 450]{paechter2004}}
\end{singlespace}
