
\begin{singlespace}
\begin{savequote}[105mm]
Metaphor is used repeatedly [in undergraduate lectures]\ldots but
there are few elaborated or developed metaphors; those there are tend
to be short, unconnected with later metaphors and used primarily to
serve local, rather than global purposes.  \qauthor{\mycite[page 428]{low2008}}
\end{savequote}
\end{singlespace}

\chapter{Metaphor in spoken undergraduate statistics lectures}
\label{chapter4}

\section{Chapter overview}

In spoken academic discourse, deliberate metaphor seems to be a
powerful tool.  Published research on spoken metaphor use in education
appears to be focused on its use by teachers at primary or secondary
level \parencite{munby1986,cameron2003}, with an emphasis on science
education.  \mycite{low2008} is one of the very few publications to
consider metaphor use in university lectures, although attention is
confined to humanities subjects.

As far as academic discourse is concerned, it is widely accepted that
metaphor is a \dquote{basic epistemological, discourse-organizational,
  and pedagogical device}~\parencite{beger2015}.  As such, one might
expect metaphor to be part of spoken education at an undergraduate
level.  This chapter will consider the extent to which metaphor is
part of the most widely recognized aspect of spoken language in
undergraduate statistics education: the lecture.

\section{Introduction}

The traditional spoken lecture is a pedagogical genre that has been
much maligned as a learning tool~\parencite{friesen2011}; authors such
as~\mycite{king1993} decry lectures as an antiquated \dquote{sage on the
  stage} and urge their replacement with a constructivist
\dquote{guide on the side}. 

One of the most cogent and fierce critics of the traditional lecture
is~\mycite{laurillard1993}: \dquote{Lectures are profoundly defective,
  inefficient, and outmoded}.  They are, she asserts, \dquote{a very
  unreliable way of transferring the lecturer’s knowledge to the
  students}.  Perhaps this is true, but observe the casual use of
\conmet{communication is transfer} metaphor and the \conmet{minds are
  containers}; also note the implicit use of \citeauthor{sfard1998}'s
\conmet{education is acquisition} metaphor.

However, lectures also have their champions: \mycite{burgan2006}, for
example, lauds the \dquote{public display of daring and dazzling
  intellectual expertise} that only a live lecture can provide.
Students too defend lectures, specifically citing the
\dquote{efficiency} of lecturing, but note here too the unquestioned
use of the transmission metaphor: \conmet{learning is acquisition}.

\mycite{yoon2011} report that students overwhelmingly defend the
transmission mode of lecturing, while simultaneously acknowledging
that lectures did little to contribute towards understanding.
Students, in interviews, emphasized the efficiency of this mode of
teaching and noted the practical necessity for the lecturer to get
through the allotted lecture content.

%\dquote{However, students unanimously defended the transmission mode
%  of lecturing, even while acknowledging that it did little to
%  engender understanding during the lecture.  They argued that the
%  practices involved in the transmission mode of lecturing were
%  efficient, practical and necessary for the lecturer to get through
%  the allotted content for the lecture}~\parencite{yoon2011}

Nevertheless, lectures are an important feature of undergraduate
education, with the traditional lecture comprising just over half a
student's contact hours in a typical statistics course.
\mycite{yoon2011} attribute the intransigence of lectures to a
combination of academic inertia and students' familiarity with the
format.

What role does \emph{metaphor} play in this problematic learning
environment, peculiar as it is to higher education?  In this section,
I will consider spoken lectures and their use of metaphor from a
pedagogical perspective.

\section{Initialization: call for quiet}

As a lecture is a performance, there are certain normative standards
that are necessary for the process to function as intended.  One of
these is the maintenance of silence among the audience so the lecturer
can hold the floor (student questions are dealt with later in this
chapter).

Many lecturers signal the start of the lecture with a stylized speech
act\footnote{A \emph{speech act} is an utterance considered as an
  action; the canonical example is \dquote{I pronounce you man and
    wife} and \dquote{I name this ship\ldots}.  In this case, the
  start signal is simultaneously a call for quiet, a statement that
  the lecture has started, and the actual beginning of the lecture.
  \mycite{searle1969} would classify this utterance as an
  \emph{assertive}, a \emph{directive}, a \emph{commissive}, and a
  \emph{declaration}.} ranging from a simple \dquote{good morning} to
more sophisticated rituals which may include non-verbal components
such as dimming the lights.

My own lectures have a mixture of these two things: the system clock
is visible to the students on the display screen and when the second
hand passes the precise start time I say \dquote{right, let's go}.
The utterance is clearly metaphorical; nothing is \dquote{going}.  In
this case the first person plural is inclusive (compare
chapter~\ref{chapter5}, in which \dquote{we} is used
idiosyncratically).  It is interesting to observe that the subjunctive
mood is used: the intention is clearly one of inclusion.

It is difficult to study this aspect of language use.  Lecturers 
rarely allow \dquote{outsiders} to attend their lectures, having
\dquote{entrenched norms} of autonomy and
privacy~\parencite{evans2012}; and when they do, this is likely to
change the atmosphere in the lecture hall.


\section{Lecture content}

Metaphor use in spoken undergraduate lectures has been studied
by~\mycite{low2008}, who observed that metaphorical language was used
in large quantities in social science lectures.  They found that
metaphors occurred in \emph{clusters}: conceptually coherent segments
of speech, rich in metaphor.  \citeauthor{low2008} hypothesized that
metaphor clusters marked the boundary between two distinct themes in
the lecturer's narrative, signalling a major turning point.  Metaphor
also appeared when the lecturer was placed under pressure to think
quickly.  The extent to which these findings from teaching in the
social sciences apply to statistics lectures is investigated in this
chapter.

One might expect that, given the pervasiveness of metaphor in
lectures, that figures of speech such as simile would also be common.
However, \mycite{low2010_wot_no_simile} found a \dquote{virtual absence}
of simile in a large corpora of academic English, which included
undergraduate lectures.

In the following, I will discuss a number of metaphors used in my own
videotaped lectures, using the protocol developed
by~\mycite{bergsten2007} for undergraduate pure mathematics courses.
\citeauthor{bergsten2007} split lectures into fragments of a few words
and analysed the fragments individually, focusing on the relation
between the spoken and written content and observing other features of
the lecture such as student questions and the lecturer's gestures.

The source material used here is taken verbatim from a lecture in
which I introduced the Poisson approximation to the binomial
distribution.  This particular lecture was chosen because the limiting
process discussed is an examplar of the basic metaphor of
infinity~\parencite{lakoff2000}.  The lectures were recorded two years
ago.  The sentence fragments are those containing metaphorical
language, as determined by the~\citeauthor{pragglejaz2007} metaphor
identification protocol; the intervening utterances contained no
metaphor.

\begin{description}
\item[\metaphor{We} have been talking quite a lot about the Binomial
  distribution\ldots]{An example of a metaphorical \metaphor{we}.  The
  audience is almost totally silent; the \metaphor{we} is actually
  \metaphor{I}.  \mycite{pimm1984} discusses the use of the first
  person plural in this context\footnote{\citeauthor{pimm1984} also
    writes about this issue in educational contexts using written
    English; I draw on his work in Chapters~\ref{chapter5}
    and~\ref{chapter6}.}, pointing out that the \dquote{educational
    we} often effectively excludes the speaker.  \citeauthor{pimm1984}
  expresses bafflement as to exactly which community \metaphor{we}
  indicates, and conjectures that it induces (either deliberately or
  inadvertently) \dquote{passive acquiescence} in the
  student\footnote{This interpretation has been cited in a small
    number (A \emph{Web of Science} cited reference search gives~10
    citations at the time of writing) of published sources.  They
    uniformly refer to \mycite{pimm1984} only in passing; and none of
    them offers any conflicting viewpoints.}.} 
\item[\squareb{the Poisson} is one of a \metaphor{family} of
  distributions]{A lexicalized metaphor, one that is standard
  terminology in statistics.  In undergraduate statistics,
  \dquote{family} is usually reserved to describe a set of
  distributions indexed by one or more (possibly real) parameters.
  The literal meaning of \dquote{family} is sociological: a group
  consisting of one or two parents and one or more dependent children
  living together.  However, one striking misalignment of this
  metaphor is that the vehicle is a set (of humans)that is not only
  \emph{discrete} and \emph{finite}, but also has a very small
  membership, typically in the range 2-5. Note also the culturally
  specific nature of this metaphor.  Contrast the topic, which is not
  only continuous but generally infinite.  There are other
  differences: familiarity and ready identification are salient
  features of the vehicle, yet in the topic, complicated and
  unreliable mathematical inference is needed.  Such differences are
  the essence of pedagogical metaphor, as the topic is rendered
  comprehensible due to familiarity with the vehicle.}
\item[Bernoulli trials with a probability of
  \metaphor{success}\ldots]{Standard statistical terminology is to
  refer to the support of any random variable with two outcomes as
  \dquote{success} and \dquote{failure}.  However, there is no value
  judgment inherent in these words and one finds (in studies of
  family sex balancing, for example), that a birth being male is a
  \dquote{success} and female a \dquote{failure}.  Mathematically, the
  two terms are equivalent as they are symmetric with respect
  to~$p\longleftrightarrow 1-p$.

  The terminology is arguably metaphorical: the topic (support of the
  random variable) is described using the vehicle of an abstract
  experiment which may succeed or fail.  This abstractness is, I would
  argue, a virtue on the grounds that one \emph{wants} to extirpate
  any traces of value judgments from the narrative.}

\item[simply because this $\mathbf{1-p}$ here \metaphor{turns into}
  a~$\mathbf{q}$ there]{The context was that the binomial probability
  mass function~${n\choose r}p^r(1-p)^{n-r}$ was rewritten
  as~${n\choose r}p^rq^{n-r}$, with~$q$ being substituted for~$1-p$.
  This is another example of agency metaphor, but one of a peculiar
  kind: the equation is somehow imbued with the ability to transform
  itself from one form to another.}

\item[a much more \metaphor{symmetric} way of writing it]{Here the
  binomial probability mass function was written in the
  form~${a+b\choose a\, b}p^aq^b$, with~$a$ being the number of
  successes and~$b$ the number of failures (the intent was to
  introduce the Dirichlet distribution).  The words \dquote{symmetry}
  and \dquote{symmetric} are problematic for mathematicians; the words
  originally referred to spatial harmony and, for most
  people---including the lecture audience, I would suggest---symmetry
  is an inherently geometric concept.  Here the contextual meaning is
  that of algebraic symmetry between parts of an equation, a concept
  likely to be new to much of the audience.}
\item[because \metaphor{we have} asymmetry between~$\mathbf{a}$
  and~$\mathbf{b}$]{The context was referring to~${a+b\choose
    a\,b}p^a(1-p)^b$, clearly placing~$a$ and~$b$ on a different
  footing.  This is not really an example of \citeauthor{pimm1984}'s
  \dquote{we}: in this case the asymmetry was undesirable.}
\item[it's a much more pleasing way of \metaphor{handling} this]{There
  are two metaphors here.  Firstly, the use of \dquote{pleasing}: note
  the passive voice.  Pleasing to whom?  The intended sense is that
  the form of the equation is intrinsically appealing, independently
  of any particular viewer.  This is not an uncommon
  viewpoint~\citep{rota2011}.  The intended sense is that the
  community of practice into which the students are being acculturated
  is one which collectively finds that particular expression pleasing.

  The other metaphor is the use of the word \dquote{handling}.  The
  intended sense is that the equation under consideration is a
  physical object, and expressing the equation in different
  mathematical ways corresponds to physical manipulation of the
  object.  This might suggest a new conceptual metaphor:
  \conmet{equations are objects}.}
\item[and I'll \metaphor{return to} this formulation later] This is an
  example of \conmet{an argument is a journey}, here underscoring the
  difficulty of the material.
\item[One disadvantage of the binomial distribution\ldots] Arguably a
  peculiar metaphor, perhaps \conmet{equations are tools}.  The
  intended sense was that the distribution included analytically
  intractable terms such as the factorial function, which made the
  equation hard to deal with.
\item[\ldots is that it has this factorial function \metaphor{in it}]
  \conmet{An equation is a container}
\item[The first thing that should \metaphor{come into your head} is]
  {\bf to verify \squareb{that formula}\ldots} This is an example of
  \conmet{ideas are objects and the mind a container};
  \citeauthor{bereiter2005} would dismiss this as \dquote{folk theory
    of mind}.  However, the form of words used is interesting because
  there is no indication of the \metaphor{conduit}
  metaphor~\parencite{reddy1993}: there is no suggestion that the
  concept of verification originated from the lecturer.  The clear
  import is that the (habit of) verification should spontaneously
  arise, unbidden, in the student's mind: this would be the
  \metaphor{agency} metaphor.
\item[Am I doing what I think I'm doing?\ldots can \metaphor{I} check
  it, can \metaphor{I} verify it\ldots] (the context was an
  exhortation to the audience to check their work continually, to
  identify and rectify algebraic and conceptual errors).  This is an
  interesting use of the metaphorical \dquote{I}.  This is arguably an
  inversion of \citeauthor{pimm1984}'s educational \dquote{we}: here,
  \dquote{I} is clearly intended to indicate what the class should be
  doing.

  \mycite{fauvel1988} would observe that such rhetorical devices are
  Cartesian rather than Euclidean: the audience is being personified
  directly and quotes attributed directly to them.

\item[however this is quite a difficult and unwieldy process]
  \conmet{equations are tools}, in this case undesirable qualities of
  algebraic manipulation.
\item[I will \metaphor{give you} this formula in a different form]
  This is one of a large number of \dquote{give} metaphors used in
  this series of statistics lectures (e.g., \dquote{I can
    \metaphor{give} you an exact answer to that}).  Such phrases are
  direct examples of the \metaphor{conduit} metaphor
  of~\mycite{reddy1993}.  However, note the simultaneous use of the
  \metaphor{acquisition} metaphor of \mycite{sfard1998}.  In this
  case, the contextual meaning is a promise to re-write the formula;
  but the basic meaning is clearly both acquisitive and transferative.
\item[The factorial function isn't easy \metaphor{to deal
    with}]{(also, later, \dquote{the standard deviation is harder to
    deal with than the mean}).  Arguably a personification metaphor:
  the factorial function is given agency.  \mycite{low2008} assert that
  personification is by far the most common metaphor in humanities
  lectures and it is certainly common in these statistics lectures.}
\item[I'll \metaphor{cover} the first two or three members of the
  series]{In this context, \metaphor{covering} is a very commonly used
  metaphor.  Many authors (\mycite{biggs2011} and~\mycite{vella2007},
  for example) criticize the very notion of \dquote{covering} a topic,
  on the grounds that it obscures any learning objectives;
  \mycite{paechter2004} oberves that it is a spatial metaphor.

  Note also that the context also carries the implication that this
  material will be assessed at some point, as the concept being
  covered appears on the learning objectives, which are explicitly
  assessed.}
\item[the probability of success is p=0.5, so it is \metaphor{a fair
    coin}\ldots] A coin toss is a prototypical example of a Bernoulli
  trial: it is indisputably random, the probability of success (heads)
  is known precisely, and successive tosses are demonstrably
  independent.  But to say that a Bernoulli trial \emph{is} a fair
  coin is clearly metaphorical.  The coin metaphor is standard
  terminology in statistics.
\item[I can hear my mathematical colleagues \metaphor{howling in
    outrage}]{This was in the context of a somewhat low-status
  technique involving numerical approximation; also, later,
  \dquote{these disadvantages wouldn't \metaphor{cut much mustard}
    with a mathematician}.

  The thrust of these comments is that there are differing schools of
  thought in mathematics and the approach taken in the lecture
  sacrifices exactness (which is highly prized in some disciplines)
  for computational convenience (which is highly prized in this
  particular course).
  
  These metaphors are used in rapid succession, and qualify as part of
  a metaphor \emph{cluster} in the sense of~\mycite{low2008}.  However,
  this cluster did not mark a \dquote{major turning point} in the
  lecture, unlike the clusters identified by~\mycite{low2008}.}
  \item[last time we had the binomial
    distribution~$\mathbf{\operatorname{\mathbf{Bin}}\left({n,p}\right)}$]
    {\bf\ldots I'm going to \metaphor{make~$\mathbf{n}$ get larger} in
      a particular way} This qualifies as a metaphor, in this case the
    basic meaning of \dquote{make} is \dquote{force} but the topic is
    an equation, in this case a probability mass function.

  \item[Last time we discussed~$\mathbf{n}$ \metaphor{getting larger},
    with~$\mathbf{p}$ fixed] In terms of the Basic Metaphor of
    Infinity (\bmi) of~\mycite{lakoff2000} (see
    chapter~\ref{chapter5}), this would be \dquote{potential
      infinity}.  The mathematical statement is here the well-known
    Gaussian approximation to the Binomial, another limiting
    distribution this time arising from the central limit theorem.
    
  \item[I asserted that the limit] {\bf (in scare quotes, I'm not
    defining formal limits) as~$\mathbf{n}$ \metaphor{approaches
      infinity}, of the distribution of~$\mathbf{r}$, the number of
    successes, is normal or Gaussian with mean~$\mathbf{np}$ and
    standard deviation~$\mathbf{\sqrt{npq}}$}.  It is difficult to
    interpret this utterance in terms of the \bmi, yet metaphor is
    clearly used.  The phrase \dquote{approaches infinity} is,
    although standard terminology, clearly metaphoric: nothing
    actually \emph{approaches} anything; and in any event,
    \dquote{approaches} implies \dquote{getting closer (to
      something)}, which is emphatically not occurring.
  \item[Now $\mathbf{n}$ is getting larger and~$\mathbf{p}$ is] {\bf
    fixed\ldots and \metaphor{it looks like this} (draws a Gaussian on
    the whiteboard)} I would suggest that this is a metonym: a random
    variable is identified with its probability density function.
    Also note that the random variable is imbued with a visual
    appearance.
  \item[I'm going to think about the] {\bf two parameters,
    $\mathbf{n}$ and $\mathbf{p}$, and I'm going to consider a
    sequence in which $\mathbf{n}$ gets bigger, and simultaneously
    $\mathbf{p}$ gets smaller in such a way that $\mathbf{np}$ stays
    fixed.} The very essence of the (non-metaphorical) Cauchy
    sequence~\citep{hardy1952}: this shows that metaphor is \emph{not}
    necessary for everyday mathematical teaching.
\end{description}

\subsection{Summary}

The sentence fragments quoted above illustrate the frequency and
ubiquity of metaphor in statisics lectures.  Conceptual metaphors were
frequent, specifically \conmet{equations are tools} which was used
several times.

Metaphor, at least in the source material above, is a key pedagogical
tool in the sense that many of the utterances could not easily be
rephrased nonmetaphorically.  None of the metaphors (with the possible
exception of the highly idiomatic \metaphor{cut much mustard}) are
natural part of language and would not draw attention to themselves.

Such observations are consistent with those of~\mycite{geary2011}:
metaphors are indeed \dquote{hiding in plain sight}.  Their very
unremarkability, even invisibility, combined with their frequency and
power, suggests that we take metaphor very seriously in education.

\section{Students' use of language in lectures}

The lecturer is not the only source of language in lectures: students
also, on occasion, ask questions.  \mycite{marbach2000} consider
students' asking of classroom questions as they progress from
elementary level to college, and conclude that students learn not to
ask questions in class.  At undergraduate level, the lecture
environment has implicit social norms which generally enforce a
passive role during lectures~\citep{yoon2011}.

Questions are generally infrequent, with~\mycite{pearson1991}
reporting an average of only three questions per hour, the majority of
which were on-task but restricted to procedural clarification, such as
due dates on assignments or venues for lectures.  The only questions
in the chosen lectures on the Poisson distribution were clarificatory.
There does not seem to be enough data to make any claims about
students' use of metaphor in this context, either from the literature
or my own observations in my lectures.

\section{Conclusions}


Metaphor is certainly a component of spoken undergraduate statistics
lectures, but its frequency appears to be low and the metaphors used
appear to be somewhat light.

The majority of metaphors in the corpus analysed appear to relate to
\dquote{domestic} aspects of the lecture such as promises to discuss
certain content, or standard pedagogical constructs such as
\conmet{communication is transfer}.

A certain amount of metaphor is unavoidable in amy comprehensible
mathematical or statistical discourse.  However, much of mathematics
is arguably metaphorical, as argued by~\mycite{lakoff2000}.
Specifically, the \metaphor{basic metaphor of infinity} (\bmi)
occurred several times in the corpus under study.  A more detailed
discussion of the \bmi is given in Chapter~\ref{chapter5}.




%\conmet{learning is storage} produces [a situation that requires
%  students to] shut up, avoid wiggling, and above all avoid
%interrupting.  p345
%
%quote attributed to
%1, Marshall Gregory, 'If Education Is a Feast, Why Do We Restrict The
%Menu? A critique of pedagogical metaphors, \dquote{College Teaching},
%Vol.35,Part 3, 1987,p.103.
%
%{\bf The storage metaphor is stultifyingly utilitarian and deceives
%  young people into thinking that we are giving them information vital
%  for their survival in the adult world.  p346 }

%\setlength{\epigraphwidth}{.7\textwidth}  % default is .4
%\epigraph{[The dichotomy between expository and argumentative essay
%    styles] is somewhat muddied by the fact that all essay examination
%  writing contains an argumentative element, namely the writer's
%  attempt to persuade the reader/instructor to proffer an acceptable
%  grade in a course.  This intention must be hidden, however\ldots
%  because of the social convention that states that an expository exam
%  essay is to be written as if the reader did not already understand
%  the ideas being presented---as if a prompt were really a
%  question---and that an argumentative essay is to be written as if
%  the reader did not already agree with the thesis---as if a prompt
%  were really part of a Socratic dialogue.  Thus, essay examination
%  writing is doubly false, in that writers must hide their true
%  intention (to pass the course) behind a wall of prose designed to do
%  what has already been done}{\mycite{horowitz1986}}
%
