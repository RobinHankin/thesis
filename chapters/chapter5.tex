\begin{singlespace}
\begin{savequote}[105mm]
  How much can we infer about the basic cognitive mechanisms used in
  mathematics from what we find in texts and curricula? A study of
  navigation based on the standard manuals would tell us very little
  indeed about how the task was actually accomplished on the bridge of
  a large ship.  \qauthor{\mycite{madden2001}}
\end{savequote}
\end{singlespace}

\chapter{Metaphor in statistics textbooks}
\label{chapter5}

\section{Overview}

This chapter gives a discussion of metaphor and metaphorical language
as used in statistics textbooks.  The majority of the relevant
literature covers mathematics in general; few articles consider
metaphor in statistics education.  In this chapter I use literature
that analyses metaphor in mathematics textbooks and consider the
extent to which the findings are applicable to statistics education.

The chapter is split into two main parts.  The first part will
consider metaphor in mathematics generally, specifically as
interpreted in the controversial \emph{Where mathematics comes
  from}~\citep{lakoff2000}, henceforth~\wmcf.  I will consider this
book from the perspective of statistics education.

The second part of the chapter concentrates on one often-overlooked
aspect of language frequently used in mathematics textbooks: the
mathematician's \emph{we}.  This usage is considered to be
metaphorical because the referents are not a well-defined group.  I
will consider the educational implications of this language use.

\section{Introduction}

A \emph{textbook} is a standard work for the study of a particular
subject, here statistics; attention will be confined to those used for
undergraduate study.  Mathematics textbooks are a valuable and
often-consulted resource for university
students~\parencite{weinberg2012}.  One might expect metaphorical
language to be widely and effectively deployed.

Three textbooks were chosen for detailed study:

\begin{itemize}
\item \mycite{casella2001}\
\item \mycite{crawley2015}
\item \mycite{feller1968}
\end{itemize}

These books span a range of
sophistication---\citeauthor{casella2001}\footnote{It is standard
  practice to refer to textbooks by the author(s) name} is classified
as high-end undergraduate or mid-range postgraduate study material;
\citeauthor{feller1968} is a classic work, emphasizing rigour, while
\citeauthor{crawley2015} is practitioner-oriented, heavy on
computational examples and light on mathematical detail.

However, none of these books appeared to use metaphor at all or at
most very very sparingly.  The \mycite{pragglejaz2007} protocol, applied
to the texts, revealed that metaphor (in the sense of
\mycite{lakoff1980}) was rare to nonexistent.  This is perhaps not
surprising in such a mathematical context where accuracy is more
highly valued than clarity or even educational value.

There was, however, one non-literal usage of language that occurred
frequently throughout all three books: the mathematician's \emph{we},
which is discussed in section~\ref{we_start}.

\section{Random variables and metaphor}

The notion of \emph{random variable} is a central concept in
statistics.  The formal definition of a random variable is as follows.

\begin{quote}
Suppose we have a probability
space~$\left({\Omega,\mathcal{F},P}\right)$.  Then if~$E$ is some set
and~$X\colon\Omega\longrightarrow E$ is measurable function
from~$\Omega$ to~$X$, we say that~$X$ is a {\em random variable}.
\end{quote}

\noindent
Note the abstract and unhelpful nature of such a rigorous definition;
mathematically, the difficulty lies in ensuring consistent behaviour
when~$E$ is uncountably infinite (one prominent example would
be~$\mathbb{R}$, the real numbers; one would hope that such
definitions do not let one down in such a practically important case).

The influential {\em Khan Academy}~\parencite{khan2016} is one of many
introductions to inferential statistics that discusses random
variables from a more practical perspective.  While declining to offer
a formal definition, \citeauthor{khan2016} does give several examples,
the canonical one being

\[
X = \begin{cases}
  0 &\text{if coin lands tails}\\
  1 &\text{otherwise}
\end{cases}
\]

\noindent\citeauthor{khan2016} goes on to state that, 
together with the specification that~$p(X=0) = p(X=1)=1/2$ fully
characterizes~$X$.

Undergraduate statistics textbooks typically offer a level of rigour
between these two extremes.  To what extent do linguistic or cognitive
metaphors enter in to such discussions?  \mycite{lakoff2000} would
suggest that metaphor plays a large part in all of mathematics, and
indeed claim that \emph{all} mathematical reasoning is inherently
metaphorical.

Of all the metaphorical mathematics presented in \wmcf, by far the
most relevant is the \metaphor{basic metaphor of infinity}, discussed
in the next section.

\section{The basic metaphor of infinity}

% macro '\wmcf' is defined in main.tex

Metaphorical language and reasoning is common in mathematics and
mathematics education~\parencite{pimm_metaphor_1981}.  However, the
study of metaphor in mathematics was kick-started by publication of
the controversial {\em Where mathematics comes
  from}~\parencite{lakoff2000} which set out the authors' contention
that {\em all} mathematical reasoning is metaphorical.  The authors
also make a case for mathematics {\em per se} being a human construct.

Statistics, like many branches of mathematics, often uses the concept
of \emph{infinity}.  Here I draw on the ideas of \mycite{lakoff2000},
in a controversial work often referred to as \dquote{\wmcf} (being the
initials of the book title, \emph{Where Mathematics Comes From}).
\wmcf suggests that all mathematical thought is metaphorical and the
authors make a case for even such fundamental branches of mathematics
as axiomatic set theory being metaphorical: for example, the authors
point out that the conceptual metaphor \conmet{sets are containers and
  elements objects in them} is purely metaphorical, yet almost
universally used when thinking about set membership.

Notions such as the number line are also held to be metaphorical:
natural numbers are not points on a line; counting (enumeration) is
not temporal progression along a marked rod; sets are not containers
with elements objects inside them.

\wmcf makes a case, echoing that of~\mycite{lakoff1980}, for many if not
all such metaphors to be rooted in sensory-motor experience.  Here the
most germane is the \metaphor{basic metaphor of infinity} (\bmi) in
which processes that go on indefinitely are conceptualized as having
an end and an ultimate result.  Motivating examples are discussed,
including the one-point compactification of the plane, limits of
sequences, and mathematical induction.

Below I will discuss the relevance of the \bmi to my own teaching,
specifically the limiting behaviour of the binomial distribution to
the Poisson.

It should be pointed out that the book has come under severe criticism
and indeed the ideas have been met with little interest among
mathematicians.  The authors do not put forward any empirical support
whatsoever~\parencite{madden2001} for their assertions about the ways
metaphorical reasoning is used when mathematics is carried out.
\mycite{schiralli2003}, for example, considers the book to make
\dquote{fundamental oversimplifications} and observe that the authors
use the word \emph{metaphor} to serve so many purposes that
\dquote{the notion of metaphor itself begins to lose its meaning}.

The book received at best mixed reviews from both mathematicians and
cognitive scientists.  Neither of the authors is a mathematician (and
certainly no non-elementary mathematics is presented in the book).
The authors present \dquote{misconceptions of mathematics [that] are
  prevalent among non-mathematicians}~\parencite{henderson2002}.
Indeed, many reviewers point to the \dquote{rather frequent}
mathematical errors~\parencite{gold2001}; at many points in the book,
metaphorical reasoning is invoked to explain mathematical cognitive
phenomena, yet a slightly more sophisticated analysis would show an
appropriate mathematical framework.

Nevertheless, as the authors point out, cognitive mathematics is a
sorely neglected field of study; and the book provides a coherent
account of cognition's role in mathematics.

Other aspects of the book are unsatisfactory.  \mycite{goldin2001}, for
example, considers the book to be \dquote{fundamentally flawed} on the
grounds that it was poorly sourced in both cognitive science and
philosophy of mathematical thought.  When mathematicians review the
book, they observe that \wmcf includes \dquote{numerous errors of
  mathematical fact}~\parencite{henderson2002} and also conflates at
least three distinct mathematical activities: learning, using and
research.

The only evidence that the authors adduce for their assertion that
metaphor underlies all mathematical thinking is textual. This, if
nothing else, suggests that it is at least plausible that textual
analysis of the type given in chapter~2 of the current thesis is a
respectable source of information in its own right.

My own reading of \wmcf suggests that the authors appear to be
ignorant of mathematical techniques that render much of their
metaphorical interpretation unnecessary.  For example, in the context
of elementary group theory, the authors give an extended discussion of
what I would call~$C_3$, the cyclic group of three elements.  They
insist that the different examples of this group (plane rotations
by~$2n\pi/3$, arithmetic modulo~$3$, etc) are \dquote{metaphorically
  linked}, and give extensive tables; yet they appear to be ignorant
of the notion of isomorphism, a formal and non-metaphorical concept
that would render their analysis superfluous.

Considering the \bmi, the authors again appear not to have understood
(and certainly have not mentioned) the concept of \emph{Cauchy
  sequence}, which again would render much of their discussion
superfluous.  A sequence~$x_1,x_2,\ldots$ is Cauchy if, for
any~$\epsilon>0$, one can find an integer~$n_0$ such
that~$n,m\geqslant n_0$ implies~$\left|{x_n-x_m}\right|<\epsilon$.  It
is easy to show that a Cauchy sequence approaches a limit
as~$n\longrightarrow\infty$.

Cauchy's startling and elegant definition neatly sidesteps any
confusion between \dquote{actual infinity} and \dquote{realized
  infinity} as the limit itself is not mentioned; observe that no
metaphor is needed.  For this reason, Cauchy sequences are fundamental
to the understanding of many diverse mathematical concepts such as
compactness in Hilbert spaces and completeness of $p$-adic numbers.

From an undergraduate statistics education perspective, the \bmi is
used when considering convergence of random variables.  The example I
will discuss is drawn from the spoken lectures discussed in
Chapter~\ref{chapter5}: the elementary observation that the Poisson
distribution is a limiting case of the binomial.  The formal statement
I am expressing is as follows: \\ \\

{\bf Theorem.}  if~$X_n\sim\operatorname{Bin}\left({n,r/n}\right)$
assuming~$0\leqslant r\leqslant n$, then
  \begin{enumerate}
  \item ${\displaystyle
    \lim_{n\longrightarrow\infty}X_n=X}$ exists, and
  \item $X\sim\operatorname{Poisson}\left({r}\right)$; that
    is~$\operatorname{Pr}\left({X=n}\right) = \frac{e^{-r}{r^n}}{n!}$.
  \end{enumerate}

  \noindent
This fact is neither formally stated, nor any proof given; but the
underlying idea is both simple and important for statistics at this
level.  Observe that the concept of Cauchy sequence is applicable to
probability mass functions just as well for real numbers\footnote{The
  \dquote{distance} between two probability mass functions is simply
  the supremum of the differences between their cumulative
  distribution functions.}.

In this context, \wmcf (Where Mathematics Comes From,
\cite{lakoff2000}) asserts directly that the \bmi is unavoidable in
mathematical language, yet the concept of Cauchy sequence neatly
avoids any need for arguably metaphoric concepts of \dquote{limit} and
\dquote{the infinite}.  The excerpts shown above demonstrate that
careful use of Cauchy sequences can illustrate the concepts of
infinite limits---certainly in the case of Borel probability
measures---without any potentially confusing metaphorical language;
and that such methods are available in a written or spoken context.

\section{Rhetorical \label{we_start} metaphor in textbooks}

Mathematical textbooks, including statistics textbooks, frequently use
rhetorical devices as part of their communication
strategy~\citep{kane1970}.  One such rhetorical device is the use of
\emph{we} which is metaphorical in the sense that the writer is not
using literal language: the reader is identified with a poorly-defined
\dquote{community of practice}, of which the writer is one (perhaps
pre-eminent) example.

\citeauthor{pimm1984} asserts that textbooks' use of \emph{we}
attempts to \dquote{enrol the tacit acquiescence of the reader}, and
serves as an imposition which fails to take into account the wishes or
interests of participating individuals.

This peculiar use of \emph{we} among mathematicians is not limited to
textbooks; it is a ubiquitous construction in research
articles~\parencite{kuo1999}.  Consider, for example, the first
article in the most recent edition at time of writing in \emph{The
  Journal of Topology}\footnote{The discipline of topology is a
  theoretical branch of pure mathematics, notable for its extreme
  abstraction}, \parencite{lange2016}. This is a typical article in
the field, the abstract of which starts \dquote{We characterize finite
  groups~$G$ generated by orthogonal\ldots}; the \emph{we} must be
inclusive because the article is single-author.  \mycite{kuo1999}
considers that the almost complete absence of first-person singular
pronouns (I/me) to be evidence of effort to reduce personal
attributions, and this tendency is presumably operating in textbooks
too.

\subsection{Inclusivity in mathematics}

The mathematicians' \emph{we} is thus an attempt to draw the reader in
to the community of practice.  In this context, \mycite{fauvel1988}
considers the issue of inclusivity in mathematics, drawing a
distinction between the \emph{Euclidean} and \emph{Cartesian} styles
of rhetoric.  \citeauthor{fauvel1988} characterizes the Euclidean
style as follows:

\begin{singlespace}
\begin{quote}
 There is no sign he notices the existence of readers at all.  Rather,
 he seems engaged in laying down inexorable eternal truths.  The
 reader is never addressed.
\end{quote}
\end{singlespace}

\noindent and compares with the Cartesian approach:

\begin{singlespace}
\begin{quote}
The mathematics described is clearly created, not unveiled, in
rhetoric which veers from grabbing the reader by the lapels to
treating you with utter disdain\ldots
\end{quote}
\end{singlespace}

\noindent The three textbooks use the mathematicians' \emph{we}, and
are thus more Cartesian than Euclidean in outlook (at least, if the
inclusive sense is understood).

\subsection{Grammatical inclusivity}

It is interesting to note that English does not distinguish between
inclusive and exclusive \emph{we}\footnote{Inclusive \emph{we}
  specifically includes the addressee while the exclusive form does
  not}, so there is no grammatical way to detect whether the reader is
included in the writer's utterances.  Such considerations can be
important in political speech~\parencite{chen2006}.  Languages such as
Te Reo M\={a}ori do maintain a distinction between inclusive and
exclusive forms (the words are t\={a}tau and m\={a}tau respectively),
so perhaps M\={a}ori textbooks would afford some insight here.


\section{Conclusions}

Metaphor is an unavoidable component of exposition used in statistics
textbooks.  Three statistics textbooks were chosen for detailed study
and their use of metaphor seemed to be broadly similar.

Metaphors such as the mathemticians' \dquote{we} and the basic
metaphor of infinity were frequent.  These metaphors were not
discipline-specific to statistics.  Discipline-specific metaphors
included the basic metaphor of infinity used to illustrate the central
limit theorem.
