\begin{singlespace}
\begin{savequote}[105mm]
  \begin{minipage}[t]{.45\textwidth}\raggedright
    I really like Dr X's teaching style.  He's really clear and
willing to help you if you need it.  He's also pretty funny.  Totally
recommend!\end{minipage}\hfill\begin{minipage}[t]{.45\textwidth}\raggedright
    I hated him.  He is the worst teacher on the planet.
And his jokes are not as funny as he thinks they are
\end{minipage}\\ \rule{0mm}{8mm}---Typical
positive and negative comments on the \emph{Rate my professor}
website.
\end{savequote}
\end{singlespace}

\chapter{Metaphor in course evaluation}
\label{chapter7}

\section{Overview}

Many institutions mandate polling of students' opinions on the quality
of instruction.  Such polling is almost universally administered by
questionnaires distributed to students at or close to the end of a
course~\citep{wachtel1998}.  Reasons given for the collection of
student evaluation include instructional improvement, and as a
personnel or management tool (student evaluation is an input to
promotion and tenure decisions).

One aspect of student evaluation that is often overlooked is student
\emph{comments}: written responses to open-ended
questions~\citep{stewart2015}.  In this chapter, I discuss students'
spontaneous\footnote{A few authors~\parencite{kemp1999,starrglass2005}
  report on the results of explicitly asking students to produce
  metaphors for the course (the cue is along the lines of \dquote{For
    us, this programme was like\ldots because\ldots}.} use of metaphor
in the free-form comments section of course evaluation.

\section{Introduction}
The overwhelming majority of research into student evaluation focuses
on \emph{ratings} or \emph{scores}, that is, quantitative assessments
attributed to various aspects of a course~\citep{stewart2015}.  A
typical rating might be a point on a five- or seven- point Likert
scale, given in response to a cue such as \dquote{The course was
  interesting and stimulating}.

There are two situations in which metaphorical language may be used in
course evaluation: the cues used in the questionnaire design; and the
comments made by students in response to open-ended questions.

The cues used in questionnaires do not appear to have been studied in
detail~\citep{aleamoni1980} but if AUT evaluation are typical, the
salient component of the process is requesting students to give a
Likert score to various prompts.  The Likert cues are generally
literal---and non-metaphorical---noun phrases: \dquote{appropriate
  workload}; \dquote{overall experience}; \dquote{good teaching}.

Cues for comment were, in contrast, phrased as questions: \dquote{What
  were the best aspects of this paper?}; \dquote{What aspects of this
  paper are most in need of improvement?} and these were again devoid
of anything but the most lexicalised metaphor.

\section{Students' comments in course evaluation}

The only aspect of student evaluation in which metaphor may be used by
students is the in the \emph{comments} section: written responses to
open-ended questions~\citep{stewart2015}.  Despite the extensive
research literature on student \emph{ratings}, comparatively little is
known about the quality of data obtained from students' written
comments, their content, and the relationship between them and other
variables.

Student comments are time-consuming to interpret, as standard
descriptive and inferential techniques cannot be directly applied to
free-form text.

One over-riding difficulty in the study of students' written comments
is ethical: evaluation usually operates under strict confidentiality
guarantees.  \mycite{stewart2015}, for example, considers only an
aggregated corpus of comments, being prevented from analysing
individual responses by ethical concerns.

The few studies which have been reported in the literature tend to
focus on whether the comments are positive or negative.  
\mycite{alhija2009}, considering the frequency with which students'
feedback includes comments, reports that the response rate lies
between~10\% and~70\% and indicates that such a wide range reveals the
\dquote{relatively small body of research} in this area (page 38).

\mycite{braskamp1981} is one of the few reports on comments \emph{per
  se}; these authors report close agreement between written comments
and objective responses.  Nevertheless, comments can be informative in
ways not possible for ratings.  \mycite{tucker2014}, considering
students' evaluation comments, provides a comprehensive review of
students' use of language in this aspect of a course, and observes
that student comments can provide \dquote{valuable insights} into the
student experience.  To what extent does metaphor play a part in this
aspect of education?

\mycite{alhija2009} consider open-form requests for comment and report
a range of student responses.  The students appear to use metaphor
very sparingly, with the majority being lexicalized (\dquote{time was
  wasted}; \conmet{time is a resource}) or standard educational
metaphors (\dquote{we never got to the end}, \conmet{a course is a
  journey}).  Such metaphors are strikingly similar to those used in
documents such as course descriptors (discussed in
section~\ref{course_descriptor_discussion}).

\section{Conclusions}

Course evaluation typically includes a free-form comments section in
which students are invited to give feedback on a course.  This section
is the only section in which metaphor occurs.

It is difficult to study students' use of metaphor in this context and
only a limited amount of research has been carried out to date.
Students appear to use metaphor sparingly, and those that are used are
similar to the metaphors used in institutional documents such as
course descriptors.
