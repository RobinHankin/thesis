\begin{savequote}[105mm]\begin{singlespace}
Sometimes, there is little awareness of the metaphors that guide our
behaviours and shape our institutional structures.  We may not even be
aware of the negative consequences of the metaphors we live
by\end{singlespace} \qauthor{\mycite[page 37]{marland2005}}
\end{savequote}

\chapter{Introduction and literature review} % Main chapter title
\label{chapter1} % For referencing the chapter elsewhere, use \ref{Chapter1} 

Metaphor is usually defined as an invitation for the reader or listener to
consider one thing in terms of another \parencite{steen1994}, although
its definition is far from clear; \mycite{knowles2006} and others
state that metaphor is generally easier to recognize than to define.  Modern
treatments go to some lengths to refute the notion that metaphor is
confined to poetical use, pointing to the ubiquity of metaphor at all
levels of formality \parencite{deignan2005} in spoken
\parencite{cameron2003} and written \parencite{charteris-black2004}
English.  Textbooks usually distinguish metaphor from simile, in which
X is held to be \emph{like} Y; compare metaphor, in which X is said to
\emph{be} Y, although this characterization does not stand close
scrutiny.

Here, I will investigate the use of metaphorical language in
undergraduate statistics education.  In this thesis, \emph{metaphor}
will be interpreted as meaning any non-literal language and will thus
include idiosyncratic use such as word problems.

Metaphor is a ubiquitous phenomenon in language and this thesis
examines metaphor as used in education, specifically undergraduate
statistics education.  Although research into tertiary mathematics
education is reasonably well represented, only a small proportion is
devoted to statistics education; and it is not clear whether statistics
should be viewed as a subset of mathematics.

An example of metaphor in an educational context might be \dquote{this
  student is the \metaphor{cream} of the cohort}\footnote{Slanted type
  is used to denote metaphor, as in \dquote{all the world's a
    \metaphor{stage}}.}.  Students are not dairy products and in this
case we are invited to compare the student's elite academic
performance (in relation to his peers) with the most desirable
component of milk, viz. the cream. Observe that the comparison to
cream stimulates only a small part of the semantic connotations of
actual cream: specifically the desirable properties of tastiness and
expensiveness.  Cream itself has many undesirable features: it is
unhealthy, it is sickly, and it is fattening but these properties are
never used metaphorically.

A more pertinent example might be to describe a student as \dquote{top
  of the class}.  \mycite{paechter2004} considers this metaphor to drive
a particular view of assessment, specifically one based on comparisons
between students rather than on whether each student has achieved the
learning objectives of the unit of study at hand.

Skillful use of metaphor does not necessarily include any form of the
verb \emph{to be}, for example:

\begin{quote}
      An iron curtain has descended across the continent\\
---W. Churchill (attributed; circa 1946)
\end{quote}

In this famous quote, the verb is \emph{to descend}: the reader
(originally the listener) must infer that the dividing boundary
developing between East and Western Europe is to be considered a
physical object. Note too Churchill's mentioning iron, its utilitarian
connotations underscoring the perceived economic poverty of communism
when compared with the prosperous West.  Curtains literally descending
would be familiar to Churchill's theatre-going audience as marking the
end of a performance, and it is reasonable to believe that Churchill
was alluding to communism's extinguishing of democratic rights.

In this thesis, metaphor is discussed using standard terminology: the
\emph{topic} (sometimes \emph{tenor}) is the concept being described,
and the \emph{vehicle} (sometimes \emph{figure}) is the concept used
to describe the topic.  In Churchill's quote above, the topic would be
the political divide between East and West Europe, and the vehicle
would be a stage curtain made of iron.

It is clear that such metaphors can be informative about a speaker's
thoughts; skillful orators can utilize metaphors effectively to
encapsulate the mood of a nation, or indeed to provoke debate about a
nation's educational system~\citep{robinson2011}.  Theoretical
linguists and sociologists have published a large amount of material
dissecting and analyzing this form of language from many perspectives:
workers have studied the effective use of metaphor (by politicians;
see \mycite{perrez2015}, for example), and the effect of rhetorical
metaphor on listeners~\citep{keating2015}.

In this thesis, I investigate the usage of metaphor in education, and
focus on one particular aspect of education that is familiar and
important to me, that of undergraduate statistics.  At the
undergraduate level, \dquote{statistics} is both a practical and a
mathematical discipline; it is practical in that many students will be
expected to learn the skills required to extract useful information
from data, but mathematical in the sense that many important
statistical ideas can only be understood in a relatively sophisticated
mathematical context.

\section{Metaphors we live by}

\setlength{\epigraphwidth}{.7\textwidth} % default is .4
\begin{singlespace}
\epigraph{Metaphors may create realities for us, especially social
  realities.  A metaphor may thus be a guide for future action. Such
  actions will, of course, fit the metaphor.  This will, in turn,
  reinforce the power of the metaphor to make experience coherent.  In
  this sense, metaphors can be self-fulfilling
  prophecies}{\mycite[page 156]{lakoff1980}}\end{singlespace}

The publication of \emph{Metaphors we live by} \parencite{lakoff1980}
ushered in the modern era of metaphorical thinking.  In this short and
accessible book, the authors argue that metaphor is in fact a
cognitive phenomenon (in which one thing is considered in terms of
another), which happens to have a linguistic manifestation (for
example, writing that \dquote{cancer is a \metaphor{battle}}.

As \mycite{lakoff1980} and others point out, metaphors are not confined
to literary or rhetorical contexts, and are used frequently in
everyday language.  For example, one might say student attendance was
\metaphor{up}, indicating an increased in attendance.
\mycite{lakoff1980} would classify this as an orientational metaphor:
the word \metaphor{up} denotes increased altitude in literal speech,
but is used here as part of a systematic scheme whereby an
orientational vehicle (up, down, in, out, etc) refers to a
non-spatial topic (happy, sad, rich, hot, cold, etc).  Such systematic
schemes are traditionally denoted using small capitals, as in:
\conmet{the mind is a container}; other apposite examples might
include \conmet{communication is transfer}, or \conmet{education is
  acquisition}.  When sensitized to the issue, one tends to see
conceptual metaphors everywhere.  For example, we \metaphor{pass} an
exam (\conmet{life is a journey} or possibly \conmet{challenges are
  obstacles}), and either \metaphor{progress through} college
(\conmet{situations are containers}), or \metaphor{drop out}
(\conmet{down is bad}).

Such quotidian use of metaphor can easily pass unnoticed. However,
\mycite{lakoff1980} argue persuasively that metaphors can and do inform
our conceptual system and influence our actions; they make a strong
case for a serious study of metaphor in a variety of contexts.
\mycite{bereiter2005}, in particular, cautions against the \conmet{mind
  is a container} metaphor, deriding it as ``folk theory of mind'',
and goes on to argue that its uncritical adoption is damaging to
education.

\section{Cognitive metaphor theory}

Cognitive metaphors such as \conmet{the mind is a container} are
rarely used directly in speaking or writing but are influential in
that they function at the level of thought, below language
\parencite{deignan2005}; it is common to say that linguistic metaphors
          {\em realize} conceptual metaphors.  Deignan goes on to give
          five tenets of cognitive metaphor theory:

\begin{enumerate}
\item Metaphors structure thinking
\item Metaphors structure knowledge
\item Metaphor is central to abstract language
\item Metaphor is grounded in physical experience
\item Metaphor is ideological.
\end{enumerate}

These five tenets of metaphor theory may be used to structure thinking
of corpus analysis when considering metaphors in education.

\subsection{Metaphors structure thinking and knowledge}

Corpus linguistics is the study of language as expressed in naturally-occurring
text; the standard methodology (annotation-abstraction-analysis) is
due to \mycite{wallis2001}, although this is not well-suited to
analysis of metaphor in education.  Metaphorical analysis of corpora is discussed
by~\mycite{deignan2005}, in the context of understanding metaphor per
se.  This approach is not especially suitable for the investigation of
a specific metaphor (or set of metaphors), as here, but the
methodology has been adopted by~\mycite{charteris-black2004} and
others, to assess specific areas of language use.  The author takes
corpora from political rhetoric, financial reporting, and religious
texts, and assesses metaphor use in a range of written corpora.  But
note that the texts all have one feature in common: they are written
specifically to convince the reader to think in a particular way, to
adopt a particular stance. The techniques used in such corpus analysis
are suitable for political rhetoric or similar texts, but tend to
focus on metaphor in general rather than specific metaphors such as
the level metaphor or the foundation metaphor.

\subsection{Metaphor is central to abstract language}

In a very heavily-cited work, \mycite{reddy1993} discusses one very
important concept in education, that of communication.  In essence,
his thesis is that communication is frequently discussed using the
\metaphor{conduit} metaphor \parencite{lakoff1980}; he later characterized this
as a distillation of the conceptual metaphors \conmet{ideas are
  objects}, \conmet{linguistic expressions are containers}, and
\conmet{communication is sending}.  Reddy states that stories about
communication (of which education figures prominently) are largely determined by semantic structures and it is
clear that the primary structure he has in mind is (linguistic)
metaphor.

\subsection{Metaphors are grounded in physical experience}

Many metaphors have their origins in physical space; \mycite{lakoff1980}
term these \emph{orientational} metaphors.  Orientations may include
up-down, in-out, central-peripheral, near-far, shallow-deep, and so on; the best
examples include \conmet{happy is up}; \conmet{intimacy is proximity}
and it is clear that the \metaphor{level} and \metaphor{foundation} metaphors are at least
partially spatial.  \metaphor{Ascending the levels} of a course of
study and indeed the foundation metaphor itself would be exemplars of
\conmet{sophisticated is up}.

\subsection{Metaphors are ideological}

Although most authors discussing metaphor interpret the word
\dquote{ideological} in terms of either corporate or national
policy~\citep{perrez2015}, it is worth remembering that ideology can
apply to any system of ideas by any group or community; here the
relevant community would be tertiary teachers and students.  Consider
the \metaphor{level} metaphor as an example.  Whether the implications
of the metaphor constitute an ideology (or indeed whether
considering current mathematical education's level-based philosophy
from an ideological perspective is a coherent or desirable scheme) remains an open
question.


\section{Research methodologies}

\begin{singlespace}
\begin{quote}
  Metaphor is both important and odd---its importance odd and its
  oddity important---\mycite[page 125]{goodman1979}
\end{quote}
\end{singlespace}

A systematic citation analysis of~\mycite{lakoff1980}'s seminal work
revealed few studies of metaphorical language in specific contexts;
the majority of those discussed the ramifications of metaphor in the
medical profession.  \mycite{montgomery1991} discusses
medicine-as-combat and~\mycite{harrington2012} presents cancer-as-war.

This type of research on metaphor in education is sparse, and what
does exist is largely focused on its use by teachers at primary or
secondary level \citep{munby1986,cameron2003}.  \mycite{paechter2004}
is one of the very few works discussing the level, foundation, or
framework metaphor; perhaps this is because their use constitutes an
\dquote{unquestioned norm}, which renders investigation difficult.

Established research methodologies appear to be poorly suited to this
work.  Hermeneutic analysis would seem to be inappropriate on the
grounds that there is no canonical text~\parencite{mantzavinos2005}.
It might be argued that the Education Act 1989 or the New Zealand
Qualifications Framework constitute revelatory texts but a close
word-by-word reading of these documents would seem to be unlikely to
produce any insight into metaphor as used in the field.

Documentary analysis techniques~\parencite{fitzgerald2012} might be
more promising although they are geared towards historical rather than
social analysis and again there is no canonical document to analyze.

\mycite{harrington2012} presents one of the very few
discipline-specific surveys of metaphor usage in discourse, in this
case medical science.  She presents a careful and insightful study of
metaphor use in (written) discourse about cancer---military and
journey metaphors figure prominently---but her work does not appear to
fall into any recognizable research methodology.  She does analyze
various documents which she believes to be representative or
influential, but makes no attempt at a systematic survey or to trace
any historical drift.  Having said that, her work is extremely
convincing, and very heavily cited, and provided inspiration for the
present work.

In an educational context, \mycite{cameron2003} considers metaphor in
spoken English, using 10- and 11- year old students as informants.
She employs concepts from applied linguistics to infer students'
learning strategies, and to improve teaching methods.  Her data
comprised natural utterances (and a small amount of written material)
and her conclusions centered around detailed analysis of carefully
selected, and mostly very short, fragments.

These two approaches contrast sharply in their epistemological
assumptions about the nature of knowledge: \citeauthor{harrington2012}
is clearly oriented toward propositional knowledge [of patients' and
  doctors' speech]; while Cameron is more focused on knowledge by
acquaintance: she makes little attempt to generalize her findings
beyond the confines of her classroom study.

Both works, however, share a common understanding of ontology: both
writers maintain the existence of an entity, here the linguistic
phenomenon of metaphor, with certain properties.  The assumption is
that it is possible to observe this entity, albeit imperfectly, and
make inferences about its nature and properties.  Admittedly, Cameron
observes human behaviour through the lens of linguistic theory, while
Harrington considers only published research, but both clearly have an
entity in mind which they wish to learn about.

It appears that different individuals hold widely differing
interpretations of the vehicles \dquote{level}, \dquote{foundation}
and \dquote{framework} when used in educational policy.  For example,
to me the level metaphor involves ascending a multistory building; but
many of my colleagues view the level metaphor as actually constructing
a large structure or edifice, something that was not part of my
thinking.

One possibility might be to interview, say, ten practising mathematics
or statistics lecturers in a semi-structured interview; this might
produce interesting results.  However, one potential pitfall might be
the recruitment of informants, who would need to be chosen carefully:
the interviewees would not be a random sample, but on the other hand
statistical validity is not an issue in studies of this
type~\citep{ribbins2012}.  Also, merely stating the purpose of the
interview might distort their perceptions; but an appropriately
structured system of questioning might be able to mitigate this
deficiency.  Further work would be needed to assess whether this
approach would be worthwile.

\mycite{cornelissen2012} points out that metaphor is a commonly
considered aspect of organizational theory.  \mycite{amernic2007}, for
example, consider the metaphors in a series of letters written by a
CEO to his shareholders; and \mycite{tourish2012} consider the
metaphors used by disgraced bankers following the 2006 financial
crisis.  These authors study \emph{root metaphors}---that is,
metaphors which provide \dquote{rich summaries of the world and reveal
  dominant and powerful ways of seeing}.  Root metaphors differ from
conceptual metaphors in that a root metaphor is generally used as a
rhetorical device with the intent to frame public discussion.
\mycite{tourish2012} do not classify root metaphor as deception as
such, but rather as systematic distortion; compare conceptual
metaphor, in which the emphasis lies more in its role as a cognitive
mechanism.

Drawing on these ideas, this thesis presents a semi-systematic survey
of language as used in the various phases of a standard undergraduate
statistics course.  

\section{Metaphors in tertiary education}


If~\citeauthor{lakoff1980} are correct in their view of metaphor being
a cognitive---rather than a linguistic----phenomenon, then it is clear
that metaphor will have an important part to play in education.  This
thesis considers metaphors as used in tertiary education, with a focus
on undergraduate statistics.

The overwhelming majority of studies of metaphor in an educational
context focus on its use by teachers, as opposed to students or
administrators.  \mycite{willox2010} observe that most literature
focuses on metaphor as an \dquote{instructor driven pedagogical
  tool}); however, students too create and use metaphor in many
educational contexts including learning activities such as lectures as
well as in assessment.

In this thesis, I consider metaphor as used in written and spoken
language by instructors and students in the following aspects of
undergraduate education.  The focus is on the case of undergraduate
statistics; most of my teaching is in this area.

The thesis chapters cover the different aspects of an undergraduate
statistics course; they are ordered roughly in chronological order as
experienced by the lecturer.  The aspects covered are as follows:

\begin{itemize}
\item Metaphor in educational planning (level/foundation metaphors and
  possibly the factory metaphor or the acquisition/participation
  metaphor) (chapter~\ref{chapter2});
\item Metaphor educational administration (chapter~\ref{chapter3});
\item Metaphor used in spoken or recorded lectures (chapter~\ref{chapter4})
\item Metaphor in statistics textbooks (chapter~\ref{chapter5})
\item Metaphor in statistics assessment (chapter~\ref{chapter6})
\item Metaphor in course evaluation (chapter~\ref{chapter7}).
\end{itemize}


\section{Conclusions}

Following \citeauthor{lakoff1980}'s seminal publication, many scholars
have written about the power of metaphor to shape and guide thinking.
One way in which we can study the effect of metaphor is via its
linguistic manifestation, which is open to study in both corpora and
spoken English.  Metaphor is known to be influential in thinking about
education (\cite{sfard1998}'s acquisition metaphor; the factory
metaphor), and also more directly in educational policy documents, and
even more directly in classroom practice.  There are a few fields,
medical science in particular, where particular metaphors (martial
metaphor for cancer) have been studied and shown to have powerful and
sometimes harmful effects.  Given the undeniable power of metaphor,
one might expect that metaphorical language to be an important
component of statistical education.  The extent to which this is true
is the topic of this thesis.

\begin{singlespace}
\epigraph{We have no literal language for talking about what
  thoughts do\ldots [there is] no possible way of literally saying
  what has to be said: so that if it is to be said at all, metaphor
  is essential}{\mycite[page 49]{ortony1975}}

\epigraph{Your brain does not process information, retrieve knowledge
  or store memories}{\mycite[front cover]{epstein2016}}

\epigraph{The \dquote{dead metaphor} account misses an important
  point: namely, that what is deeply entrenched, hardly noticed, and
  thus effortlessly used is most active in our thought.  The metaphors
  listed above may be highly conventional and effortlessly used, but
  this does not mean that they have lost their vigor in thought and
  that they are dead.  On the contrary, they are \dquote{alive} in the
  most important sense---they govern our thought: they are
  \emph{metaphors we live by}.  One example of this involves our
  comprehension of the mind as a machine.}{\mycite[page
    12]{kovecses2010}}

\epigraph{Just over a year ago, on a visit to one of the world's most
  prestigious research institutes, I challenged researchers there to
  account for intelligent human behaviour without reference to any
  aspect of the [Information Processing] metaphor.  \emph{They
    couldn't do it}, and when I politely raised the issue in
  subsequent email communications, they still had nothing to offer
  months later.  They saw the problem.  They didn’t dismiss the
  challenge as trivial.  But they couldn’t offer an alternative.  In
  other words, the IP metaphor is \dquote{sticky}.  It encumbers our
  thinking with language and ideas that are so powerful we have
  trouble thinking around them\ldots [T]he idea that humans must be
  information processors just because computers are information
  processors is just plain silly, and when, some day, the IP metaphor
  is finally abandoned, it will almost certainly be seen that way by
  historians, just as we now view the hydraulic and mechanical
  metaphors to be silly.}{\mycite[unpaginated]{epstein2016}}
\end{singlespace}
