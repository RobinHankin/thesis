\chapter{Conclusions}
\label{chapter8}

In this thesis, I have considered metaphorical language as used in
various aspects of an undergraduate statistics course.  Conceptual
metaphor, whether viewed as a cognitive or a linguistic phenomenon,
appears in each of these aspects to a greater or lesser extent.

Metaphor is such a ubiquitous phenomenon that it is easy to overlook,
but its influence is deep and pervasive.  Conceptual metaphors such as
\conmet{the mind is a container} directly influence educational
practice, not always for the better: for the objects \metaphor{in} the
mind frequently include misconceptions, gaps, and fallacies.  And
surely if the mind is a container and \conmet{concepts are objects},
then this directly implies that the measurement model is the correct
way to assess these objects.

Taking the \metaphor{level} metaphor as another example, this
metaphorical system structures thinking about education and encourages
perceptions about mathematics (such that mathematical knowledge is
organised into a discrete set of standalone building blocks possessing
a natural sequence) that are an educational policy dream but a
pedagogical nightmare.

Statistics courses share many points of similarity with mathematics
courses which have been more widely and thoroughly studied; it is not
clear to what extent statistics should be viewed as a branch of
mathematics.

Statistics as a discipline uses standard mathematical metaphors such
as the \conmet{basic metaphor of infinity} but this does not have the
central role that it has in (for example) analysis or topology.  

The \metaphor{agency} and \metaphor{personification} metaphors occur
frequently in course descriptors: a statistics course is implicitly
embued with the ability to act independently and intelligently.  Such
language encourages the perception of the course content itself as
active and personified. 

Metaphor is a component of spoken lectures, statistics textbooks, and
undergraduate assessment.  Its use is comparable in frequency and
intensity to that in mainstream mathematics, and similar metaphorical
constructions are found.


%
%
%\epigraph{Just over a year ago, on a visit to one of the world's most
%  prestigious research institutes, I challenged researchers there to
%  account for intelligent human behaviour without reference to any
%  aspect of [the Information Processing metaphor].  \emph{They
%    couldn't do it}, and when I politely raised the issue in
%  subsequent email communications, they still had nothing to offer
%  months later.  They saw the problem.  They didn’t dismiss the
%  challenge as trivial.  But they couldn’t offer an alternative.  In
%  other words, the IP metaphor is \dquote{sticky}.  It encumbers our
%  thinking with language and ideas that are so powerful we have
%  trouble thinking around them\ldots [T]he idea that humans must be
%  information processors just because computers are information
%  processors is just plain silly, and when, some day, the IP metaphor
%  is finally abandoned, it will almost certainly be seen that way by
%  historians, just as we now view the hydraulic and mechanical
%  metaphors to be silly.}{\mycite{epstein2016}}
%
