\chapter{Conclusions}
\label{chapter8}

In this thesis, I have looked across the AUT educational system within
which an undergraduate statistics course was couched, and made a study
of metaphorical language as used in the various aspects of such a
course.  The aspects range from course descriptors, through spoken
lectures, to assessment cues and responses.  For each aspect, I have
identified metaphorical language using the Pragglejaz procedure, and
analyed using methods drawn from similar studies in the fields of
medicine, law, and business.  Strengths of this methodology include
wide coverage over various aspects of a course, and leverage of
similar metaphorical studies in other disciplines.  Weaknesses include
lack of systematic and statistically representative corpus of data,
although it is unclear how this could be achieved.

Conceptual metaphor, whether viewed as a cognitive or a linguistic
phenomenon, appears in each of these aspects of a statistics course,
to a greater or lesser extent.  Metaphor is such a ubiquitous
phenomenon that it is easy to overlook, but it is undoubtedly a
component of many aspects of statistical education.

Following \citeauthor{lakoff1980}'s seminal publication, metaphor is
frequently viewed as a cognitive phenomenon which may be structured in
terms of conceptual metaphors such as \conmet{argument is war}.  One
way in which the metaphor may be studied is via its linguistic
manifestation in both corpora and spoken English, and both are carried
out in this thesis.  Metaphor is shown to be influential in
educational thought---notable examples including the acquisition
metaphor of~\mycite{sfard1998}.

Taking the \metaphor{level} metaphor as another example, this
metaphorical system structures thinking about education and encourages
particular perceptions about mathematics---specifically, that
mathematical knowledge is organised into a discrete set of standalone
building blocks possessing a natural sequence.  The result of this
perception may well facilitate the organization of education (into
annual cohorts), but the pedagogical effect of this is not clear.

It is not clear whether statistics education should be viewed as a
subset of mathematics education, or a separate field.  Statistics
courses do share many points of similarity with mathematics courses
such as hierarchical discipline structure, heavy dependence on
abstraction, and an emphasis on numerical and algebraic techniques.
The two subjects are also similar in their use of standard
mathematical metaphors such as the \conmet{basic metaphor of infinity}
but this does not have the central role that it has in (for example)
analysis or topology.

Metaphor is a component of educational administration.  In particular,
metaphorical language is a commonly occurring component of course
descriptors.  Pre-eminent among the metaphors used in this context is
the \metaphor{agency} metaphor: a statistics course is implicitly
imbued with the ability to act independently and intelligently.  Such
language encourages the perception of the course content itself as
active and personified.  Such metaphors are often used deliberately to
counteract the somewhat impersonal nature of many course descripors.
Metaphor occurs in other parts of educational planning too, notably
institutional \emph{graduate outcomes}.  The metaphorical language
appears to promote the acquisition metaphor at the expense of the
participation metaphor.

Metaphor is a component of spoken lectures, but only rarely would the
metaphors used be recognised as such by the audience.  Also, in spoken
lectures the metaphors seldom draw attention to the language used.

Metaphor is an unavoidable component of statistics textbooks.  In the
three textbooks chosen for detailed study, metaphor was rare, with the
exception of the basic metaphor of infinity, and the mathematicians'
\dquote{we}, which used throughout mathematics and is not specific to
statistics.

It is not straightforward to analyse assessed work for metaphor
(ethical considerations make informed consent difficult to obtain).
However, a small number of assessed pieces of work is available, and
it appears that tertiary statistcs students do employ metaphor when
assessed.  The British Academic Written English corpus is one of the
very few corpuses of assessed undergraduate student writing available,
and this contains two pieces of assessed work which were part of a
statistics course.  The students used the conceptual metaphor
\conmet{equations are tools} in much the same way as textbooks do.

Course evaluation is another component of statistics education in
which metaphor is potentially used.  Typically, course evaluation
forms include a free-form comments section in which students are
invited to give feedback on a course.  It is again difficult to study
students' use of metaphor in this context and only a limited amount of
research has been carried out to date.  Students appear to use
metaphor sparingly, and those that are used are similar to the
metaphors used in institutional documents such as course descriptors.


%
%
%\epigraph{Just over a year ago, on a visit to one of the world's most
%  prestigious research institutes, I challenged researchers there to
%  account for intelligent human behaviour without reference to any
%  aspect of [the Information Processing metaphor].  \emph{They
%    couldn't do it}, and when I politely raised the issue in
%  subsequent email communications, they still had nothing to offer
%  months later.  They saw the problem.  They didn’t dismiss the
%  challenge as trivial.  But they couldn’t offer an alternative.  In
%  other words, the IP metaphor is \dquote{sticky}.  It encumbers our
%  thinking with language and ideas that are so powerful we have
%  trouble thinking around them\ldots [T]he idea that humans must be
%  information processors just because computers are information
%  processors is just plain silly, and when, some day, the IP metaphor
%  is finally abandoned, it will almost certainly be seen that way by
%  historians, just as we now view the hydraulic and mechanical
%  metaphors to be silly.}{\mycite{epstein2016}}
%
