%%%%%%%%%%%%%%%%%%%%%%%%%%%%%%%%%%%%%%%%%
% Masters/Doctoral Thesis 
% LaTeX Template
% Version 2.3 (25/3/16)
%
% This template has been downloaded from:
% http://www.LaTeXTemplates.com
%
% Version 2.x major modifications by:
% Vel (vel@latextemplates.com)
%
% This template is based on a template by:
% Steve Gunn (http://users.ecs.soton.ac.uk/srg/softwaretools/document/templates/)
% Sunil Patel (http://www.sunilpatel.co.uk/thesis-template/)
%
% Template license:
% CC BY-NC-SA 3.0 (http://creativecommons.org/licenses/by-nc-sa/3.0/)
%
%%%%%%%%%%%%%%%%%%%%%%%%%%%%%%%%%%%%%%%%%

%----------------------------------------------------------------------------------------
%	PACKAGES AND OTHER DOCUMENT CONFIGURATIONS
%----------------------------------------------------------------------------------------




\documentclass[
12pt, % The default document font size, options: 10pt, 11pt, 12pt
oneside, % Two side (alternating margins) for binding by default, uncomment to switch to one side
%chapterinoneline,% Have the chapter title next to the number in one single line
english, % ngerman for German
%singlespacing, % Single line spacing, alternatives: onehalfspacing or doublespacing
onehalfspacing,
%draft, % Uncomment to enable draft mode (no pictures, no links, overfull hboxes indicated)
%nolistspacing, % If the document is onehalfspacing or doublespacing, uncomment this to set spacing in lists to single
%liststotoc, % Uncomment to add the list of figures/tables/etc to the table of contents
%toctotoc, % Uncomment to add the main table of contents to the table of contents
%parskip, % Uncomment to add space between paragraphs
%nohyperref, % Uncomment to not load the hyperref package
headsepline, % Uncomment to get a line under the header
]{MastersDoctoralThesis} % The class file specifying the document structure

\usepackage{epigraph}
\usepackage[english,american]{babel}
\usepackage[utf8]{inputenc} % Required for inputting international characters
\usepackage[T1]{fontenc} % Output font encoding for international characters

\usepackage{amssymb}
\usepackage{amsmath}

\usepackage{palatino} % Use the Palatino font by default
\usepackage[autostyle=true]{csquotes} % Required to generate language-dependent quotes in the bibliography
 
\usepackage[backend=biber,style=apa,natbib=true]{biblatex}
\newcommand\mycite[2][]{\citeauthor{#2}\ (\citeyear{#2})\ifx#1\undefined\else, #1\fi}


\usepackage{quotchap}
\usepackage{color}
\usepackage{xspace}  % needed for \wmcf

\usepackage{dirtytalk}

\definecolor{darkblue}{rgb}{0,0,0.5}
\definecolor{darkgreen}{rgb}{0,0.3,0.1} % defined in MastersDoctoralThesis.cls atn urlcolor
\definecolor{darkbrown}{rgb}{0.4,0.1,0.1} % defined in MastersDoctoralThesis.cls atn urlcolor

\DeclareLanguageMapping{english}{american-apa}

\addbibresource{metaphor.bib} % The filename of the bibliography




%----------------------------------------------------------------------------------------
%	MARGIN SETTINGS
%----------------------------------------------------------------------------------------

\geometry{
	paper=a4paper, % Change to letterpaper for US letter
	inner=2.5cm, % Inner margin
	outer=3.8cm, % Outer margin
	bindingoffset=2cm, % Binding offset
	top=1.5cm, % Top margin
	bottom=1.5cm, % Bottom margin
	%showframe,% show how the type block is set on the page
}

%----------------------------------------------------------------------------------------
%	THESIS INFORMATION
%----------------------------------------------------------------------------------------

\thesistitle{Metaphors we teach by} % Your thesis title, this is used in the title and abstract, print it elsewhere with \ttitle
\supervisor{Andrew J. C. \textsc{Begg}} %  print it elsewhere with \supname
\examiner{} % Your examiner's name, print it elsewhere with \examname
\degree{Master of Education}
\author{Robin K. S. \textsc{Hankin}} 
\addresses{} % Your address, this is not currently used anywhere in the template, print it elsewhere with \addressname

\subject{Education} % Your subject area, this is not currently used anywhere in the template, print it elsewhere with \subjectname
\keywords{} % Keywords for your thesis, this is not currently used anywhere in the template, print it elsewhere with \keywordnames
\university{\href{http://www.university.com}{Auckland University of Technology}} % Your university's name and URL, this is used in the title page and abstract, print it elsewhere with \univname
\department{\href{http://aut.ac.nz}{School of Education}} % Your department's name and URL, this is used in the title page and abstract, print it elsewhere with \deptname
\group{\href{http://researchgroup.university.com}{Research Group Name}} % Your research group's name and URL, this is used in the title page, print it elsewhere with \groupname
\faculty{\href{http://faculty.university.com}{Faculty Name}} % Your faculty's name and URL, this is used in the title page and abstract, print it elsewhere with \facname

\hypersetup{pdftitle=\ttitle} % Set the PDF's title to your title
\hypersetup{pdfauthor=\authorname} % Set the PDF's author to your name
\hypersetup{pdfkeywords=\keywordnames} % Set the PDF's keywords to your keywords

\begin{document}

\newcommand{\conmet}[1]{{\textsc{#1}}} % conmet == CONceptual METaphor
\newcommand{\metaphor}[1]{{\textsl{#1}}}
\newcommand{\paper}[1]{{\textsf{#1}}} % paper
\newcommand{\dquote}[1]{``{#1}''}   % double quote
\newcommand{\squote}[1]{`{#1}'}      % single quote
\newcommand{\minsec}[2]{{[}#1:#2{]}}
\newcommand{\squareb}[1]{{[}#1{]}}
\newcommand{\wmcf}{{\sc{wmcf}}\xspace}
%\newcommand{\bmi}{{\sc{bmi}}\xspace}
\newcommand{\bmi}{\emph{\textsc{bmi}}\xspace}

%\frontmatter % Use roman page numbering style (i, ii, iii, iv...) for the pre-content pages

\pagestyle{plain} % Default to the plain heading style until the thesis style is called for the body content

%----------------------------------------------------------------------------------------
%	TITLE PAGE
%----------------------------------------------------------------------------------------

\begin{titlepage}
\begin{center}

{\scshape\LARGE \univname\par}\vspace{1.5cm} % University name
\textsc{\Large Master's Thesis}\\[0.5cm] % Thesis type

\HRule \\[0.4cm] % Horizontal line
{\huge \bfseries \ttitle\par}\vspace{0.4cm} % Thesis title
\HRule \\[1.5cm] % Horizontal line
 
\begin{minipage}[t]{0.4\textwidth}
\begin{flushleft} \large
\emph{Author:}\\
\href{http://www.johnsmith.com}{\authorname} % Author name - remove the \href bracket to remove the link
\end{flushleft}
\end{minipage}
\begin{minipage}[t]{0.4\textwidth}
\begin{flushright} \large
\emph{Supervisor:} \\
\href{http://www.jamessmith.com}{\supname} % Supervisor name - remove the \href bracket to remove the link  
\end{flushright}
\end{minipage}\\[3cm]
 
\large \textit{A thesis submitted to Auckland University of Technology\\ in partial fulfilment of the requirements for the degree of \degreename}\\[0.3cm] % University requirement text
\textit{in the}\\[0.4cm]
%\groupname\\\deptname\\[2cm] % Research group name and department name
\deptname\\[2cm] % Research group name and department name
 
{\large \today}\\[4cm] % Date
%\includegraphics{Logo} % University/department logo - uncomment to place it
 
\vfill
\end{center}
\end{titlepage}




%----------------------------------------------------------------------------------------
%	ABSTRACT PAGE
%----------------------------------------------------------------------------------------

\begin{abstract}
\addchaptertocentry{\abstractname} % Add the abstract to the table of contents

Many researchers have commented on the use of metaphor in education.
This thesis considers the extent to which metaphor and metaphorical
language is used in the case of undergraduate statistics education.

There are many aspects of an undergraduate statistics course that
employ metaphor.  I will consider a typical undergraduate statistics
course in terms of the following components: educational policy,
educational administration, spoken lectures, statistics textbooks,
assessment, and course evaluation.  Metaphor is present in all these
aspects, and I discuss its importance in each aspect in turn.

Although much research into tertiary mathematics education has been
undertaken, only a small proportion is devoted to statistics
education; and it is not clear whether statistics should be viewed as
a subset of mathematics for this purpose.  I discuss the extent to
which insights from the study of metaphor in the teaching of
mathematics are applicable to the teaching of statistics.
\end{abstract}

%----------------------------------------------------------------------------------------
%	LIST OF CONTENTS/FIGURES/TABLES PAGES
%----------------------------------------------------------------------------------------

\tableofcontents % Prints the main table of contents

%\listoffigures % Prints the list of figures

%\listoftables % Prints the list of tables

%----------------------------------------------------------------------------------------
%	ABBREVIATIONS
%----------------------------------------------------------------------------------------

%\begin{abbreviations}{ll} % Include a list of abbreviations (a table of two columns)
%
%\textbf{LAH} & \textbf{L}ist \textbf{A}bbreviations \textbf{H}ere\\
%\textbf{WSF} & \textbf{W}hat (it) \textbf{S}tands \textbf{F}or\\
%
%\end{abbreviations}

%----------------------------------------------------------------------------------------
%	DEDICATION
%----------------------------------------------------------------------------------------

%\dedicatory{For/Dedicated to/To my\ldots} 



%----------------------------------------------------------------------------------------
%	DECLARATION PAGE
%----------------------------------------------------------------------------------------

\begin{declaration}
\addchaptertocentry{\authorshipname}

\noindent
I hereby declare that this submission is my own work and that, to the
best of my knowledge and belief, it contains no material previously
published or written by another person (except where explicitly
defined in the acknowledgements), nor material which, to a substantial
extent has been submitted for the award of any degree or diploma of a
university or other institution of higher learning.

\rule{0mm}{20mm}

\noindent Signed:\\
\rule[0.5em]{25em}{0.5pt} % This prints a line for the signature
 
\noindent Date:\\
\rule[0.5em]{25em}{0.5pt} % This prints a line to write the date
\end{declaration}

\cleardoublepage






%---------------------------------------------
%	QUOTATION PAGE
%---------------------------------------------

\vspace*{0.2\textheight}


\noindent\begin{singlespace}\enquote{\itshape
Educators have been talking about \metaphor{foundations} for so long
that it no longer seems metaphorical---but that is when metaphors
become the most dangerous.  All that \dquote{foundation} literally
means in the context of instruction is something taught initially in
order to facilitate future learning
}\bigbreak\end{singlespace}
\hfill\mycite{bereiter2005}

\rule{0mm}{30mm}
\noindent\begin{singlespace}\enquote{\itshape We believe that
  educators need to organize curriculum, instruction, assessment and
  schools using metaphors that they choose intentionally and
  deliberately.  Metaphors do shape educational practice.  If we do
  not choose our metaphors, our metaphors will simply choose
  us
}\bigbreak\end{singlespace}
\hfill\mycite{badley2012} % p7

\rule{0mm}{30mm}
\noindent\begin{singlespace}\enquote{\itshape Language, and
  metaphorical language in particular, constructs the worlds in which
  we live. Consequently, the metaphors we use to talk and to think
  about education have a profound effect on what it is like to be
  educated in the societies in which we use them.  We have a long
  history of using metaphors in education, both as calls to arms and
  as supposed neutral descriptors.  Both have ideological force, and
  both have the potential to steer us into particular ways of thinking
  about, and going about, the processes of
  education}\bigbreak\end{singlespace}
\hfill\mycite{paechter2004}

%\begin{centering}
%\includegraphics[width=10cm]{xkcd_metaphor.png}
%\end{centering}

%----------------------------------------------------------------------------------------
%	ACKNOWLEDGEMENTS
%----------------------------------------------------------------------------------------

%\begin{acknowledgements}
%\addchaptertocentry{\acknowledgementname} % Add the acknowledgements to the table of contents

%The acknowledgments and the people to thank go here, don't forget to include your project advisor\ldots

%\end{acknowledgements}




%----------------------------------------------------------------------------------------
%	THESIS CONTENT - CHAPTERS
%----------------------------------------------------------------------------------------

%\mainmatter % Begin numeric (1,2,3...) page numbering

\pagestyle{thesis} % Return the page headers back to the "thesis" style

% Include the chapters of the thesis as separate files from the Chapters folder
% Uncomment the lines as you write the chapters


  \begin{savequote}[105mm]\begin{singlespace}
Sometimes, there is little awareness of the metaphors that guide our
behaviours and shape our institutional structures.  We may not even be
aware of the negative consequences of the metaphors we live
by\end{singlespace} \qauthor{Perc Marland}
\end{savequote}

\chapter{Introduction and literature review} % Main chapter title
\label{chapter1} % For referencing the chapter elsewhere, use \ref{Chapter1} 

Metaphor is usually defined as an invitation for the reader or listener to
consider one thing in terms of another \parencite{steen1994}, although
its definition is far from clear; \mycite{knowles2006} and others
state that metaphor is generally easier to recognize than to define.  Modern
treatments go to some lengths to refute the notion that metaphor is
confined to poetical use, pointing to the ubiquity of metaphor at all
levels of formality \parencite{deignan2005} in spoken
\parencite{cameron2003} and written \parencite{charteris-black2004}
English.  Textbooks usually distinguish metaphor from simile, in which
X is held to be \emph{like} Y; compare metaphor, in which X is said to
\emph{be} Y, although this characterization does not stand close
scrutiny.

Here, I will investigate the use of metaphorical language in
undergraduate statistics education.  In this thesis, \emph{metaphor}
will be interpreted as meaning any non-literal language and will thus
include idiosyncratic use such as word problems.

Metaphor is a ubiquitous phenomenon in language and this thesis
examines metaphor as used in education, specifically undergraduate
statistics education.  Although research into tertiary mathematics
education is reasonably well represented, only a small proportion is
devoted to statistics education; and it is not clear whether statistics
should be viewed as a subset of mathematics.

An example of metaphor in an educational context might be \dquote{this
  student is the \metaphor{cream} of the cohort}\footnote{Slanted type
  is used to denote metaphor, as in \dquote{all the world's a
    \metaphor{stage}}.}.  Students are not dairy products and in this
case we are invited to compare the student's elite academic
performance (in relation to his peers) with the most desirable
component of milk, viz. the cream. Observe that the comparison to
cream stimulates only a small part of the semantic connotations of
actual cream: specifically the desirable properties of tastiness and
expensiveness.  Cream itself has many undesirable features: it is
unhealthy, it is sickly, and it is fattening but these properties are
never used metaphorically.

A more pertinent example might be to describe a student as \dquote{top
  of the class}.  \mycite{paechter2004} considers this metaphor to drive
a particular view of assessment, specifically one based on comparisons
between students rather than on whether each student has achieved the
learning objectives of the unit of study at hand.

Skillful use of metaphor does not necessarily include any form of the
verb \emph{to be}, for example:

\begin{quote}
      An iron curtain has descended across the continent\\
---W. Churchill (attributed; circa 1946)
\end{quote}

In this famous quote, the verb is \emph{to descend}: the reader
(originally the listener) must infer that the dividing boundary
developing between East and Western Europe is to be considered a
physical object. Note too Churchill's mentioning iron, its utilitarian
connotations underscoring the perceived economic poverty of communism
when compared with the prosperous West.  Curtains literally descending
would be familiar to Churchill's theatre-going audience as marking the
end of a performance, and it is reasonable to believe that Churchill
was alluding to communism's extinguishing of democratic rights.

In this thesis, metaphor is discussed using standard terminology: the
\emph{topic} (sometimes \emph{tenor}) is the concept being described,
and the \emph{vehicle} (sometimes \emph{figure}) is the concept used
to describe the topic.  In Churchill's quote above, the topic would be
the political divide between East and West Europe, and the vehicle
would be a stage curtain made of iron.

It is clear that such metaphors can be informative about a speaker's
thoughts; skillful orators can utilize metaphors effectively to
encapsulate the mood of a nation, or indeed to provoke debate about a
nation's educational system~\citep{robinson2011}.  Theoretical
linguists and sociologists have published a large amount of material
dissecting and analyzing this form of language from many perspectives:
workers have studied the effective use of metaphor (by politicians;
see \mycite{perrez2015}, for example), and the effect of rhetorical
metaphor on listeners~\citep{keating2015}.

In this thesis, I investigate the usage of metaphor in education, and
focus on one particular aspect of education that is familiar and
important to me, that of undergraduate statistics.  At the
undergraduate level, \dquote{statistics} is both a practical and a
mathematical discipline; it is practical in that many students will be
expected to learn the skills required to extract useful information
from data, but mathematical in the sense that many important
statistical ideas can only be understood in a relatively sophisticated
mathematical context.

\section{Metaphors we live by}

\setlength{\epigraphwidth}{.7\textwidth} % default is .4
\begin{singlespace}
\epigraph{Metaphors may create realities for us, especially social
  realities.  A metaphor may thus be a guide for future action. Such
  actions will, of course, fit the metaphor.  This will, in turn,
  reinforce the power of the metaphor to make experience coherent.  In
  this sense, metaphors can be self-fulfilling
  prophecies}{\mycite[page 156]{lakoff1980}}\end{singlespace}

The publication of \emph{Metaphors we live by} \parencite{lakoff1980}
ushered in the modern era of metaphorical thinking.  In this short and
accessible book, the authors argue that metaphor is in fact a
cognitive phenomenon (in which one thing is considered in terms of
another), which happens to have a linguistic manifestation (for
example, writing that \dquote{cancer is a \metaphor{battle}}.

As \mycite{lakoff1980} and others point out, metaphors are not confined
to literary or rhetorical contexts, and are used frequently in
everyday language.  For example, one might say student attendance was
\metaphor{up}, indicating an increased in attendance.
\mycite{lakoff1980} would classify this as an orientational metaphor:
the word \metaphor{up} denotes increased altitude in literal speech,
but is used here as part of a systematic scheme whereby an
orientational vehicle (up, down, in, out, etc) refers to a
non-spatial topic (happy, sad, rich, hot, cold, etc).  Such systematic
schemes are traditionally denoted using small capitals, as in:
\conmet{the mind is a container}; other apposite examples might
include \conmet{communication is transfer}, or \conmet{education is
  acquisition}.  When sensitized to the issue, one tends to see
conceptual metaphors everywhere.  For example, we \metaphor{pass} an
exam (\conmet{life is a journey} or possibly \conmet{challenges are
  obstacles}), and either \metaphor{progress through} college
(\conmet{situations are containers}), or \metaphor{drop out}
(\conmet{down is bad}).

Such quotidian use of metaphor can easily pass unnoticed. However,
\mycite{lakoff1980} argue persuasively that metaphors can and do inform
our conceptual system and influence our actions; they make a strong
case for a serious study of metaphor in a variety of contexts.
\mycite{bereiter2005}, in particular, cautions against the \conmet{mind
  is a container} metaphor, deriding it as ``folk theory of mind'',
and goes on to argue that its uncritical adoption is damaging to
education.

\section{Cognitive metaphor theory}

Cognitive metaphors such as \conmet{the mind is a container} are
rarely used directly in speaking or writing but are influential in
that they function at the level of thought, below language
\parencite{deignan2005}; it is common to say that linguistic metaphors
          {\em realize} conceptual metaphors.  Deignan goes on to give
          five tenets of cognitive metaphor theory:

\begin{enumerate}
\item Metaphors structure thinking
\item Metaphors structure knowledge
\item Metaphor is central to abstract language
\item Metaphor is grounded in physical experience
\item Metaphor is ideological.
\end{enumerate}

These five tenets of metaphor theory may be used to structure thinking
of corpus analysis when considering metaphors in education.

\subsection{Metaphors structure thinking and knowledge}

Corpus linguistics is the study of language as expressed in naturally-occurring
text; the standard methodology (annotation-abstraction-analysis) is
due to \mycite{wallis2001}, although this is not well-suited to
analysis of metaphor in education.  Metaphorical analysis of corpora is discussed
by~\mycite{deignan2005}, in the context of understanding metaphor per
se.  This approach is not especially suitable for the investigation of
a specific metaphor (or set of metaphors), as here, but the
methodology has been adopted by~\mycite{charteris-black2004} and
others, to assess specific areas of language use.  The author takes
corpora from political rhetoric, financial reporting, and religious
texts, and assesses metaphor use in a range of written corpora.  But
note that the texts all have one feature in common: they are written
specifically to convince the reader to think in a particular way, to
adopt a particular stance. The techniques used in such corpus analysis
are suitable for political rhetoric or similar texts, but tend to
focus on metaphor in general rather than specific metaphors such as
the level metaphor or the foundation metaphor.

\subsection{Metaphor is central to abstract language}

In a very heavily-cited work, \mycite{reddy1993} discusses one very
important concept in education, that of communication.  In essence,
his thesis is that communication is frequently discussed using the
\metaphor{conduit} metaphor \parencite{lakoff1980}; he later characterized this
as a distillation of the conceptual metaphors \conmet{ideas are
  objects}, \conmet{linguistic expressions are containers}, and
\conmet{communication is sending}.  Reddy states that stories about
communication (of which education figures prominently) are largely determined by semantic structures and it is
clear that the primary structure he has in mind is (linguistic)
metaphor.

\subsection{Metaphors are grounded in physical experience}

Many metaphors have their origins in physical space; \mycite{lakoff1980}
term these \emph{orientational} metaphors.  Orientations may include
up-down, in-out, central-peripheral, near-far, shallow-deep, and so on; the best
examples include \conmet{happy is up}; \conmet{intimacy is proximity}
and it is clear that the \metaphor{level} and \metaphor{foundation} metaphors are at least
partially spatial.  \metaphor{Ascending the levels} of a course of
study and indeed the foundation metaphor itself would be exemplars of
\conmet{sophisticated is up}.

\subsection{Metaphors are ideological}

Although most authors discussing metaphor interpret the word
\dquote{ideological} in terms of either corporate or national
policy~\citep{perrez2015}, it is worth remembering that ideology can
apply to any system of ideas by any group or community; here the
relevant community would be tertiary teachers and students.  Consider
the \metaphor{level} metaphor as an example.  Whether the implications
of the metaphor constitute an ideology (or indeed whether
considering current mathematical education's level-based philosophy
from an ideological perspective is a coherent or desirable scheme) remains an open
question.


\section{Research methodologies}

A systematic citation analysis of~\mycite{lakoff1980}'s seminal work
revealed few studies of metaphorical language in specific contexts;
the majority of those discussed the ramifications of metaphor in the
medical profession.  \mycite{montgomery1991} discusses
medicine-as-combat and~\mycite{harrington2012} presents cancer-as-war.

This type of research on metaphor in education is sparse, and what
does exist is largely focused on its use by teachers at primary or
secondary level \citep{munby1986,cameron2003}.  \mycite{paechter2004}
is one of the very few works discussing the level, foundation, or
framework metaphor; perhaps this is because their use constitutes an
\dquote{unquestioned norm}, which renders investigation difficult.

Established research methodologies appear to be poorly suited to this
work.  Hermeneutic analysis would seem to be
inappropriate~\mycite{mantzavinos2005} on the grounds that there is no
canonical text.  It might be argued that the Education Act 1989 or the
New Zealand Qualifications Framework constitute revelatory texts but a
close word-by-word reading of these documents would seem to be unlikely to produce
any insight into metaphor as used in the field.

Documentary analysis techniques~\parencite{fitzgerald2012} might be
more promising although they are geared towards historical rather than
social analysis and again there is no canonical document to analyze.

\mycite{harrington2012} presents one of the very few
discipline-specific surveys of metaphor usage in discourse, in this
case medical science.  She presents a careful and insightful study of
metaphor use in (written) discourse about cancer---military and
journey metaphors figure prominently---but her work does not appear to
fall into any recognizable research methodology.  She does analyze
various documents which she believes to be representative or
influential, but makes no attempt at a systematic survey or to trace
any historical drift.  Having said that, her work is extremely
convincing, and very heavily cited, and provided inspiration for the
present work.

In an educational context, \mycite{cameron2003} considers metaphor in
spoken English, using 10- and 11- year old students as informants.
She employs concepts from applied linguistics to infer students'
learning strategies, and to improve teaching methods.  Her data
comprised natural utterances (and a small amount of written material)
and her conclusions centered round detailed analysis of carefully
selected, and mostly very short, fragments.

These two approaches contrast sharply in their epistemological
assumptions about the nature of knowledge: \citeauthor{harrington2012}
is clearly oriented toward propositional knowledge [of patients' and
  doctors' speech]; while Cameron is more focused on knowledge by
acquaintance: she makes little attempt to generalize her findings
beyond the confines of her classroom study.

Both works, however, share a common understanding of ontology: both
writers maintain the existence of an entity, here the linguistic
phenomenon of metaphor, with certain properties.  The assumption is
that it is possible to observe this entity, albeit imperfectly, and
make inferences about its nature and properties.  Admittedly, Cameron
observes human behaviour through the lens of linguistic theory, while
Harrington considers only published research, but both clearly have an
entity in mind which they wish to learn about.

One result of recent informal conversations by the author is that
there is a much wider interpretation of the vehicles \dquote{level},
\dquote{foundation} and \dquote{framework} (when used in educational
policy) than might be expected.  For example, to me the level metaphor
involves ascending a multistory building; but many of my colleagues
view the level metaphor as actually constructing a large structure or
edifice, something that was not part of my thinking.

One possibility might be to interview, say, ten practising mathematics
or statistics lecturers in a semi-structured interview; this might
produce interesting results.  However, one potential pitfall might be
the recruitment of informants, who would need to be chosen carefully:
the interviewees would not be a random sample, but on the other hand
statistical validity is not an issue in studies of this
type~\citep{ribbins2012}.  Also, merely stating the purpose of the
interview might distort their perceptions; but an appropriately
structured system of questioning might be able to mitigate this
deficiency.  Further work would be needed to assess whether this approach would be worthwile.

A number of workers have considered metaphor from an organizational
theory perspective~\citep{cornelissen2012}.  \mycite{amernic2007}, for
example, consider the metaphors in a series of letters written by a
CEO to his shareholders; and \mycite{tourish2012} consider the metaphors
used by disgraced bankers following the 2006 financial crisis.  These
authors study \emph{root metaphors}---that is, metaphors which provide
\dquote{rich summaries of the world and reveal dominant and powerful
  ways of seeing}.  In this thesis, I discuss root metaphors in the
context of statistics education.

Drawing on these ideas, this thesis presents a semi-systematic survey
of language as used in the various phases of a standard undergraduate
statistics course.  

\begin{singlespace}
\begin{quote}
  Metaphor is both important and odd---its importance odd and its
  oddity important---\mycite{goodman1979}
\end{quote}
\end{singlespace}

\section{Metaphors in tertiary education}

If~\citeauthor{lakoff1980} are correct in their view of metaphor being
a cognitive---rather than a linguistic----phenomenon, then it is clear
that metaphor will have an important part to play in education.  This
thesis considers metaphors as used in tertiary education, with a focus
on undergraduate statistics.

Most studies of metaphor in an educational context focus on its use by
teachers~\citep{willox2010} observe that most literature focuses on
metaphor as an \dquote{instructor driven pedagogical tool}); however,
students too too create and use metaphor in many educational contexts
including learning activities such as lectures as well as in
assessment.

In this thesis, I consider metaphor as used in written and spoken
language by instructors and students in the following aspects of
undergraduate education.  The focus is on the case of undergraduate
statistics; most of my teaching is in this area.

The thesis chapters cover the different aspects of an undergraduate
statistics course; they are ordered roughly in chronological order as
experienced by the lecturer.  The aspects covered are as follows:

\begin{itemize}
\item Metaphor in educational planning (level/foundation metaphors and
  possibly the factory metaphor or the acquisition/participation
  metaphor) (chapter~\ref{chapter2});
\item Metaphor educational administration (chapter~\ref{chapter3});
\item Metaphor used in spoken or recorded lectures (chapter~\ref{chapter4})
\item Metaphor in statistics textbooks (chapter~\ref{chapter5})
\item Metaphor in statistics assessment (chapter~\ref{chapter6})
\item Metaphor in course evaluation (chapter~\ref{chapter7}).
\end{itemize}


\section{Conclusions}

Following \citeauthor{lakoff1980}'s seminal publication, many scholars
have written about the power of metaphor to shape and guide thinking.
One way in which we can study the effect of metaphor is via its
linguistic manifestation, which is open to study in both corpora and
spoken English.  Metaphor is known to be influential in thinking about
education (\citeauthor{sfard1998}'s acquisition metaphor; the factory
metaphor), and also more directly in educational policy documents, and
even more directly in classroom practice.  There are a few fields,
medical science in particular, where particular metaphors (martial
metaphor for cancer) have been studied and shown to have powerful and
sometimes harmful effects.  Given the undeniable power of metaphor,
one might expect that metaphorical language to be an important
component of statistical education.  The extent to which this is true
is the topic of this thesis.

\begin{singlespace}
\epigraph{We have no literal language for talking about what
  thoughts do\ldots [there is] no possible way of literally saying
  what has to be said: so that if it is to be said at all, metaphor
  is essential}{\mycite{ortony1975}}

\epigraph{Your brain does not process information, retrieve knowledge
  or store memories}{\mycite{epstein2016}}

\epigraph{The \dquote{dead metaphor} account misses an important
  point: namely, that what is deeply entrenched, hardly noticed, and
  thus effortlessly used is most active in our thought.  The metaphors
  listed above may be highly conventional and effortlessly used, but
  this does not mean that they have lost their vigor in thought and
  that they are dead.  On the contrary, they are \dquote{alive} in the
  most important sense---they govern our thought: they are
  \emph{metaphors we live by}.  One example of this involves our
  comprehension of the mind as a machine.}{\mycite{kovecses2010}}

\epigraph{Just over a year ago, on a visit to one of the world's most
  prestigious research institutes, I challenged researchers there to
  account for intelligent human behaviour without reference to any
  aspect of the [Information Processing] metaphor.  \emph{They
    couldn't do it}, and when I politely raised the issue in
  subsequent email communications, they still had nothing to offer
  months later.  They saw the problem.  They didn’t dismiss the
  challenge as trivial.  But they couldn’t offer an alternative.  In
  other words, the IP metaphor is \dquote{sticky}.  It encumbers our
  thinking with language and ideas that are so powerful we have
  trouble thinking around them\ldots [T]he idea that humans must be
  information processors just because computers are information
  processors is just plain silly, and when, some day, the IP metaphor
  is finally abandoned, it will almost certainly be seen that way by
  historians, just as we now view the hydraulic and mechanical
  metaphors to be silly.}{\mycite{epstein2016}}
\end{singlespace}
 % overview of metaphor
  \begin{savequote}[105mm]
  \begin{singlespace}
    Educators have been talking about \metaphor{foundations} for so long
that it no longer seems metaphorical---but that is when metaphors
become the most dangerous.  All that \dquote{foundation} literally
means in the context of instruction is something taught initially in
order to facilitate future learning.  This may or may not have
anything to do with foundational ideas of the discipline, but the
metaphor disposes people to prejudge this issue.
\qauthor{\mycite{bereiter2005}}
\end{singlespace}
\end{savequote}

\chapter{Metaphor in educational policy}
\label{chapter2} 

\section{Overview}

In this chapter I investigate metaphor in educational planning and
policy, with special reference to undergraduate statistics education.
I discuss metaphor in general, and then show why metaphorical language
is both important and informative in educational policy.  I present a
literature review of research that has been carried out in this area, and
set out a proposal for further work.

\mycite{miller1990} consider metaphor in the context of educational
policy.  They state that metaphors may serve as important clues in
better understanding of the implicit ideological preferences of the
policymakers themselves.

There are three metaphors that appear to be particularly pervasive in
the structuring of undergraduate education: the \metaphor{level}
metaphor, the \metaphor{foundation} metaphor, and the
\metaphor{framework} metaphor.  These metaphors are commonly used in
the context of undergraduate statistics education, and this chapter
gives an overview of these and related metaphors.

\section{The \metaphor{level} metaphor}

The Oxford English Dictionary (henceforth OED) gives a number of
literal senses of the word \dquote{level}, the most germane of which
are \dquote{a horizontal plane} and \dquote{position on a real or
  imaginary scale}.  Much undergraduate mathematics is structured into
\dquote{levels} (see, for example, the New Zealand Qualifications
Framework) that broadly correspond to time spent in further education.
Typically, a first year undergraduate course would be described as
\dquote{level 5}, a second year course as \dquote{level 6}, and so on.
However, it should be noted that the OED does not admit that levels
are discrete, although other dictionaries include \dquote{a floor
  within a multi-storey building} which makes the discretization of
the vehicle explicit.

Note that this metaphor is easily adapted to a constructivist
framework: the students construct the multi-storey building as they
ascend the levels.  However, the metaphor fails in certain key
respects.  Firstly, the content of each level is generally held to be
of equal, uniform, difficulty.  This is questionable at best and, I
would suggest, demoralizing at worst.  Secondly, higher levels are
supposed to be of successively greater difficulty; this is unlikely to
be true if \dquote{foundational} mathematics is studied.  And thirdly,
any cohort is imbued with some form of magical elevation from one
level to the next at the beginning of a school year.
\mycite{paechter2004} points out that the \metaphor{level} metaphor
entails that every layer rests on the one before, observing that such
metaphors are temporal rather than spatial.

\mycite{miller1990} consider such metaphors in the context of national
educational policies and point out that such metaphors reflect a
desire for permanence, stability, and predictability.

In the context of statistics education, one might point to the concept
of statistical independence.  This is considered to be a \dquote{level
  5} concept and indeed the basic definition is readily
understandable: events~$A$ and~$B$ are independent
if~$P\left(A\left|B\right.\right)=P\left(A\right)$.  However, the
concept of statistical independence is notoriously tricky, even for
professional statisticians---with~\mycite{dawid1979} detailing a
number of common fallacies surrounding the distinction between
independence and conditional independence.


\subsection{The International Standard Classification of Education}

Explicit statements on the nature of educational levels appear to be
rare\footnote{The Bologna process~\parencite{keeling2006} discusses
  \dquote{cycles} corresponding broadly to BA, MA, and PhD degrees.}.
However, one of the very few places where the level metaphor is
discussed explicitly is the International Standard Classification of
Education,
ISCED~\parencite{unesco_institute_for_statistics_international_2012}.
This is the \dquote{standard framework used to facilitate
  international comparisons of education systems}.  Item~47 is worth
quoting in full:

\begin{singlespace}
\begin{quote}
\dquote{The notion of \dquote{\emph{levels}} of education is
  represented by an ordered set, grouping education \emph{programmes}
  in relation to \emph{gradations of learning experiences}, as well as
  the knowledge, skills and competencies which each programme is
  \emph{designed} to \emph{impart}.  The ISCED level reflects the
  \emph{degree} of complexity and specialization of the \emph{content}
  of an education programme, from \emph{foundational} to
  complex}---\parencite[item
  47]{unesco_institute_for_statistics_international_2012}
\end{quote}
\end{singlespace}

\noindent
(here, salient metaphorical terms are indicated in italics).  In these
short quotes, ISCED uses the level, foundation, and framework policy
in concert.  All three are examples of the conceptual metaphor
\conmet{ideas are buildings}; but note that the first and second are
also examples of orientational metaphors, in this case \conmet{up is
  good}; and the third is---arguably---an example of \conmet{the mind
  is a machine}; or just possibly \conmet{theories are objects}.
ISCED seems to be aware that the \metaphor{level} metaphor is indeed
only a metaphor.  Item~48 reads:

\begin{singlespace}
\begin{quote}
  \dquote{Levels of education are therefore a construct based on the
    assumption that education programmes can be grouped into an
    ordered series of categories.  These categories represent broad
    steps of educational progression in terms of the complexity of
    educational content.  The more advanced the programme, the higher
    the level of education}---\parencite[item
    47]{unesco_institute_for_statistics_international_2012}
\end{quote}
\end{singlespace}

\noindent
Documents such as the New Zealand Qualifications Framework employ the
level metaphor extensively, presenting tables of properties of study
from level~1 (certificate) through level~10 (PhD).  One might expect
that Bloom's taxonomy or the SOLO taxonomy (themselves examples of
discrete orientational metaphors) would be directly relevant here, but
this does not appear to be the case.

\section{The \metaphor{foundation} metaphor}

For \emph{foundation}, the OED gives \dquote{the solid ground or base
  on which an edifice or other structure is erected} and the word is
often used to refer to mathematics content that is more basic or
fundamental than other material.  In this context the phrase is an
instantiation of the \conmet{theories are buildings} conceptual
metaphor.

The term \dquote{foundation} in the context of statistics education
has two meanings: firstly, it refers to statistical knowledge and
techniques that are frequently assumed knowledge in more advanced
courses; and secondly, it refers to the foundations of the discipline,
usually meaning the relationship between statistical reasoning and the
more fundamental science of probability.

\mycite{bereiter2005} gives a disarming, yet devastating, observation on
the first sense of \metaphor{foundation} (itself rich in metaphor):

\begin{singlespace}
\begin{quote}
  \dquote{But the insidious effect of the foundation metaphor does not
    stop there.  No builder would construct a foundation without
    having a pretty clear idea of the building to be erected upon it;
    only a subcontractor would do that.  Beginning students, having no
    way to foresee the eventual structure of knowledge, are
    therefore cast into the role of subcontractors}---\mycite[page
    336]{bereiter2005}
\end{quote}
\end{singlespace}

\noindent
It is perhaps worth pointing out that, to the professional
mathematician or physicist, those working in the foundations of the
discipline enjoy the highest status in the profession:
\cite[pp469--571]{mcculloch2013} talk of the \dquote{high status} and
\dquote{great prestige} of pure mathematics when compared with
applied; note that there is no equivalent of the
\emph{Apology}~\parencite{hardy1940} for applied mathematics (or
indeed statistics).

\subsection{Foundational statistics}

The term \dquote{foundational statistics} usually refers to study of
the relationship between probability and statistics.
Probability---itself on a shaky and arguably meaningless logical
footing\footnote{The hugely influential treatise
  of~\mycite{definetti1975} famously begins with the provocative
  statement that PROBABILITY DOES NOT EXIST (the intended sense was
  that probability has no objective meaning).  Many subsequent
  authors, notably~\mycite{nau2001}, quote this rather subversive
  assertion with approval, retaining the startling capitalization of
  the original.}---has its roots in pure mathematics, and I consider
metaphor usage in mathematics in Chapters~\ref{chapter4}
and~\ref{chapter5}.

The \metaphor{foundation} metaphor might suggest that foundational
statistics is somehow more fixed, more solid, or more
well-established, than other branches of statistics.  One could
reasonably demand that \dquote{foundations} of any discipline be firm.
How can a study of statistics be built on anything but the most sturdy
of fundaments?

Even a cursory study of foundational issues in statistics reveals two
surprising features: firstly, the large number of mutually exclusive
and inconsistent statistical principles in common
use~\parencite{edwards1984}; and secondly, the deficiencies and
unavoidable contradictions of inferential statistics as practised in
the applied sciences~\parencite{wasserstein2016}.

\begin{singlespace}
\begin{quote}
There is no universal agreement on which principles are \dquote{right}
or which should take precedence over others.  Indeed, the study of
foundations includes much debate and controversy.  \mycite[p1340]{robins2000}
\end{quote}
\end{singlespace}

\noindent
Thus statistical education is structured in such a way that
pedagogical strategies which obscure such difficulties (in the
interests of provision of techniques useful in practice) are needed.
One such strategy is the use of metaphor (see chapter~6).

\section{The \metaphor{framework} metaphor}

For \dquote{framework}, the OED gives \dquote{a structure made of
  parts joined to form a frame; esp. one designed to enclose or
  support; a frame or skeleton}, with senses supporting this use in an
industrial or horticultural context.

In an educational policy, \dquote{framework} usually refers to an
organized set of standards, aims, or learning objectives that loosely
specify the type of material to be learned.  This is frequently used
to support the \metaphor{level} and \metaphor{foundation} metaphors.
However, frameworks are generally held to be rigid and inflexible
structures and the effect of the framework metaphor is not necessarily
beneficial.

\mycite{paechter2004} observes that the framework metaphor, along with
its close relation the \metaphor{scaffolding} metaphor, is an exemplar
of a wider class of structural spatial metaphors.

\section{Metaphors in educational policy}
The value of metaphor has been clear to practising educators for a
very long time; \mycite{cameron2003} observes that metaphors are
peculiarly susceptible to being misinterpreted in a classroom context
but emphasizes the fact that education simply cannot function without
them.

In a wide-ranging review, \mycite{botha2009} considers the epistemic and
ideological freight carried by metaphor in an educational context; yet
she omits entirely any mention of metaphor in educational policy.  One
of the very few scholarly writings to consider metaphor's role in
educational policy per se is that of \mycite{bessant2002}.  Bessant
considers the political rhetoric surrounding an influential period of
educational reform in Australia, focusing on the use of metaphorical
language.  Like \mycite{charteris-black2004} and \mycite{deignan2005},
Bessant considers metaphor as a persuasive device, but emphasizes
metaphor's ability to inform our thinking without us being aware of
its influence.

There are two further metaphors that appear in connection with
education: the {\em factory} metaphor, and the {\em acquisition
  metaphor} of \mycite{sfard1998}.  I discuss each in turn below.

\subsubsection*{The factory metaphor}

By far the best-known educational metaphor is the factory metaphor.
\mycite{claxton2013}, following \mycite{toffler1990}, draws several
paragraphs of analogies between modern schools and production lines:
cohorts become batches; (educational) standards and indeed examination
grading become quality control; and so on.  Mass production of
literate, honest, punctual and dutiful workers was conceived of in
exactly the same way as mass production of anything else.  Although
\citeauthor{claxton2013} did not actually use the term
\dquote{conceptual metaphor} here, he emphasized elsewhere the power
of metaphor to guide and sculpt thinking.

\subsubsection*{Participation vs acquisition}

No study of metaphor in educational theory would be complete without
mentioning the work of \mycite{sfard1998}, who points out that
\dquote{human learning [has always been] conceived of as an
  acquisition of something}.  She goes on to develop this acquisition
metaphor, observing its ubiquity in educational discourse, and its
implicit comparison of education with accumulation of material wealth.
\citeauthor{sfard1998} then posits a new \metaphor{participation}
metaphor, on the grounds that education is something one {\em does},
rather than {\em gets}.

This dichotomy has proved fruitful and persistent: \mycite{wegner2015},
for example, point out that the very existence of tuition fees and
credit points highlights the acquisition metaphor and actively
discourages the participation metaphor.  They go on to observe that
the acquisition metaphor is predicated on the assumption that
knowledge can be seen as an entity; the appropriate conceptual
metaphors are \conmet{ideas are objects} and \conmet{the mind is a
  container}.

Note too that the acquisition metaphor is neutral with respect to
constructivism: learners may either receive knowledge entities or
actively construct them.

Expressions like \dquote{knowledge \metaphor{transfer}},
\dquote{intellectual \metaphor{property}} or
\dquote{\metaphor{grasping} ideas} show how deeply engrained this
metaphor is in western language.  The acquisition metaphor includes
both transmissive views (the assumption that knowledge can be passed
by transmission from one person to the other) and constructivist views
(knowledge is constructed individually by each person), because both
conceptualize knowledge as an entity.

\mycite{sfard1998} considers these issues and, using the central
thesis of \mycite{lakoff1980}, applies them in the context of
educational policy.  She leaves us in no doubt that metaphors are
important and influential: \dquote{Different metaphors lead to
  different ways of thinking and to different
  activities}~\parencite[page 5]{sfard1998}.

\section{Conclusions}

Metaphor thus appears to be threaded through educational policy, and
exerts a powerful yet hidden effect on our thought.  Our unthinking
use of the \metaphor{level} metaphor, for example, normalizes the
notion that mathematics and indeed statistics has discrete units of
ascending difficulty.

%% following three epigraphs refer to the same text, with increasing
%% tersification.


%\epigraph{The educational discourse in England has thus become so
%  dominated by metaphors of height-privileged hierarchical space that
%  almost everything now operates in relation to it, including
%  teachers' and students' views of themselves and each other, the
%  operation of schools (targeted mentoring to enable students to reach
%  the benchmark) and the relationship between parents and teachers
%  (the last time I went to a parents' consultation evening, the first
%  thing my eight-year-old's teacher said about him was that he was in
%  the top groups for mathematics and English, as if this was the most
%  important thing to report about his education).  This is a clear
%  example of the way spatial metaphors of schooling can capture us,
%  almost unthinking, in particular, pernicious,
%  discourses}{\mycite{paechter2004}}

%\epigraph{The educational discourse in England has thus become so
%  dominated by metaphors of height-privileged hierarchical space that
%  almost everything now operates in relation to it, including
%  teachers' and students' views of themselves and each other, the
%  operation of schools (targeted mentoring to enable students to reach
%  the benchmark) and the relationship between parents and
%  teachers\ldots spatial metaphors of schooling can capture us, almost
%  unthinking, in particular, pernicious,
%  discourses}{\mycite{paechter2004}}

\begin{singlespace}
\epigraph{The educational discourse in England has thus become so
  dominated by metaphors of height-privileged hierarchical space that
  almost everything now operates in relation to it, including
  teachers' and students' views of themselves and each other\ldots
  spatial metaphors of schooling can capture us, almost unthinking, in
  particular, pernicious, discourses}{\mycite{paechter2004}}
\end{singlespace}
  % planning: level & foundation metaphors
  \begin{singlespace}
\begin{savequote}[105mm]
The framework [for higher education qualifications] should be regarded
as a framework, not a straitjacket\qauthor{Quality Assurance Agency
  for Higher Education, \citeyear{fheq2008}, page 3}

[W]hen there are explicit culturally sanctioned warnings not to do
something, you can be sure that people are doing it.  Otherwise there
would be no point to the warnings.\qauthor{\mycite[page
    164]{lakoff2000}}
\end{savequote}
\end{singlespace}

\chapter{Metaphor in educational administration}
\label{chapter3}

\begin{singlespace}
\setlength{\epigraphwidth}{.7\textwidth} % default is .4 
\epigraph{Metaphors, and their frequency of use, may serve as
  important clues not only in better understanding the stated intent
  of the policy but also the implicit ideological preferences of the
  policymakers themselves.  In our view, the use of metaphorical
  expressions in major policy statements reflects a largely
  unconscious process whereby implicit beliefs, attitudes, and
  ideological presuppositions concerning the desirability or utility
  of a course of action are made explicit}{\mycite[page 68]{miller1990}}
\end{singlespace}

\section{Overview}
\label{course_descriptor_discussion}
In education a \emph{course descriptor} is a terse, self-contained
specification for a unit of study; a \emph{syllabus} lists the
specific course requirements a student must complete.  The term
\emph{curriculum} is usually reserved for the entirety of student
experience while attending the institution.

In this chapter I will consider metaphor in course descriptors, using
first year statistics course from AUT and the University of Cambridge
as examples.

\subsection{The Metaphor Identification Procedure}

In this thesis, I will analyse language (text and speech) using the
metaphor identification protocol (MIP) of~\mycite{pragglejaz2007}.
This is a formal procedure in which metaphor may be identified;
\dquote{\citeauthor{pragglejaz2007}} is the name of a group of
scholars.

Slightly paraphrased, the MIP is as follows:
\begin{enumerate}
\item Establish a general understanding of the text's meaning
\item Determine appropriate lexical units for analysis
\item For each lexical unit, establish whether the contextual meaning
  is the same as the \emph{basic meaning}, which is effectively the
  meaning of the lexical unit when used in isolation
  \item If the basic meaning differs from the contextual meaning, mark
    the unit as metaphorical.
\end{enumerate}

The term \dquote{metaphor} is not used in its literary or poetic
sense, and the procedure identifies many metaphors that would not be
recognised as such by ordinary listeners or readers without prompting.
Also, the MIP does not address the issue of \emph{intent:} the speaker
or writer's intentions are not part of the procedure.  This procedure,
while somewhat subjective, has been used with some success on a
variety of spoken and written sources.

\section{Introduction}

The syllabus and course descriptor are an ideal place to investigate
metaphor usage in education: they are terse, short documents, with an
intended audience comprising both educators and students.  Course
descriptors are official documents, representing a \dquote{legal
  contract of sorts between academy and student}~\parencite{luke2013}

\mycite{obrien2009} consider syllabus from a learning perspective, with
a strong emphasis on students' taking responsibility for their own
learning.  They observe that a syllabus can serve a wide variety of
functions that will support, engage, and challenge students; and it
can establish an early point of contact and connection between student
and instructor.

\mycite{rubin1985}, however, takes a rather pessimistic view of the
institutional course syllabus, mentioning \dquote{badly thought-out
  efforts}.  She classifies syllabuses into \emph{listers} which
emphasize reading lists and topics covered (rather than selection
principles); and \emph{scolders} which give little more than lengthy
sets of instructions detailing what will happen if a student commits
any of a list of minor misdemeanours such as handing work in after the
deadline.

In the field of undergraduate statistics education, many syllabuses
appear to be somewhat unhelpful.  Take The University of Sydney's
\paper{STAT5002}, for example.  This promises to introduce students to
\dquote{basic statistical concepts} and develop
\dquote{computer-oriented estimation procedures}.

Observe that in the absence of any contextual information, these
phrases are highly ambiguous.  \dquote{Basic statistical concepts},
for example, might be the use of histograms, or Fisher sufficiency;
\dquote{computer-oriented estimation procedures} might be using
\texttt{t.test()} in an R session, or writing recursive Archimedean
copul\ae.  It is only when accounting for the fact that
\paper{STAT5002} is a first-year service course that one would be able
to say what sense these phrases are being used; but if this is the
case, what is the point of a syllabus at all?

\section{Auckland University of Technology}

Auckland University of Technology (AUT) is the newest of New Zealand's
universities, having attained university status in 2000.  AUT has the
lowest QS ranking of the six and its promotional literature emphasizes
the practical nature of the study and the employability of its
graduates.

AUT's \paper{STAT500} paper is a first-year mandatory course in
introductory statistics, intended to ensure that second-year science
subjects have appropriate statistical underpinning.  The course
descriptor has several sections, the most salient of which is the
informal course description: 

\begin{singlespace}
\begin{quote}
  An introduction to applied statistics.  Provides techniques for
  describing and summarizing a data set.  Delivers an understanding of
  how to use a sample to infer the properties of the population it was
  taken from.  Students in this course also learn how to use
  statistical software to undertake descriptive analysis as well as
  perform statistical inference.
\end{quote}
\end{singlespace}

The metaphors in this course description may be identified by
following the metaphor identification protocol (MIP)
of~\mycite{pragglejaz2007}.  There are several groups of words that
are used possibly metaphorically:

\begin{description}
\item[introduction] The OED actually gives this sense of the word
  (\dquote{bringing into use or practice}) as the first sense.
  Conclusion: not metaphorical (or, at best, lexicalized metaphor)
\item[applied] the OED gives \dquote{put to practical use}; not
  metaphorical
\item[provides; delivers] Clearly metaphorical.  The notion that a
  course \metaphor{provides} anything is a clear example of the
  conduit metaphor, \conmet{communication is
    transfer}~\citep{reddy1993}; this is supported by
  \metaphor{delivers} as used in the next sentence.  Together these
  suggest that the student is a passive recipient of information,
  specifically information which is delivered by the institution.  The
  form of words specifically excludes the constructivist approach.

  Authors such as~\mycite{krippendorff1993} consider the transmission
  metaphor of lecturing, stating that students becoming
  \dquote{passive and uninformed receivers or consumers} is a
  side-effect of the conduit metaphor.  \mycite{yoon2011} concur,
  associating the transmission model with a social norm of student
  passivity in the context of undergraduate mathematics lectures.
\item[use; infer; population] metaphorical but part of the
  disciplinary jargon; see Chapter~\ref{chapter7}
\item[in (this course)] The use of the preposition \dquote{in}
  suggests that the cognitive metaphor \conmet{courses are containers}
  is being used here.
\end{description}

It is clear that this particular course descriptor is rich in
metaphor, which reveals underlying assumptions and attitudes to
education.

\section{The University of Cambridge}

The University of Cambridge is consistently ranked as one of the
world's best universities and some measure of its status is indicated
by the fact that its alumni include 95 Nobel laureates.  Undergraduate
mathematics at Cambridge includes a substantial amount of compulsory
statistics education.  The course descriptor for the first year
statistics course reads, in part:

\begin{singlespace}
\begin{quote}
The course introduces the basic ideas of probability and should be
accessible to students who have no previous experience of probability
or statistics.  While developing the underlying theory, the course
should strengthen students' general mathematical background and
manipulative skills by its use of the axiomatic approach.  There are
links with other courses [a list is given].  Students should be left
with a sense of the power of mathematics in relation to a variety of
application areas.  After a discussion of basic concepts [list] the
course studies [list] \ldots Through its treatment of discrete and
continuous random variables, the course lays the foundation for the
later study of statistical inference.
\end{quote}
\end{singlespace}

\noindent
This text is rich in metaphor.   In the list below, I discuss the
metaphorical language used, with the exception of the
\metaphor{agency} metaphor, which is discussed at the end.

\begin{description}
\item[The course \metaphor{introduces}]\qquad (agency metaphor);
  also an example of \conmet{ideas are objects}.  Note the active
  voice here, contrasted with the passive voice (\dquote{the course
    introduces}) used by AUT.
\item[the \metaphor{basic} ideas of probability]\qquad the OED
  gives \dquote{fundamental, essential}.  Note the absence of the word
  \dquote{foundation} here: the authors presumably wished to avoid
  confusion with foundational statistics, a separate academic
  discipline (not one typically taught at undergraduate level).
\item[and should be \metaphor{accessible}]\qquad clearly an
  orientational metaphor.  The basic meaning of \dquote{accessible}
  given by the OED is \dquote{readily approached or reached}; however,
  the contextual meaning is that the course can be understood by a
  student with no specialized knowledge.
\item[to students who \metaphor{have} no previous
  experience]\qquad\conmet{abilities are entities}; \conmet{knowledge
  is a possession}.  The word \emph{have} is being used
  metaphorically: familiarity with mathematics is expressed using the
  language of ownership.
\item[of probability or statistics]\qquad Both words used
  metonymically for \dquote{formal courses of study covering
    probability or statistics}.
\item[while \metaphor{developing}]\qquad The OED gives \dquote{to
  bring out what is implicitly contained}.  But note the inherent
  ambiguity in the contextual meaning: is it the theory that is being
  developed, or the student's understanding of it?  If the former,
  this is metaphorical usage: there is no ready sense in which
  underlying theory is implicitly contained in anything.  If the
  latter, the word is being used literally (although then one would
  arguably be using the conceptual metaphor \conmet{the mind is a
    container}).
\item[the \metaphor{underlying} theory]\qquad Underlying:
  \dquote{lying under or beneath}.  Possibly an example of
  \conmet{sophisticated is up} or just possibly \conmet{theories are
    buildings}
\item[the course should]\qquad (agency metaphor)
\item[\metaphor{strengthen}]\qquad\conmet{theories are constructed
  objects}; in particular, \mycite{kovecses2010} offers
  \conmet{abstract stability is physical strength}.
\item[\metaphor{students'}]\qquad The apostrophe would indicate
  \conmet{knowledge is a possession} again.  Observe its placing: the
  possessor is plural, suggesting that the students as a group (in
  contrast to individual students) somehow possess the knowledge in
  question.
\item[mathematical \metaphor{background}]\qquad An orientational
  metaphor; the course is being metaphorically identified with an
  object, to which previous mathematical learning is cast as a
  \dquote{background}. \mycite{lakoff1980} observe that the use of
  such terms imputes two features on the course: firstly, the course
  is rendered as an object; and secondly, it imposes an
  anthropomorphic orientation on it.  In this case the course
  \emph{per se} is not only personified but given an orientation
  facing the speaker (student?).
\item[and \metaphor{manipulative} skills]\qquad\conmet{ideas are
  objects}.  The OED gives \dquote{reposition or reshape [a physical
    object] manually} as the literal sense---and offers
  \dquote{handle} as a synonym---before moving on to metaphorical
  senses in which mental or logical operations are emphasised.

  In mathematical education, one so often encounters the
  word \dquote{manipulate} (or various declensions thereof) that it is
  easy to forget that the word is being used metaphorically.
  \mycite{kovecses2010} lists \conmet{control is holding something in
    the hand}.  Note that the word \dquote{skill} is being used
  \emph{literally} here.  Skill is not necessarily adroit physical
  manipulation of objects---one can be a skilled orator or investment
  banker (or indeed educator).
\item[\metaphor{by its use of}]\qquad(agency metaphor)
\item[the axiomatic \metaphor{approach}.]\qquad Clearly an orientational
  metaphor, but does not appear to be systematic.
\item[There are \metaphor{links} with other
  courses\ldots]\qquad\conmet{courses are objects}, or possibly
  \conmet{theories are objects}.  In this case \metaphor{link} is
  understood in its sense of a component of a chain; a rare example of
  a non-hierarchical vehicle used to describe relationships between
  mathematical ideas.
\item[Students should be \metaphor{left with}]\qquad\conmet{the mind
  is a container}.  This phrase suggests very explicitly that the
  student is a passive receptacle of information.  Also \conmet{a
    course is a journey}: in this case the student is visualized as
  being stationary while the course moves past him or her; hardly
  conducive to a constructionist attitude.
\item[a \metaphor{sense} of]\qquad A lexicalized metaphor.  Here,
  \metaphor{sense} is being used in sense~17 of the OED:
  \dquote{Mental apprehension, appreciation}, or possibly sense~13:
  \dquote{more or less vague perception or impression}.
\item[the \metaphor{power} of mathematics]\qquad In modern usage,
  largely lexicalized.  Possibly \conmet{ideas are objects} or
  \conmet{theories are machines}.  The OED explicitly lists mental
  strength and effectiveness in the primary sense for \dquote{power}
  but the metaphorical interpretation is one of personification.
\item[in relation to a variety of \metaphor{application}]\qquad
  Lexicalized metaphor.  The sense intended is that the ideas
  presented in course are useful in other (perhaps non-mathematical)
  disciplines such as medicine or physics.
\item[areas.]\qquad clearly metaphorical: \conmet{theories are
  containers}.
\item[After a \metaphor{discussion}]\qquad Not quite metaphorical but
  not literal either.  This is arguably a personification metaphor
  (the course does not \metaphor{discuss} anything);
  \mycite{knowles2006} would classify it as a nominalization metaphor,
  observing that nominalization generally tends to de-emphasize human
  agency, while \mycite{miller1990} observe that personification
  metaphors contribute towards a sense of real individuals with real
  concerns.
\item[of \metaphor{basic} concepts]\qquad as above
\item[\metaphor{the course studies}\ldots]\qquad (agency metaphor)
\item[Through its treatment]\qquad Personification again, this time in
  the passive voice.
\item[of discrete and continuous \metaphor{random variables}]\qquad
  The phrase \dquote{random variables} is another lexicalized
  metaphor, occurring frequently in learning resources and research
  literature alike.  It is discussed at length in
  chapters~\ref{chapter3} and~\ref{chapter4}.
\item[The course lays the foundation]\qquad (agency metaphor); also
  \conmet{theories are buildings}, as discussed in~\ref{chapter5}.
\item[for the later study of statistical inference]\qquad Note the
  implicit assumption the the course in question is merely a
  pre-requisite for more advanced study.
\end{description}

\noindent One notable conceptual metaphor in the above course
description is the \emph{agency} metaphor: the course itself is
ascribed agency.  \mycite{tourish2012} consider agency metaphors as
language that portrays events as volitional actions that reflect
internal mental states, and this seems appropriate in the present
context.  \mycite{lakoff1980} might categorize this such agency
metaphors as \conmet{controlled for controller}, whereas
\mycite{knowles2006} suggest that ascribing agency directly to a
controlled item is a \metaphor{personification} metaphor.


%Learning outcomes By the end of
%this course, you should:
%\begin{itemize}
%  \item understand the basic concepts of
%probability theory, including independence, conditional probability,
%Bayes' formula, expectation, variance and generating functions;
%\item be familiar with the properties of commonly-used distribution
%  functions for discrete and continuous random variables;
%\item understand and be able to apply the central limit theorem.
%\item be able to apply the above theory to ``real world'' problems,
%  including random walks and branching processes.
%\end{itemize}


%Statistics is the study of what can be learnt from data.  We regard
%our data as realisations of random variables, and consider models for
%the (joint) distribution of these random variables.  In this course,
%we focus entirely on parametric models, where the class of
%distributions considered can be indexed by a finite-dimensional
%parameter.  As a simple example, the family of normal distributions
%can be indexed by a two-dimensional parameter, representing the mean
%and variance.  Nonparametric models are treated in more advanced
%courses.  Our aim is to make inference about the unknown parameter by,
%for example, providing a point estimate, a confidence interval or
%conducting a hypothesis test.  Building on Part IA Probability, this
%course will present basic techniques of inference, together with their
%theoretical justification.  The final chapter will cover the
%ubiquitous linear model, with its elegant theory of orthogonal
%projection and application of results from linear algebra.  The most
%appropriate book for the course is Statistical inference by Casella
%and Berger (Duxbury, 2001).
%
%
%Learning outcomes
%By the end of this course, you should:
%\begin{itemize}
%\item understand the basic concepts involved in point estimation, the
%  construction of confidence intervals and Bayesian inference;
%\item understand and be able to apply the ideas of hypothesis testing,
%  including the Neyman-Pearson lemma, and generalised likelihood ratio
%  tests, including applications to goodness of fit tests and
%  contingency tables.
%  \item understand and be able to apply the theory of the linear
%    model, including examples of linear regression and one-way
%    analysis of variance.
%\end{itemize}
%

%\section{UNC, Chapel Hill}
%
%University of North Carolina, STOR 555: \dquote{Functions of random
%  samples and their probability distributions, introductory theory of
%  point and interval estimation and hypothesis testing, elementary
%  decision theory}.  Just a list!


\section{Graduate outcomes}

State support for tertiary education is often justified by the belief
that graduates have superior cognitive abilities to non-graduates, and
such superiority will result in a more effective workforce.

Such beliefs are often embodied in terms of institutionalized
\emph{graduate outcomes}, brief idealized statements of the qualities
that successful graduates are expected to possess.

Graduate outcomes, in general, make heavy use of \conmet{abilities are
  possessions}, also reinforcing the \conmet{education is acquisition}
metaphor of \mycite{sfard1998}.  Australia's Higher Education Council,
for example, explicitly state that graduate outcomes are \dquote{the
  personal attributes and values which should be \metaphor{acquired}
  by all graduates\ldots} (my emphasis).  Such wording echoes course
descriptors (chapter~\ref{chapter2}) in their promotion of the
acquisition metaphor at the expense of the participation metaphor.

\section{Conclusions}

Metaphor occurs frequently in educational administration, with
metaphorical language being a pervasive component of the course
descriptors chosen for detailed analysis.  Pre-eminent among the
metaphors used in course descriptors is the agency metaphor: a course
is held to possess agency and as such is personified.  Such metaphors
help to create an impression of \dquote{a sense of real individuals
  with real concerns}.

Metaphor occurs in other parts of educational planning, notably
institutional \emph{graduate outcomes}.  The metaphorical language
appears to promote the acquisition metaphor at the expense of the
participation metaphor.


\begin{singlespace}
\epigraph{Metaphors are important both as indicators of the ways we
  think and as rallying-cries for particular
  worldviews}{\mycite[page 450]{paechter2004}}
\end{singlespace}
  % graduate outcomes
  
\begin{singlespace}
\begin{savequote}[105mm]
Metaphor is used repeatedly [in undergraduate lectures]\ldots but
there are few elaborated or developed metaphors; those there are tend
to be short, unconnected with later metaphors and used primarily to
serve local, rather than global purposes.  \qauthor{\mycite[page 428]{low2008}}
\end{savequote}
\end{singlespace}

\chapter{Metaphor in spoken undergraduate statistics lectures}
\label{chapter4}

\section{Chapter overview}

In spoken academic discourse, deliberate metaphor seems to be a
powerful tool.  Published research on spoken metaphor use in education
appears to be focused on its use by teachers at primary or secondary
level \parencite{munby1986,cameron2003}, with an emphasis on science
education.  \mycite{low2008} is one of the very few publications to
consider metaphor use in university lectures, although attention is
confined to humanities subjects.

As far as academic discourse is concerned, it is widely accepted that
metaphor is a \dquote{basic epistemological, discourse-organizational,
  and pedagogical device}~\parencite{beger2015}.  As such, one might
expect metaphor to be part of spoken education at an undergraduate
level.  This chapter will consider the extent to which metaphor is
part of the most widely recognized aspect of spoken language in
undergraduate statistics education: the lecture.

\section{Introduction}

The traditional spoken lecture is a pedagogical genre that has been
much maligned as a learning tool~\parencite{friesen2011}; authors such
as~\mycite{king1993} decry lectures as an antiquated \dquote{sage on the
  stage} and urge their replacement with a constructivist
\dquote{guide on the side}. 

One of the most cogent and fierce critics of the traditional lecture
is~\mycite{laurillard1993}: \dquote{Lectures are profoundly defective,
  inefficient, and outmoded}.  They are, she asserts, \dquote{a very
  unreliable way of transferring the lecturer’s knowledge to the
  students}.  Perhaps this is true, but observe the casual use of
\conmet{communication is transfer} metaphor and the \conmet{minds are
  containers}; also note the implicit use of \citeauthor{sfard1998}'s
\conmet{education is acquisition} metaphor.

However, lectures also have their champions: \mycite{burgan2006}, for
example, lauds the \dquote{public display of daring and dazzling
  intellectual expertise} that only a live lecture can provide.
Students too defend lectures, specifically citing the
\dquote{efficiency} of lecturing, but note here too the unquestioned
use of the transmission metaphor: \conmet{learning is acquisition}.

\mycite{yoon2011} report that students overwhelmingly defend the
transmission mode of lecturing, while simultaneously acknowledging
that lectures did little to contribute towards understanding.
Students, in interviews, emphasized the efficiency of this mode of
teaching and noted the practical necessity for the lecturer to get
through the allotted lecture content.

%\dquote{However, students unanimously defended the transmission mode
%  of lecturing, even while acknowledging that it did little to
%  engender understanding during the lecture.  They argued that the
%  practices involved in the transmission mode of lecturing were
%  efficient, practical and necessary for the lecturer to get through
%  the allotted content for the lecture}~\parencite{yoon2011}

Nevertheless, lectures are an important feature of undergraduate
education, with the traditional lecture comprising just over half a
student's contact hours in a typical statistics course.
\mycite{yoon2011} attribute the intransigence of lectures to a
combination of academic inertia and students' familiarity with the
format.

What role does \emph{metaphor} play in this problematic learning
environment, peculiar as it is to higher education?  In this section,
I will consider spoken lectures and their use of metaphor from a
pedagogical perspective.

\section{Initialization: call for quiet}

As a lecture is a performance, there are certain normative standards
that are necessary for the process to function as intended.  One of
these is the maintenance of silence among the audience so the lecturer
can hold the floor (student questions are dealt with later in this
chapter).

Many lecturers signal the start of the lecture with a stylized speech
act\footnote{A \emph{speech act} is an utterance considered as an
  action; the canonical example is \dquote{I pronounce you man and
    wife} and \dquote{I name this ship\ldots}.  In this case, the
  start signal is simultaneously a call for quiet, a statement that
  the lecture has started, and the actual beginning of the lecture.
  \mycite{searle1969} would classify this utterance as an
  \emph{assertive}, a \emph{directive}, a \emph{commissive}, and a
  \emph{declaration}.} ranging from a simple \dquote{good morning} to
more sophisticated rituals which may include non-verbal components
such as dimming the lights.

My own lectures have a mixture of these two things: the system clock
is visible to the students on the display screen and when the second
hand passes the precise start time I say \dquote{right, let's go}.
The utterance is metaphorical; nothing is \dquote{going}.  In this
case the first person plural is inclusive (compare
chapter~\ref{chapter5}, in which \dquote{we} is used
idiosyncratically).  It is interesting to observe that the subjunctive
mood is used: the intention is clearly one of inclusion.

It is difficult to study this aspect of language use.  Lecturers 
rarely allow \dquote{outsiders} to attend their lectures, having
\dquote{entrenched norms} of autonomy and
privacy~\parencite{evans2012}; and when they do, this is likely to
change the atmosphere in the lecture hall.


\section{Lecture content}

Metaphor use in spoken undergraduate lectures has been studied
by~\mycite{low2008}, who observed that metaphorical language was used
in large quantities in social science lectures.  They found that
metaphors occurred in \emph{clusters}: conceptually coherent segments
of speech, rich in metaphor.  \citeauthor{low2008} hypothesized that
metaphor clusters marked the boundary between two distinct themes in
the lecturer's narrative, signalling a major turning point.  Metaphor
also appeared when the lecturer was placed under pressure to think
quickly.  The extent to which these findings from teaching in the
social sciences apply to statistics lectures is investigated in this
chapter.

One might expect that, given the pervasiveness of metaphor in
lectures, that figures of speech such as simile would also be common.
However, \mycite{low2010_wot_no_simile} found a \dquote{virtual absence}
of simile in a large corpora of academic English, which included
undergraduate lectures.

In the following, I will discuss a number of metaphors used in my own
videotaped lectures, using the protocol developed
by~\mycite{bergsten2007} for undergraduate pure mathematics courses.
\citeauthor{bergsten2007} split lectures into fragments of a few words
and analysed the fragments individually, focusing on the relation
between the spoken and written content and observing other features of
the lecture such as student questions and the lecturer's gestures.

The source material used here is taken verbatim from a lecture in
which I introduced the Poisson approximation to the binomial
distribution.  This particular lecture was chosen because the limiting
process discussed is an exemplar of the basic metaphor of
infinity~\parencite{lakoff2000}.  The lectures were recorded two years
ago.  The sentence fragments are those containing metaphorical
language, as determined by the~\citeauthor{pragglejaz2007} metaphor
identification protocol; the intervening utterances contained no
metaphor.

\begin{description}
\item[\metaphor{We} have been talking quite a lot about the Binomial
  distribution\ldots]{An example of a metaphorical \metaphor{we}.  The
  audience is almost totally silent; the \metaphor{we} is actually
  \metaphor{I}.  \mycite{pimm1984} discusses the use of the first
  person plural in this context\footnote{\citeauthor{pimm1984} also
    writes about this issue in educational contexts using written
    English; I draw on his work in Chapters~\ref{chapter5}
    and~\ref{chapter6}.}, pointing out that the \dquote{educational
    we} often effectively excludes the speaker.  \citeauthor{pimm1984}
  expresses bafflement as to exactly which community \metaphor{we}
  indicates, and conjectures that it induces (either deliberately or
  inadvertently) \dquote{passive acquiescence} in the
  student\footnote{This interpretation has been cited in a small
    number (A \emph{Web of Science} cited reference search gives~10
    citations at the time of writing) of published sources.  They
    uniformly refer to \mycite{pimm1984} only in passing; and none of
    them offers any conflicting viewpoints.}.} 
\item[\squareb{the Poisson} is one of a \metaphor{family} of
  distributions]{A lexicalized metaphor, one that is standard
  terminology in statistics.  In undergraduate statistics,
  \dquote{family} is usually reserved to describe a set of
  distributions indexed by one or more (possibly real) parameters.
  The literal meaning of \dquote{family} is sociological: a group
  consisting of one or two parents and one or more dependent children
  living together.  However, one striking misalignment of this
  metaphor is that the vehicle is a set (of humans)that is not only
  \emph{discrete} and \emph{finite}, but also has a very small
  membership, typically in the range 2-5. Note also the culturally
  specific nature of this metaphor.  Contrast the topic, which is not
  only continuous but generally infinite.  There are other
  differences: familiarity and ready identification are salient
  features of the vehicle, yet in the topic, complicated and
  unreliable mathematical inference is needed.  Such differences are
  the essence of pedagogical metaphor, as the topic is rendered
  comprehensible due to familiarity with the vehicle.}
\item[Bernoulli trials with a probability of
  \metaphor{success}\ldots]{Standard statistical terminology is to
  refer to the support of any random variable with two outcomes as
  \dquote{success} and \dquote{failure}.  However, there is no value
  judgment inherent in these words and one finds (in studies of
  family sex balancing, for example), that a birth being male is a
  \dquote{success} and female a \dquote{failure}.  Mathematically, the
  two terms are equivalent as they are symmetric with respect
  to~$p\longleftrightarrow 1-p$.

  The terminology is arguably metaphorical: the topic (support of the
  random variable) is described using the vehicle of an abstract
  experiment which may succeed or fail.  This abstractness is, I would
  argue, a virtue on the grounds that one \emph{wants} to extirpate
  any traces of value judgments from the narrative.}

\item[simply because this $\mathbf{1-p}$ here \metaphor{turns into}
  a~$\mathbf{q}$ there]{The context was that the binomial probability
  mass function~${n\choose r}p^r(1-p)^{n-r}$ was rewritten
  as~${n\choose r}p^rq^{n-r}$, with~$q$ being substituted for~$1-p$.
  This is another example of agency metaphor, but one of a peculiar
  kind: the equation is somehow imbued with the ability to transform
  itself from one form to another.}

\item[a much more \metaphor{symmetric} way of writing it]{Here the
  binomial probability mass function was written in the
  form~${a+b\choose a\, b}p^aq^b$, with~$a$ being the number of
  successes and~$b$ the number of failures (the intent was to
  introduce the Dirichlet distribution).  The words \dquote{symmetry}
  and \dquote{symmetric} are problematic for mathematicians; the words
  originally referred to spatial harmony and, for most
  people---including the lecture audience, I would suggest---symmetry
  is an inherently geometric concept.  Here the contextual meaning is
  that of algebraic symmetry between parts of an equation, a concept
  likely to be new to much of the audience.}
\item[because \metaphor{we have} asymmetry between~$\mathbf{a}$
  and~$\mathbf{b}$]{The context was referring to~${a+b\choose
    a\,b}p^a(1-p)^b$, clearly placing~$a$ and~$b$ on a different
  footing.  This is not really an example of \citeauthor{pimm1984}'s
  \dquote{we}: in this case the asymmetry was undesirable.}
\item[it's a much more pleasing way of \metaphor{handling} this]{There
  are two metaphors here.  Firstly, the use of \dquote{pleasing}: note
  the passive voice.  Pleasing to whom?  The intended sense is that
  the form of the equation is intrinsically appealing, independently
  of any particular viewer.  This is not an uncommon
  viewpoint~\citep{rota2011}.  The intended sense is that the
  community of practice into which the students are being acculturated
  is one which collectively finds that particular expression pleasing.

  The other metaphor is the use of the word \dquote{handling}.  The
  intended sense is that the equation under consideration is a
  physical object, and expressing the equation in different
  mathematical ways corresponds to physical manipulation of the
  object.  This might suggest a new conceptual metaphor:
  \conmet{equations are objects}.}
\item[and I'll \metaphor{return to} this formulation later] This is an
  example of \conmet{an argument is a journey}, here underscoring the
  difficulty of the material.
\item[One disadvantage of the binomial distribution\ldots] Arguably a
  peculiar metaphor, perhaps \conmet{equations are tools}.  The
  intended sense was that the distribution included analytically
  intractable terms such as the factorial function, which made the
  equation hard to deal with.
\item[\ldots is that it has this factorial function \metaphor{in it}]
  \conmet{An equation is a container}
\item[The first thing that should \metaphor{come into your head} is]
  {\bf to verify \squareb{that formula}\ldots} This is an example of
  \conmet{ideas are objects and the mind a container};
  \citeauthor{bereiter2005} would dismiss this as \dquote{folk theory
    of mind}.  However, the form of words used is interesting because
  there is no indication of the \metaphor{conduit}
  metaphor~\parencite{reddy1993}: there is no suggestion that the
  concept of verification originated from the lecturer.  The clear
  import is that the (habit of) verification should spontaneously
  arise, unbidden, in the student's mind: this would be the
  \metaphor{agency} metaphor.
\item[Am I doing what I think I'm doing?\ldots can \metaphor{I} check
  it, can \metaphor{I} verify it\ldots] (the context was an
  exhortation to the audience to check their work continually, to
  identify and rectify algebraic and conceptual errors).  This is an
  interesting use of the metaphorical \dquote{I}.  This is arguably an
  inversion of \citeauthor{pimm1984}'s educational \dquote{we}: here,
  \dquote{I} is clearly intended to indicate what the class should be
  doing.

  \mycite{fauvel1988} would observe that such rhetorical devices are
  Cartesian rather than Euclidean: the audience is being personified
  directly and quotes attributed directly to them.

\item[however this is quite a difficult and unwieldy process]
  \conmet{equations are tools}, in this case undesirable qualities of
  algebraic manipulation.
\item[I will \metaphor{give you} this formula in a different form]
  This is one of a large number of \dquote{give} metaphors used in
  this series of statistics lectures (e.g., \dquote{I can
    \metaphor{give} you an exact answer to that}).  Such phrases are
  direct examples of the \metaphor{conduit} metaphor
  of~\mycite{reddy1993}.  However, note the simultaneous use of the
  \metaphor{acquisition} metaphor of \mycite{sfard1998}.  In this
  case, the contextual meaning is a promise to re-write the formula;
  but the basic meaning is  both acquisitive and transferative.
\item[The factorial function isn't easy \metaphor{to deal
    with}]{(also, later, \dquote{the standard deviation is harder to
    deal with than the mean}).  Arguably a personification metaphor:
  the factorial function is given agency.  \mycite{low2008} assert that
  personification is by far the most common metaphor in humanities
  lectures and it is certainly common in these statistics lectures.}
\item[I'll \metaphor{cover} the first two or three members of the
  series]{In this context, \metaphor{covering} is a very commonly used
  metaphor.  Many authors (\mycite{biggs2011} and~\mycite{vella2007},
  for example) criticize the very notion of \dquote{covering} a topic,
  on the grounds that it obscures any learning objectives;
  \mycite{paechter2004} observes that it is a spatial metaphor.

  Note also that the context also carries the implication that this
  material will be assessed at some point, as the concept being
  covered appears on the learning objectives, which are explicitly
  assessed.}
\item[the probability of success is p=0.5, so it is \metaphor{a fair
    coin}\ldots] A coin toss is a prototypical example of a Bernoulli
  trial: it is indisputably random, the probability of success (heads)
  is known precisely, and successive tosses are demonstrably
  independent.  But to say that a Bernoulli trial \emph{is} a fair
  coin is clearly metaphorical.  The coin metaphor is standard
  terminology in statistics.
\item[I can hear my mathematical colleagues \metaphor{howling in
    outrage}]{This was in the context of a somewhat low-status
  technique involving numerical approximation; also, later,
  \dquote{these disadvantages wouldn't \metaphor{cut much mustard}
    with a mathematician}.

  The thrust of these comments is that there are differing schools of
  thought in mathematics and the approach taken in the lecture
  sacrifices exactness (which is highly prized in some disciplines)
  for computational convenience (which is highly prized in this
  particular course).
  
  These metaphors are used in rapid succession, and qualify as part of
  a metaphor \emph{cluster} in the sense of~\mycite{low2008}.  However,
  this cluster did not mark a \dquote{major turning point} in the
  lecture, unlike the clusters identified by~\mycite{low2008}.}
  
  \item[last time we had the binomial
    distribution~$\mathbf{\operatorname{\mathbf{Bin}}\left({n,p}\right)}$]
    {\bf\ldots I'm going to \metaphor{make~$\mathbf{n}$ get larger} in
      a particular way} This qualifies as a metaphor, in this case the
    basic meaning of \dquote{make} is \dquote{force} but the topic is
    an equation, in this case a probability mass function.

  \item[Last time we discussed~$\mathbf{n}$ \metaphor{getting larger},
    with~$\mathbf{p}$ fixed] In terms of the Basic Metaphor of
    Infinity (\bmi) of~\mycite{lakoff2000} (see
    chapter~\ref{chapter5}), this would be \dquote{potential
      infinity}.  The mathematical statement is here the well-known
    Gaussian approximation to the Binomial, another limiting
    distribution this time arising from the central limit theorem.
    
  \item[I asserted that the limit] {\bf (in scare quotes, I'm not
    defining formal limits) as~$\mathbf{n}$ \metaphor{approaches
      infinity}, of the distribution of~$\mathbf{r}$, the number of
    successes, is normal or Gaussian with mean~$\mathbf{np}$ and
    standard deviation~$\mathbf{\sqrt{npq}}$}.  It is difficult to
    interpret this utterance in terms of the \bmi, yet metaphor is
    definitely used.  The phrase \dquote{approaches infinity} is,
    although standard terminology, metaphoric: nothing actually
    \emph{approaches} anything; and in any event, \dquote{approaches}
    implies \dquote{getting closer (to something)}, which is
    emphatically not occurring.
  \item[Now $\mathbf{n}$ is getting larger and~$\mathbf{p}$ is] {\bf
    fixed\ldots and \metaphor{it looks like this} (draws a Gaussian on
    the whiteboard)} I would suggest that this is a metonym: a random
    variable is identified with its probability density function.
    Also note that the random variable is imbued with a visual
    appearance.
  \item[I'm going to think about the] {\bf two parameters,
    $\mathbf{n}$ and $\mathbf{p}$, and I'm going to consider a
    sequence in which $\mathbf{n}$ gets bigger, and simultaneously
    $\mathbf{p}$ gets smaller in such a way that $\mathbf{np}$ stays
    fixed.} The very essence of the (non-metaphorical) Cauchy
    sequence~\citep{hardy1952}: this shows that metaphor is \emph{not}
    necessary for everyday mathematical teaching.
\end{description}

\subsection{Summary}

The sentence fragments quoted above illustrate the frequency and
ubiquity of metaphor in statistics lectures.  Conceptual metaphors were
frequent, specifically \conmet{equations are tools} which was used
several times.

Metaphor, at least in the source material above, is a key pedagogical
tool in the sense that many of the utterances could not easily be
rephrased nonmetaphorically.  None of the metaphors (with the possible
exception of the highly idiomatic \metaphor{cut much mustard}) are
natural part of language and would not draw attention to themselves.

Such observations are consistent with those of~\mycite{geary2011}:
metaphors are indeed \dquote{hiding in plain sight}.  Their very
unremarkability, even invisibility, combined with their frequency and
power, suggests that we take metaphor very seriously in education.

\section{Students' use of language in lectures}

The lecturer is not the only source of language in lectures: students
also, on occasion, ask questions.  \mycite{marbach2000} consider
students' asking of classroom questions as they progress from
elementary level to college, and conclude that students learn not to
ask questions in class.  At undergraduate level, the lecture
environment has implicit social norms which generally enforce a
passive role during lectures~\citep{yoon2011}.

Questions are generally infrequent, with~\mycite{pearson1991}
reporting an average of only three questions per hour, the majority of
which were on-task but restricted to procedural clarification, such as
due dates on assignments or venues for lectures.  The only questions
in the chosen lectures on the Poisson distribution were clarificatory.
Student questions do not appear to be sufficiently frequent to make
any claims about students' use of metaphor in this context, either
from the literature or my own observations in my lectures.

\section{Conclusions}


Metaphor is certainly a component of spoken undergraduate statistics
lectures, but with only two exceptions (\dquote{howling in outrage}
and the idiomatic \dquote{cut the mustard}), none of them call
attention to the language used, nor are likely to be perceived as
metaphorical by the audience.

A number of metaphors in the corpus analysed appear to relate to
\dquote{domestic} aspects of the lecture such as promises to discuss
certain content, or standard pedagogical constructs such as
\conmet{communication is transfer} (the most apposite example being
\dquote{I will \metaphor{give you} this formula\ldots})

A certain amount of metaphor is unavoidable in any comprehensible
mathematical or statistical discourse.  However, much of mathematics
is arguably metaphorical, as argued by~\mycite{lakoff2000}.
Specifically, the \metaphor{basic metaphor of infinity} (\bmi)
occurred several times in the corpus under study.  A more detailed
discussion of the \bmi is given in Chapter~\ref{chapter5}.




%\conmet{learning is storage} produces [a situation that requires
%  students to] shut up, avoid wiggling, and above all avoid
%interrupting.  p345
%
%quote attributed to
%1, Marshall Gregory, 'If Education Is a Feast, Why Do We Restrict The
%Menu? A critique of pedagogical metaphors, \dquote{College Teaching},
%Vol.35,Part 3, 1987,p.103.
%
%{\bf The storage metaphor is stultifyingly utilitarian and deceives
%  young people into thinking that we are giving them information vital
%  for their survival in the adult world.  p346 }

%\setlength{\epigraphwidth}{.7\textwidth}  % default is .4
%\epigraph{[The dichotomy between expository and argumentative essay
%    styles] is somewhat muddied by the fact that all essay examination
%  writing contains an argumentative element, namely the writer's
%  attempt to persuade the reader/instructor to proffer an acceptable
%  grade in a course.  This intention must be hidden, however\ldots
%  because of the social convention that states that an expository exam
%  essay is to be written as if the reader did not already understand
%  the ideas being presented---as if a prompt were really a
%  question---and that an argumentative essay is to be written as if
%  the reader did not already agree with the thesis---as if a prompt
%  were really part of a Socratic dialogue.  Thus, essay examination
%  writing is doubly false, in that writers must hide their true
%  intention (to pass the course) behind a wall of prose designed to do
%  what has already been done}{\mycite{horowitz1986}}
%
  % spoken statistics lectures
  \begin{singlespace}
\begin{savequote}[105mm]
  How much can we infer about the basic cognitive mechanisms used in
  mathematics from what we find in texts and curricula? A study of
  navigation based on the standard manuals would tell us very little
  indeed about how the task was actually accomplished on the bridge of
  a large ship.  \qauthor{\mycite[page 1185]{madden2001}}
\end{savequote}
\end{singlespace}

\chapter{Metaphor in statistics textbooks}
\label{chapter5}

\section{Overview}

This chapter gives a discussion of metaphor and metaphorical language
as used in statistics textbooks.  The majority of the relevant
literature covers mathematics in general; few articles consider
metaphor in statistics education.  In this chapter I use literature
that analyses metaphor in mathematics textbooks and consider the
extent to which the findings are applicable to statistics education.

The chapter is split into two main parts.  The first part will
consider metaphor in mathematics generally, specifically as
interpreted in the controversial \emph{Where mathematics comes
  from}~\citep{lakoff2000}, henceforth~\wmcf.  I will consider this
book from the perspective of statistics education.

The second part of the chapter concentrates on one often-overlooked
aspect of language frequently used in mathematics textbooks: the
mathematician's \emph{we}.  This usage is considered to be
metaphorical because the referents are not a well-defined group.  I
will consider the educational implications of this language use.

\section{Introduction}

A \emph{textbook} is a standard work for the study of a particular
subject, here statistics; attention will be confined to those used for
undergraduate study.  Mathematics textbooks are a valuable and
often-consulted resource for university
students~\parencite{weinberg2012}.  One might expect metaphorical
language to be widely and effectively deployed.

Three textbooks were chosen for detailed study:

\begin{itemize}
\item \mycite{crawley2015}
\item \mycite{feller1968}
\item \mycite{casella2001}
\end{itemize}

These books span a range of
sophistication---\citeauthor{crawley2015}

\footnote{It is standard practice to refer to textbooks by the
  author(s) name.} is practitioner-oriented, heavy on computational
examples and light on mathematical detail.  \citeauthor{casella2001}
is classified as high-end undergraduate or mid-range postgraduate
study material; while \citeauthor{feller1968} is a classic work,
emphasizing rigour.  These are books that I use in my own teaching and
broadly correspond to first, second and third year courses.

The \mycite{pragglejaz2007} protocol, applied to these three texts,
revealed that metaphor (in the sense of \mycite{lakoff1980}) was rare
to nonexistent.  This is perhaps not surprising in such a mathematical
context where accuracy is more highly valued than clarity or even
educational value.

There was, however, one non-literal usage of language that occurred
frequently throughout all three books: the mathematician's \emph{we},
which is discussed in section~\ref{we_start}.

\section{Random variables and metaphor}

The notion of \emph{random variable} is a central concept in
statistics.  The formal definition of a random variable is as follows.

\begin{quote}
Suppose we have a probability
space~$\left({\Omega,\mathcal{F},P}\right)$.  Then if~$E$ is some set
and~$X\colon\Omega\longrightarrow E$ is measurable function
from~$\Omega$ to~$X$, we say that~$X$ is a {\em random variable}.
\end{quote}

\noindent
Note the abstract and unhelpful nature of such a rigorous definition;
mathematically, the difficulty lies in ensuring consistent behaviour
when~$E$ is uncountably infinite (one prominent example would
be~$\mathbb{R}$, the real numbers; one would hope that such
definitions do not let one down in such a practically important case).

The influential {\em Khan Academy}~\parencite{khan2016} is one of many
introductions to inferential statistics that discusses random
variables from a more practical perspective.  While declining to offer
a formal definition, \citeauthor{khan2016} does give several examples,
the canonical one being

\[
X = \begin{cases}
  0 &\text{if coin lands tails}\\
  1 &\text{otherwise}
\end{cases}
\]

\noindent\citeauthor{khan2016} goes on to state that, 
together with the specification that~$p(X=0) = p(X=1)=1/2$ fully
characterizes~$X$.

Undergraduate statistics textbooks typically offer a level of rigour
between these two extremes.  To what extent do linguistic or cognitive
metaphors enter in to such discussions?  \mycite{lakoff2000} would
suggest that metaphor plays a large part in all of mathematics, and
indeed claim that \emph{all} mathematical reasoning is inherently
metaphorical.

Of all the metaphorical mathematics presented in \wmcf, by far the
most relevant is the \metaphor{basic metaphor of infinity}, discussed
in the next section.

\section{The basic metaphor of infinity}

% macro '\wmcf' is defined in main.tex

Metaphorical language and reasoning is common in mathematics and
mathematics education~\parencite{pimm_metaphor_1981}.  However, the
study of metaphor in mathematics was kick-started by publication of
the controversial {\em Where mathematics comes
  from}~\parencite{lakoff2000} which set out the authors' contention
that {\em all} mathematical reasoning is metaphorical.  The authors
also make a case for mathematics {\em per se} being a human construct.

Statistics, like many branches of mathematics, often uses the concept
of \emph{infinity}.  Here I draw on the ideas of \mycite{lakoff2000},
in a controversial work often referred to as \dquote{\wmcf} (being the
initials of the book title, \emph{Where Mathematics Comes From}).
\wmcf suggests that all mathematical thought is metaphorical and the
authors make a case for even such fundamental branches of mathematics
as axiomatic set theory being metaphorical: for example, the authors
point out that the conceptual metaphor \conmet{sets are containers and
  elements objects in them} is purely metaphorical, yet almost
universally used when thinking about set membership.

Notions such as the number line are also held to be metaphorical:
natural numbers are not points on a line; counting (enumeration) is
not temporal progression along a marked rod; sets are not containers
with elements objects inside them.

\wmcf makes a case, echoing that of~\mycite{lakoff1980}, for many if not
all such metaphors to be rooted in sensory-motor experience.  Here the
most germane is the \metaphor{basic metaphor of infinity} (\bmi) in
which processes that go on indefinitely are conceptualized as having
an end and an ultimate result.  Motivating examples are discussed,
including the one-point compactification of the plane, limits of
sequences, and mathematical induction.

Below I will discuss the relevance of the \bmi to my own teaching,
specifically the limiting behaviour of the binomial distribution to
the Poisson.

It should be pointed out that the book has come under severe criticism
and indeed the ideas have been met with little interest among
mathematicians.  The authors do not put forward any empirical support
whatsoever~\parencite{madden2001} for their assertions about the ways
metaphorical reasoning is used when mathematics is carried out.
\mycite{schiralli2003}, for example, considers the book to make
\dquote{fundamental oversimplifications} and observe that the authors
use the word \emph{metaphor} to serve so many purposes that
\dquote{the notion of metaphor itself begins to lose its meaning}.

The book received at best mixed reviews from both mathematicians and
cognitive scientists.  Neither of the authors is a mathematician (and
certainly no non-elementary mathematics is presented in the book).
The authors present \dquote{misconceptions of mathematics [that] are
  prevalent among non-mathematicians}~\parencite{henderson2002}.
Indeed, many reviewers point to the \dquote{rather frequent}
mathematical errors~\parencite{gold2001}; at many points in the book,
metaphorical reasoning is invoked to explain mathematical cognitive
phenomena, yet a slightly more sophisticated analysis would show an
appropriate mathematical framework.

Nevertheless, as the authors point out, cognitive mathematics is a
sorely neglected field of study; and the book provides a coherent
account of cognition's role in mathematics.

Other aspects of the book are unsatisfactory.  \mycite{goldin2001}, for
example, considers the book to be \dquote{fundamentally flawed} on the
grounds that it was poorly sourced in both cognitive science and
philosophy of mathematical thought.  When mathematicians review the
book, they observe that \wmcf includes \dquote{numerous errors of
  mathematical fact}~\parencite{henderson2002} and also conflates at
least three distinct mathematical activities: learning, using and
research.

The only evidence that the authors adduce for their assertion that
metaphor underlies all mathematical thinking is textual. This, if
nothing else, suggests that it is at least plausible that textual
analysis of the type given in chapter~2 of the current thesis is a
respectable source of information in its own right.

My own reading of \wmcf suggests that the authors appear to be
ignorant of mathematical techniques that render much of their
metaphorical interpretation unnecessary.  For example, in the context
of elementary group theory, the authors give an extended discussion of
what I would call~$C_3$, the cyclic group of three elements.  They
insist that the different examples of this group (plane rotations
by~$2n\pi/3$, arithmetic modulo~$3$, etc) are \dquote{metaphorically
  linked}, and give extensive tables; yet they appear to be ignorant
of the notion of isomorphism, a formal and non-metaphorical concept
that would render their analysis superfluous.

Considering the \bmi, the authors again appear not to have understood
(and certainly have not mentioned) the concept of \emph{Cauchy
  sequence}, which again would render much of their discussion
superfluous.  A sequence~$x_1,x_2,\ldots$ is Cauchy if, for
any~$\epsilon>0$, one can find an integer~$n_0$ such
that~$n,m\geqslant n_0$ implies~$\left|{x_n-x_m}\right|<\epsilon$.  It
is easy to show that a Cauchy sequence approaches a limit
as~$n\longrightarrow\infty$.

Cauchy's startling and elegant definition neatly sidesteps any
confusion between \dquote{actual infinity} and \dquote{realized
  infinity} as the limit itself is not mentioned; observe that no
metaphor is needed.  For this reason, Cauchy sequences are fundamental
to the understanding of many diverse mathematical concepts such as
compactness in Hilbert spaces and completeness of $p$-adic numbers.

From an undergraduate statistics education perspective, the \bmi is
used when considering convergence of random variables.  The example I
will discuss is drawn from the spoken lectures discussed in
Chapter~\ref{chapter5}: the elementary observation that the Poisson
distribution is a limiting case of the binomial.  The formal statement
I am expressing is as follows: \\ \\

{\bf Theorem.}  if~$X_n\sim\operatorname{Bin}\left({n,r/n}\right)$
assuming~$0\leqslant r\leqslant n$, then
  \begin{enumerate}
  \item ${\displaystyle
    \lim_{n\longrightarrow\infty}X_n=X}$ exists, and
  \item $X\sim\operatorname{Poisson}\left({r}\right)$; that
    is~$\operatorname{Pr}\left({X=n}\right) = \frac{e^{-r}{r^n}}{n!}$.
  \end{enumerate}

  \noindent
This fact is neither formally stated, nor any proof given; but the
underlying idea is both simple and important for statistics at this
level.  Observe that the concept of Cauchy sequence is applicable to
probability mass functions just as well for real numbers\footnote{The
  \dquote{distance} between two probability mass functions is simply
  the supremum of the differences between their cumulative
  distribution functions.}.

In this context, \wmcf (Where Mathematics Comes From,
\cite{lakoff2000}) asserts directly that the \bmi is unavoidable in
mathematical language, yet the concept of Cauchy sequence neatly
avoids any need for arguably metaphoric concepts of \dquote{limit} and
\dquote{the infinite}.  The excerpts shown above demonstrate that
careful use of Cauchy sequences can illustrate the concepts of
infinite limits---certainly in the case of Borel probability
measures---without any potentially confusing metaphorical language;
and that such methods are available in a written or spoken context.

\section{Rhetorical \label{we_start} metaphor in textbooks}

Mathematical textbooks, including statistics textbooks, frequently use
rhetorical devices as part of their communication
strategy~\citep{kane1970}.  One such rhetorical device is the use of
\emph{we} which is metaphorical in the sense that the writer is not
using literal language: the reader is identified with a poorly-defined
\dquote{community of practice}, of which the writer is one (perhaps
pre-eminent) example.

\cite{pimm1984} asserts that textbooks' use of \emph{we}
attempts to \dquote{enrol the tacit acquiescence of the reader}, and
serves as an imposition which fails to take into account the wishes or
interests of participating individuals.

This peculiar use of \emph{we} among mathematicians is not limited to
textbooks; it is a ubiquitous construction in research
articles~\parencite{kuo1999}.  Consider, for example, the first
article in the most recent edition at time of writing in \emph{The
  Journal of Topology}\footnote{The discipline of topology is a
  theoretical branch of pure mathematics, notable for its extreme
  abstraction}, \parencite{lange2016}. This is a typical article in
the field, the abstract of which starts \dquote{We characterize finite
  groups~$G$ generated by orthogonal\ldots}; the \emph{we} must be
inclusive because the article is single-author.  \mycite{kuo1999}
considers that the almost complete absence of first-person singular
pronouns (I/me) to be evidence of effort to reduce personal
attributions, and this tendency is presumably operating in textbooks
too.

\subsection{Inclusivity in mathematics}

The mathematicians' \emph{we} is thus an attempt to draw the reader in
to the community of practice.  In this context, \mycite{fauvel1988}
considers the issue of inclusivity in mathematics, drawing a
distinction between the \emph{Euclidean} and \emph{Cartesian} styles
of rhetoric.  \citeauthor{fauvel1988} characterizes the Euclidean
style as follows:

\begin{singlespace}
\begin{quote}
 There is no sign he notices the existence of readers at all.  Rather,
 he seems engaged in laying down inexorable eternal truths.  The
 reader is never addressed.
\end{quote}
\end{singlespace}

\noindent and compares with the Cartesian approach:

\begin{singlespace}
\begin{quote}
The mathematics described is clearly created, not unveiled, in
rhetoric which veers from grabbing the reader by the lapels to
treating you with utter disdain\ldots
\end{quote}
\end{singlespace}

\noindent The three textbooks use the mathematicians' \emph{we}, and
are thus more Cartesian than Euclidean in outlook (at least, if the
inclusive sense is understood).

\subsection{Grammatical inclusivity}

It is interesting to note that English does not distinguish between
inclusive and exclusive \emph{we}\footnote{Inclusive \emph{we}
  specifically includes the addressee while the exclusive form does
  not}, so there is no grammatical way to detect whether the reader is
included in the writer's utterances.  Such considerations can be
important in political speech~\parencite{chen2006}.  Languages such as
Te Reo M\={a}ori do maintain a distinction between inclusive and
exclusive forms (the words are t\={a}tau and m\={a}tau respectively),
so perhaps M\={a}ori textbooks would afford some insight here.

\section{Conclusions}

Metaphor is an unavoidable component of exposition used in statistics
textbooks.  Three statistics textbooks were chosen for detailed study
and their use of metaphor seemed to be broadly similar.

Metaphors such as the mathemticians' \dquote{we} and the basic
metaphor of infinity were frequent.  These metaphors were not
discipline-specific to statistics.  Discipline-specific metaphors
included the basic metaphor of infinity used to illustrate the central
limit theorem.
  % statistics textbooks
  \begin{singlespace}
\begin{savequote}[105mm]
For a very long time, word problems have played
  their role as an unproblematic and transparent bridge between the
  world of mathematics and the real world.
\qauthor{\mycite[page 644]{verschaffel2014}}
\end{savequote}
\end{singlespace}

\chapter{Metaphor in assessment}
\label{chapter6}

\section{Overview}

In this chapter, I consider metaphor in the assessment phase of an
undergraduate statistics course.  Metaphor occurs in both the
assessment cue and the student's response and I consider both
separately below.

Metaphor is common in statistics examination questions, specifically
occurring when (proper) placeholder nouns are used in word problems.

I also briefly discuss students' use of metaphor when being assessed,
considering both controlled assessment (examination) and uncontrolled
(portfolio) assessment items.

\section{Essay-type questions}

Essay-type questions {\em per se} are rare in undergraduate
statistics, with one exception: a relatively open-ended cue to analyze
a dataset using the methods taught in the course, and communicate any
findings.  \mycite{horowitz1986} considers the use of language in such
assessment cues, with a typical exam prompt being along the lines of

\begin{quote}
  \emph{Using the Yanomamo as an example briefly explain Marvin
    Harris's theory of primitive warfare}
\end{quote}
  
\noindent\citeauthor{horowitz1986} went on to identify a number of
\dquote{micro-functions} of such cues which characterize acceptable
responses.  He ordered these micro-functions along an axis from
content-oriented (\dquote{identify the topic}) to form-oriented
(\dquote{specify the length of the essay}).  From the perspective of
undergraduate statistics education, the most germane micro-function
was his number 5: \emph{specify the writer's persona}.  For a typical
written assignment, \mycite{horowitz1989} points out that students
must pretend that the marker is not familiar with the issues
discussed.

However, in the context of undergraduate statistics, a typical
assignment might be to analyze a specific dataset using algebraic and
visual methods.  In this situation, a student need only make the
realistic assumption that the marker has not actually carried out such
analysis.

Is \emph{metaphor} part of this aspect of language use?  I would argue
that typical undergraduate statistics assessments do use metaphoric
language, in several senses.  Firstly, the students generally treat
the dataset provided with the assessment as just one representative of
an ensemble of possible datasets, all of which would elicit identical
statistical analyses: they would perform the same steps if the data
were perturbed slightly.  The data is thus meronymically defined.
Secondly, the student understands that the analysis is not actually
important and the premises of the cue are merely a plausible fiction
which may or may not exist.

\section{Word problems}

A \emph{word problem} is a verbal description of a problem situation
wherein one or more questions are raised, and the answer to which can
be obtained by the application of mathematical operations to numerical
data available in the problem statement~\citep{verschaffel2014}.  Word
problems are \dquote{considered to be an important part of mathematics
  education}~\citep{reusser1997}.

\mycite{gerofsky1996} states that the overwhelming majority of word
problems have three components: firstly, a backstory which establishes
the characters and possibly the location of the putative story; an
information component, in which the information needed to solve the
problem is given; and a question.

However, word problems are a problematic assessment item in terms of
educational value~\citep{gerofsky1996},
transferability~\citep{reusser1997}, and low achievement
rate~\citep{cummins1988}.  Given these concerns, it is somewhat
surprising that \mycite{johnson1976}, in a 166-page book devoted
purely to the solution of word problems, offers not the slightest
motivation for their study.

In the context of statistics education, \mycite{quilici1996} show that
word problems encourage students to attend to surface elements of the
question (such as inclusion of words such as \dquote{compare} which
induce the use of a two-sample $t$-test) at the expense of underlying
structural features.

In the following sections I consider the extent to which metaphor
occurs in this assessment trope.

\subsection{Truth value and word problems}

Metaphor analysis of word problems is not straightforward because a
word problem is semantically ambiguous.  One concept useful in the
analysis of word problems is \dquote{truth value} as originally
defined by Frege in 1891: the truth value is the attribute assigned to
a proposition in respect of its truth or falsehood.  Frege considered
the relation between propositions and truth from a philosophical
perspective; but the relevance of truth value to language encountered
in fiction remains \dquote{problematic}~\citep{lamarque1994}.

In the context of education, \mycite{gerofsky1996} considers the
semantics of word problems and observes that their truth value has no
clear status.  He observes that word problems may be rephrased without
altering their truth value and offers:

\begin{singlespace}
\begin{quote}
Every year (but it has never happened), Stella (there is no Stella)
rents a craft table at a local fun fair (which does not exist). She
has a deal for anyone who buys more than one sweater (we know this to
be false\ldots there are no people, or sweaters, or prices)
\end{quote}
\end{singlespace}

\noindent As \mycite{gerofsky1996} points out, word problems are a
peculiar trope in which one is instructed to pretend that the
background story is true, under (implicit) direction from the writer
of the problem; but simultaneously the competent reader considers the
background story to be irrelevant.  \mycite{reusser1997} contrast the
chimerical text of the backstory with the implicit web of mathematical
relations in the problem itself; they conclude that these two
\dquote{interwoven semiotic worlds} are poorly aligned and largely
irreconcilable.

The entire backstory may thus be considered to be meronymic in the
sense that the one provided is but one representative of many
possible, functionally identical, backstories.

%Representing two interwoven semiotic worlds, the story-like
%description of non-mathematical real-world situations and an implicit
%web of mathematical relations, mathematical word problems are
%considered to be an important part of mathematics education.  Reusser
%1997

%\begin{enumerate}
%\item that \dquote{this} is solvable,
%\item that \dquote{X} can be found,
%\item that the word problem itself contains all the information needed
%  to do this task,
%\item that no information extraneous to the problem may be sought
%  (apart from conventional mathematical operations which likely must
%  be supplied),
%\item that the task can be achieved using the mathematics that the
%  student has access to,
%\item that the problem has been provided to get the student to
%  practice an algorithm recently presented in their math course,
%\item that there is a single correct mathematical interpretatiof the
%  problem,
%\item that there is one right answer,
%\item that the teacher can judge an answer to be correct or incorrect,
%  and especially,
%\item that the problem can be reduced to mathematical form---in fact,
%  that the problem is at heart an arithmetic or algebraic formulation
%  which has been \dquote{dressed up} in words, and that the student's job
%  is to \dquote{undress} it again---to transform the words back into the
%  arithmetic or algebra that the writer was thinking of, then to solve
%  the problem.
%\end{enumerate}



%\mycite{reusser1997}: \dquote{\ldots most students perceive word problem
%  solving as a puzzle-like activity with no grounding in factual
%  real-world structures and with no relation to a goal-directed, more
%  authentic activity of mathematization or realistic mathematical
%  modelling\ldots at the bottom of the critique of the impoverished
%  nature of word problems is the many-faceted issue of probem
%  formulation and problem posing}.

\mycite{boaler2000} describes one enthusiastic student who, meeting word
problems for the first time, was dismayed to find that bringing her
competent, adult-level situational knowledge to bear on the problems
was counterproductive to academic success.  Boaler went on to question
the (practical) competence of the question-setter, although she does
not query the ontological status---or educational value---of the word
problem.

% Direct quote from Boaler 2000, p392:
%Rose describes feeling particularly alienated when \dquote{real world}
%problems were introduced, with which she enthusiastically engaged
%drawing upon her knowledge of the situations described, only to find
%that such knowledge was not allowed, and that engagement with the
%problems involved a step away, rather than towards the real world
%\ldots school children recognize that school mathematics is not a part
%of the world outside school, partly because of the artificiality of
%school problems.
%
%
% Boaler goes on to quote Rose (which I have not tracked down):
%It was obvious to me that many of the questions simply indicated that
%the questioner did not know enough about the craft skills involved in
%real world solutions.  Lawn rollers being pulled up slopes,
%wallpapering rooms by calculating square feet and inches: these were
%tedious and as far as that highly practical child could see, stupid
%

\section{Students' use of metaphor in assessment}

\begin{singlespace}
\epigraph{Essay examination writing is, indeed, a \dquote{display} for
  the purposes of evaluation, a time to show that one has studied
  hard, not that one is especially clever or possessed of broad
  general knowledge}{\mycite{horowitz1986}}
\end{singlespace}

\noindent
The nature of the language used by students in the assessment phase of
a statistics course is problematic: \mycite{horowitz1986} argues that a
student's response to any assessment cue is a perlocutionary act,
specifically one that persuades the reader to proffer an acceptable
grade in a course.

This viewpoint makes the analysis of metaphor difficult, as the
purpose of the writing is not clear.  On the one hand, the student is
expected to create an exposition to convince the reader of the truth
of a proposition, but on the other, the reader is already convinced.

\mycite{read2001} consider the ways in which students develop a
\dquote{voice} and points out that students must master the complex
culture of academic language in order to succeed.  These authors point
out that the conflict between the desire to score a high grade is
counter to the desire for a student to have their own voice: high-GPA
students were reported to have sought out their tutors' viewpoints in
order to write from their perspective.  \mycite{read2001} go on to pose
the rhetorical question: is such writing the students' voice or that
of their tutors?  

In pure mathematical disciplines, essay-type questions are rare but do
exist~\citep{johnson1983}.  However, by far the most common type of
mathematical examination cue requires the student to prove a (given)
statement.  The linguistic status of a student's proof (under
examination conditions) is again problematic.  The definition of
\dquote{proof} is a logically watertight demonstration of the truth of
a statement: all the student has to do is to reproduce an existing
proof.

However, analogously to an undergraduate essay, an examination proof
is intended to have a perlocutionary effect, specifically of inducing
the reader or marker to give an acceptable grade to the student.  A
poor student will (attempt to) reproduce an existing proof, with no
understanding.  This, however, is a very difficult task as without the
cognitive scaffolding that understanding provides, a proof typically
has no discernible structure and is very difficult to memorize.  The
natural way for a more able student to proceed is to convince the
marker that the writer has actually grokked (sic) the proof and can
convey this to the reader.  In this sense, the mathematics examination
is a peculiar form of performance assessment in which one has to
convince the examiner that you have indeed had the flash of insight
which mathematicians call \dquote{proof}.

Under these circumstances, can the student be said to employ metaphor?
\mycite{kyung2004} argue that students constantly employ mathematical
metaphor in class and, as such, identifies the machine metaphor and
the fictive motion metaphor as dominant metaphors in the case of
partial differential equations.

However, in the context of undergraduate statistics, a typical
assignment might be to analyze a specific dataset using algebraic and
visual methods.  In this situation, a student need only make the
realistic assumption that the marker has not actually carried out such
analysis~\citep{horowitz1986}.

%  [The dichotomy between expository and argumentative essay styles] is
%somewhat muddied by the fact that all essay examination writing
%contains an argumentative element, namely the writer's attempt to
%persuade the reader/instructor to proffer an acceptable grade in a
%course.  This intention must be hidden, however\ldots
%because of the social
%convention that states that an expository exam essay is to be written
%as if the reader did not already understand the ideas being
%presented---as if a prompt were really a question---and that an
%argumentative essay is to be written as if the reader did not already
%agree with the thesis---as if a prompt were really part of a Socratic
%dialogue.  Thus, essay examination writing is doubly false, in that
%writers must hide their true intention (to pass the course) behind a
%wall of prose designed to do what has already been done

%\epigraph{an expository exam essay is to be written as if the reader
%  did not already understand the ideas being presented---as if a
%  prompt were really a question---and that an argumentative essay is
%  to be written as if the reader did not already agree with the
%  thesis---as if a prompt were really part of a Socratic dialogue.
%  Thus, essay examination writing is doubly false, in that writers
%  must hide their true intention (to pass the course) behind a wall of
%  prose designed to do what has already been
%  done}{\mycite{horowitz1986}}

\subsection{The British Academic Written English corpus}

Assessed student writing is difficult to study owing to a scarcity of
suitable corpora~\citep{nesi2004}; and those that exist are typically
focused on English as a second language.

One of the few systematic corpora of assessment is the British
Academic Written English corpus, the BAWE~\citep{bawe2016}.  The BAWE
is a collection of student writing from undergraduate to taught
Masters level, restricted to assignments consistent with an upper
second or first class honours degree~\citep{nesi2012}.  The BAWE is
unique in the wide range of disciplines represented.

The BAWE includes samples of statistics assessment.  It is clear from
context (the anonymization protocol redacted the cue) that the student
was a first year undergraduate required to assess the relative merits
of two measures of central tendency: the mode and the median.  I will
analyse these two pieces of work for metaphor using
the~\mycite{pragglejaz2007} protocol.

The first piece of student work was responding to a cue asking to
characterize the median as a measure of central tendency:

\begin{singlespace}
\begin{quote}
\say{Median is useful in this case because it tells us that half the
  sample has more money than this with them and half has less.  It is
  not influenced by outliers as the mean is.  For example if we had a
  very high value such as \pounds 200 this would increase the mean
  greatly so that it is no longer as representative of the sample as
  the median}---Anonymous student, quoted in \mycite{bawe2016}
\end{quote}
\end{singlespace}

\noindent It is possible to analyze this assessed writing using the
\mycite{pragglejaz2007} metaphor identification protocol;

\begin{description}
\item[\squareb{The} median \metaphor{is useful} in this case]\qquad An
  example of the conceptual metaphor \conmet{equations are tools}.  In
  this case, the metaphor is likely to originate in the course itself.
\item[because \metaphor{it tells} us that]\qquad A peculiar agency
  metaphor; the estimator is personified, and in addition given a
  \dquote{voice}.
\item[half the sample has more\ldots and half less.]\qquad No
  metaphor: this is a literal definition of the median.
\item[It is not \metaphor{influenced by outliers}]\qquad This is
  metaphorical: what the student is trying to say (successfully) is
  that the \emph{value} of the median does not change as a result of a
  marginal change to an outlier.  However, she is here using a
  metonym: the topic is the measure of central tendency known as the
  median, but the vehicle is the value of the median.
\item[as the mean is.]\qquad non-metaphorical.
\item[For example if \metaphor{we} had]\qquad Another example of
  Pimm's \dquote{we}
\item[a \metaphor{very high} value such as \pounds 200]\qquad
  Orientational metaphor: \conmet{increase is up}.
\item[this would increase the mean greatly]\qquad no metaphor
\item[so that it is no longer as representative of the sample as the
  median]\qquad no metaphor
\end{description}
  
\noindent
The second piece of assessed work is from the same student considering
the mode of a dataset.

\begin{singlespace}
\begin{quote}
  \say{The mode is of a varying degree of usefulness---it tells us the
    most common value but here this is very low and not representative
    of the sample values as a whole.  If we look at various major
    peaks in the data set then this is a fairly useful tool.  For
    example, here the data is bimodal---there is a split between
    people carrying a relatively large amount of money and a small
    amount.}
\end{quote}
\end{singlespace}

\begin{description}
\item[The mode is of a varying degree of
  \metaphor{usefulness}---]\qquad \conmet{equations are tools}
\item[\metaphor{it tells us} the most common value]\qquad Clear
  personification metaphor.
\item[but here this is very low]\qquad Orientational metaphor
  (\conmet{less is down})
\item[and not representative of the sample values as a whole.]\qquad
  Non-metaphoric.
\item[If \metaphor{we} look at]\qquad Another example of Pimm's
  \dquote{we}
  \item[various major peaks \metaphor{in} the data set] Possible
    conceptual metaphor \conmet{data is a container and features of
      the data objects in it}.  This is a reasonably common metaphor.
\item[then this is a fairly useful tool.]\qquad\mycite{lakoff1980}
  state that explicit use of conceptual metaphors is rare but here is
  an example of \conmet{equations are tools} being used explicitly.
\item[For example, here the data is bimodal---]\qquad Note that
  \conmet{data is a container and features of the data objects in it}
  is not being used here: the bimodality is attributed directly to the
  dataset rather than asserting that the feature is contained inside
  it.
\item[there is a split between people carrying a relatively]{\bf large
  amount of money and a small amount.}\qquad Metaphoric use of
  \metaphor{split}.  Here, \dquote{split} is being used in the sense
  of chasm or divide: the datapoints (here people) are placed on a
  number line (itself a metaphor for the real numbers) and the student
  is observing that there is a region along this line that is sparsely
  populated.

  In this case, there is one metaphor that is conspicuous by its
  absence.  The data comprised a finite number of observations, each
  one being the amount of money carried by a \emph{specific} person in
  the sample.  The student is not metonymically referring to a person
  by the amount of money they were carrying: the student refers
  directly to the split between the people.
\end{description}

\noindent These pieces of assessed writing use metaphor in a routine
and unremarkable manner.  The metaphors used appear to be of the same
general type as used by educators in typical course resources.

\subsection{Measures of central tendency in undergraduate statistics}

Calculating measures of central tendency is a distressingly common
trope in undergraduate statistics education to the extent that one
sees derogatory descriptions of \dquote{mean-median-mode} education.
It is an easy matter to assess whether students can evaluate the
measures by requiring students to calculate the three measures for
different datasets.

From the perspective of a practising statistician, the mean and mode
are simply measures of central tendency of a sample\footnote{In the
  assessed writing above, the student was discussing the sample mean
  and sample mode.  The sample mean is the arithmetic mean of one's
  observations and is useful because it is an estimate of the
  population mean (if defined).  The mode is an altogether more
  problematic measure, having different definitions for continuous and
  discrete distributions.} and do not have \dquote{usefulness}.  The
concept of one measure being more \dquote{representative} than
another is meaningless: the measures summarize a sample in different
ways and illustrate different aspects of the sample.  The situation is
analogous to an engineer asking whether the radius is more
\dquote{representative} of the size of a given a circle than the
diameter.  So in this case the conceptual metaphor \conmet{equations
  are tools} is leading to flawed thinking.

But it should be noted that my comments above---correct as they are
from a theoretical statisticians' perspective---would probably not
attract a passing grade at undergraduate level: as I argue in
section~\ref{foundational_statistics}, deep thinking about any aspect
of inferential statistics inevitably interferes with successful
assessment

\section{Conclusions}

In assessed work, students do employ metaphor when enrolled in
tertiary statistics courses; such prompts might appear in a first year
undergraduate descriptive statistics paper.  However, ethical
considerations mean that there is very little assessed work that may
be analysed for metaphorical language: fully informed consent is
difficult to obtain.

The British Academic Written English corpus is one of the very few
corpuses of assessed undergraduate student writing available, and this
contains two pieces of assessed work which were part of a statistics
course.  The students used the conceptual metaphor \conmet{equations
  are tools} in much the same way as textbooks do.

  % assessment in statistics
  \begin{singlespace}
\begin{savequote}[105mm]
  \begin{minipage}[t]{.45\textwidth}\raggedright
    I really like Dr X's teaching style.  He's really clear and
willing to help you if you need it.  He's also pretty funny.  Totally
recommend!\end{minipage}\hfill\begin{minipage}[t]{.45\textwidth}\raggedright
    I hated him.  He is the worst teacher on the planet.
And his jokes are not as funny as he thinks they are
\end{minipage}\\ \rule{0mm}{8mm}---Typical
positive and negative comments on the \emph{Rate my professor}
website.
\end{savequote}
\end{singlespace}

\chapter{Metaphor in course evaluation}
\label{chapter7}

\section{Overview}

Many institutions mandate polling of students' opinions on the quality
of instruction.  Such polling is almost universally administered by
questionnaires distributed to students at or close to the end of a
course~\citep{wachtel1998}.  Reasons given for the collection of
student evaluation include instructional improvement, and as a
personnel or management tool (student evaluation is an input to
promotion and tenure decisions).

One aspect of student evaluation that is often overlooked is student
\emph{comments}: written responses to open-ended
questions~\citep{stewart2015}.  In this chapter, I discuss students'
use of metaphor in the free-form comments section of course
evaluation.

\section{Introduction}
The overwhelming majority of research into student evaluation focuses
on \emph{ratings} or \emph{scores}, that is, quantitative assessments
attributed to various aspects of a course~\citep{stewart2015}.  A
typical rating might be a point on a five- or seven- point Likert
scale, given in response to a cue such as \dquote{The course was
  interesting and stimulating}.

There are two situations in which metaphorical language may be used in
course evaluation: the cues used in the questionnaire design; and the
comments made by students in response to open-ended questions.

The cues used in questionnaires do not appear to have been studied in
detail~\citep{aleamoni1980} but if AUT evaluation are typical, the
salient component of the process is requesting students to give a
Likert score to various prompts.  The Likert cues are generally
literal---and non-metaphorical---noun phrases: \dquote{appropriate
  workload}; \dquote{overall experience}; \dquote{good teaching}.

Cues for comment were, in contrast, phrased as questions: \dquote{What
  were the best aspects of this paper?}; \dquote{What aspects of this
  paper are most in need of improvement?} and these were again devoid
of anything but the most lexicalised metaphor.

\section{Students' comments in course evaluation}

The only aspect of student evaluation in which metaphor may be used by
students is the in the \emph{comments} section: written responses to
open-ended questions~\citep{stewart2015}.  Despite the extensive
research literature on student \emph{ratings}, comparatively little is
known about the quality of data obtained from students' written
comments, their content, and the relationship between them and other
variables.

Student comments are time-consuming to interpret, as standard
descriptive and inferential techniques cannot be directly applied to
free-form text.

One over-riding difficulty in the study of students' written comments
is ethical: evaluation usually operates under strict confidentiality
guarantees.  \mycite{stewart2015}, for example, considers only an
aggregated corpus of comments, being prevented from analysing
individual responses by ethical concerns.

The few studies which have been reported in the literature tend to
focus on whether the comments are positive or negative.
\mycite{alhija2009}, considering the frequency with which students'
feedback includes comments, reports that the response rate lies
between~10\% and~70\% and indicates that such a wide range reveals the
\dquote{relatively small body of research} in this area (p38).

\mycite{braskamp1981} is one of the few reports on comments \emph{per
  se}; these authors report close agreement between written comments
and objective responses.  Nevertheless, comments can be informative in
ways not possible for ratings.  \mycite{tucker2014}, considering
students' evaluation comments, provides a comprehensive review of
students' use of language in this aspect of a course, and observes
that student comments can provide \dquote{valuable insights} into the
student experience.  To what extent does metaphor play a part in this
aspect of education?

\mycite{alhija2009} consider open-form requests for comment and report
a range of student responses.  The students appear to use metaphor
very sparingly, with the majority being lexicalized (\dquote{time was
  wasted}; \conmet{time is a resource}) or standard educational
metaphors (\dquote{we never got to the end}, \conmet{a course is a
  journey}).  Such metaphors are strikingly similar to those used in
documents such as course descriptors (discussed in
section~\ref{course_descriptor_discussion}).

\section{Conclusions}

Course evaluation typically includes a free-form comments section in
which students are invited to give feedback on a course.  This section
is the only section in which metaphor occurs.

It is difficult to study students' use of metaphor in this context and
only a limited amount of research has been carried out to date.
Students appear to use metaphor sparingly, and those that are used are
similar to the metaphors used in institutional documents such as
course descriptors.
  % students written work and course evaluation
  \chapter{Conclusions}
\label{chapter8}

In this thesis, I have considered metaphorical language as used in
various aspects of an undergraduate statistics course.  Conceptual
metaphor, whether viewed as a cognitive or a linguistic phenomenon,
appears in each of these aspects to a greater or lesser extent.

Metaphor is such a ubiquitous phenomenon that it is easy to overlook,
but its influence is deep and pervasive.  Conceptual metaphors such as
\conmet{the mind is a container} directly influence educational
practice, not always for the better: for the objects \metaphor{in} the
mind frequently include misconceptions, gaps, and fallacies.  And
surely if the mind is a container and \conmet{concepts are objects},
then this directly implies that the measurement model is the correct
way to assess these objects.

Taking the \metaphor{level} metaphor as another example, this
metaphorical system structures thinking about education and encourages
perceptions about mathematics (such that mathematical knowledge is
organised into a discrete set of standalone building blocks possessing
a natural sequence) that are an educational policy dream but a
pedagogical nightmare.

Statistics courses share many points of similarity with mathematics
courses which have been more widely and thoroughly studied; it is not
clear to what extent statistics should be viewed as a branch of
mathematics.

Statistics as a discipline uses standard mathematical metaphors such
as the \conmet{basic metaphor of infinity} but this does not have the
central role that it has in (for example) analysis or topology.  

The \metaphor{agency} and \metaphor{personification} metaphors occur
frequently in course descriptors: a statistics course is implicitly
embued with the ability to act independently and intelligently.  Such
language encourages the perception of the course content itself as
active and personified. 

Metaphor is a component of spoken lectures, statistics textbooks, and
undergraduate assessment.  Its use is comparable in frequency and
intensity to that in mainstream mathematics, and similar metaphorical
constructions are found.


%
%
%\epigraph{Just over a year ago, on a visit to one of the world's most
%  prestigious research institutes, I challenged researchers there to
%  account for intelligent human behaviour without reference to any
%  aspect of [the Information Processing metaphor].  \emph{They
%    couldn't do it}, and when I politely raised the issue in
%  subsequent email communications, they still had nothing to offer
%  months later.  They saw the problem.  They didn’t dismiss the
%  challenge as trivial.  But they couldn’t offer an alternative.  In
%  other words, the IP metaphor is \dquote{sticky}.  It encumbers our
%  thinking with language and ideas that are so powerful we have
%  trouble thinking around them\ldots [T]he idea that humans must be
%  information processors just because computers are information
%  processors is just plain silly, and when, some day, the IP metaphor
%  is finally abandoned, it will almost certainly be seen that way by
%  historians, just as we now view the hydraulic and mechanical
%  metaphors to be silly.}{\mycite{epstein2016}}
%
  % conclusions

  \begin{singlespace}
  \printbibliography
  \end{singlespace}

%----------------------------------------------------------------------------------------
%	THESIS CONTENT - APPENDICES
%----------------------------------------------------------------------------------------

%\appendix % Cue to tell LaTeX that the following "chapters" are Appendices

% Include the appendices of the thesis as separate files from the Appendices folder
% Uncomment the lines as you write the Appendices

%\include{appendices/AppendixA}
%\include{Appendices/AppendixB}
%\include{Appendices/AppendixC}

%----------------------------------------------------------------------------------------
%	BIBLIOGRAPHY
%----------------------------------------------------------------------------------------

\printbibliography[heading=bibintoc]

%----------------------------------------------------------------------------------------

\end{document}  
